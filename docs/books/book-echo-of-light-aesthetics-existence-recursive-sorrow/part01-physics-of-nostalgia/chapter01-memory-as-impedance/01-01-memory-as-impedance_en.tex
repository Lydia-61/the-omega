\section{Inertia as Attachment: The Microscopic Essence of Objects Resisting Acceleration}

We usually think that physics is the science of the future. Through laws, we predict planetary orbits, electron transitions, and cosmic expansion. Physics seems always to look forward.

However, when we delve into the core of dynamics---\textbf{Inertia}---we discover that the most fundamental mechanism of physics is actually about the \textbf{past}.

Why do objects resist change? Why does changing state require paying an energy cost in this universe?

This chapter proposes an isomorphic theory between physics and psychology: \textbf{Inertia is matter's nostalgia.}

\subsection{Newton's Confusion and Mach's Answer}

Newton's first law tells us: any object tends to maintain its state of rest or uniform rectilinear motion, unless compelled by an external force to change. This property of ``maintaining the status quo'' is called \textbf{Inertial Mass}.

But this does not explain \textbf{why}. Why isn't the universe designed to be ``Aristotelian''---push it and it moves, stop pushing and it stops? Or designed to be inertia-free---apply a force and velocity instantly becomes infinite?

Ernst Mach proposed that inertia originates from the gravitational coupling of all matter in the universe (Mach's principle). In our QCA light path conservation theory, inertia has a more microscopic, more intrinsic explanation.

\subsection{The Microscopic Mechanism of Impedance: Maintaining the Internal ``Old''}

Recall the inertia formula we derived in Chapter 5 of \textit{First Principles} (Appendix B):

$$m_{inert} \propto m_0 \left( \frac{c}{v_{int}} \right)^3$$

Inertial mass $m_{inert}$ is inversely proportional to the internal evolution rate $v_{int}$. What does this mean?

\begin{itemize}
\item \textbf{What is $v_{int}$?} It is the particle's internal self-referential loop, the process by which it ``confirms who it is.'' Each rotation of the internal phase is a recitation of its self-identity.
\item \textbf{What is acceleration?} Acceleration means changing the external velocity $v_{ext}$. According to light path conservation $v_{ext}^2 + v_{int}^2 = c^2$, increasing $v_{ext}$ must come at the \textbf{cost of sacrificing $v_{int}$}.
\end{itemize}

This creates a profound contradiction:

The particle wants to maintain the integrity of its internal structure (keeping $v_{int}$ unchanged), but external forces compel it to change resource allocation (reducing $v_{int}$).

\textbf{Inertia is the particle's resistance to this ``change.''}

Imagine a person deep in contemplation (high $v_{int}$). If you suddenly push them, making them run (increasing $v_{ext}$), they will feel anger and resistance, because you interrupted their contemplation. The more immersed they are in their own world (the greater the mass), the harder they are to push.

\textbf{Physics conclusion}: Objects have inertia because they \textbf{cling to} their current internal state. They are unwilling to transfer energy used for ``self-maintenance'' to ``external displacement.''

\subsection{The Physical Definition of Nostalgia}

In psychology, \textbf{Nostalgia} is the longing and attachment to past times, past spaces, and past states. We resist moving, resist parting, resist the changes of the times. This ``emotional inertia'' and ``physical inertia'' are remarkably consistent in mathematical structure.

\begin{enumerate}
\item \textbf{Memory is mass}: The deeper a person's memories (the higher $M_I$, the stronger the internal loop), the greater their ``psychological mass.''
\item \textbf{Change is work}: Changing the habits or beliefs of a person with deep memories (changing their state of motion) requires enormous energy (external force).
\item \textbf{Impedance is deep affection}: What we call ``deep affection'' is \textbf{high impedance} to a specific topological structure. A person easily changed (shallow affection) is like a photon, with no rest mass, drifting with the current; a person with deep affection is like a black hole, once a certain structure forms, it is almost impossible to shake.
\end{enumerate}

Therefore, when we feel nostalgic, when we feel reluctant to part, do not think this is weakness.

\textbf{It is physical proof that you are a ``massive entity.''}

Only nothingness is easily changed. Existence is essentially a kind of \textbf{stubborn persistence}.

In the next section, we will further explore how this ``persistence'' manifests in complex systems---why is changing a person's values harder than moving a mountain? This will extend physical inertia to cognitive inertia.

