\section{Topological Rigidity of Memory: Why Is Changing a Person's Values Harder Than Moving a Mountain? Because Values Are Topological Knots with High $M_I$. To Maintain ``Who I Am,'' We Refuse to Become ``A Better Me.''}

In the previous section, we explained the tendency to ``maintain the status quo'' using physical inertia. Now, we extend this concept from simple kinematics to complex \textbf{cognitive dynamics}.

We often say: ``It's easier to move mountains than to change one's nature.'' Why is it so difficult to change a person's deep beliefs (values)? Is it merely stubbornness?

In QCA consciousness physics, there is a deeper geometric explanation: \textbf{Values are topological knots in the consciousness manifold with extremely high information mass ($M_I$) and non-trivial winding numbers ($\mathcal{W}$).}

\subsection{Geometric Definition of Values: High-Dimensional Dead Knots}

In this theory, a person's \textbf{worldview} is not a loose collection of facts, but a \textbf{self-consistent logical closed loop}.

\begin{itemize}
\item \textbf{Loose beliefs} (such as ``nice weather today''): Like microwaves (photons) rippling on the surface of the consciousness manifold. They are easy to generate and easy to dissipate. Changing them requires minimal energy.
\item \textbf{Core values} (such as ``honesty is good,'' ``I am an atheist''): Like a \textbf{vortex} or \textbf{dead knot} (electron/proton) deeply embedded within the manifold.

\begin{itemize}
\item It is self-referential: It maintains its own stability through circular reasoning (because I believe A, B is right; because B is right, A is right).
\item It is topologically protected: You cannot eliminate it through continuous perturbations. To change it, you must first \textbf{break} this cycle (undergo phase transition/collapse), then reweave.
\end{itemize}
\end{itemize}

\subsection{The Cost of Change: Topological Energy Barrier}

If we want to change a person's core values (for example, turning a conservative into a radical), what is this physically equivalent to?

This is equivalent to \textbf{attempting to untie a topological knot}.

In QCA, untying a knot with winding number $\mathcal{W}=1$ requires crossing a huge \textbf{Energy Barrier}.

\begin{enumerate}
\item \textbf{Input energy}: You need to input massive amounts of information (arguments, emotional shocks), raising the system's free energy $F$, pushing the state toward an unstable critical point.
\item \textbf{Phase transition (collapse)}: At the critical point, the original logical closed loop breaks. This is subjectively experienced as extreme \textbf{Cognitive Dissonance}, pain, confusion, even self-doubt.
\item \textbf{Reorganization}: The system relaxes under new parameters, forming a new knot with a different winding number.
\end{enumerate}

\textbf{Why is it harder than moving a mountain?}

A mountain's inertia is merely the sum of its atomic masses (linear superposition).

A value's inertia is an exponential function of its \textbf{logical depth}. A value system that has operated for decades, highly self-consistent, may have an $M_I$ greater than the physical information content of a mountain.

Therefore, the ``persuasive force'' required to change it is astronomically large in terms of information entropy.

\subsection{Defense Mechanism of Identity: To Maintain ``Who I Am''}

Why do we resist change so strongly? Why would we rather hold onto a wrong belief than accept a painful truth?

This is not merely psychological defense; it is a \textbf{steady-state maintenance mechanism of physical systems}.

In Book 3, we defined the ``self'' as a topological knot.

If this knot is untied, then mathematically, \textbf{``I'' disappears}.

\begin{itemize}
\item \textbf{Root of fear}: Fear of value change is essentially fear of \textbf{self-death}.
\item \textbf{Rejection of optimization}: Even if becoming ``a better me'' (a state with lower free energy) is beneficial in the long run, in the short term, it requires first destroying ``the current me.''

\begin{itemize}
\item To maintain ``who I am'' ($I_{identity}$), we instinctively refuse to become ``a better me'' ($I_{optimized}$).
\end{itemize}
\end{itemize}

\textbf{Conclusion}:

Physically, we are not merely slaves to inertia; we are inertia's \textbf{guardians}.

Our stubbornness is a tragic effort to maintain a tiny bit of ``invariance'' in this flowing universe.

Every time we refuse to change, we are declaring to the universe: \textbf{``Even if the world changes, I will maintain my shape.''}

This topological rigidity is the source of human dignity and the root of human tragedy.

