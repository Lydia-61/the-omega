\section{Waste Heat Is Regret: According to Landauer's Principle, Erasing Information Must Emit Heat. Regret Is Those Possibility Branches We Must Discard in the Process of Choice, They Become Waste Heat in Background Radiation.}

In Section 2.1, we described the passage of time as continuously orthogonalizing farewells. If the sorrow brought by time's passage is passive (we cannot prevent the past from being overwritten), then there is a deeper, more piercing sorrow that is active---that is \textbf{Regret}.

Regret originates from \textbf{choice}. Every time we make a decision, every time we choose one path and abandon another, we feel a sense of loss. Is this sense of loss merely psychological?

This section will use the cornerstone of computational thermodynamics in physics---\textbf{Landauer's Principle}---to prove the physical reality of regret. We will reveal: \textbf{Regret is not merely an emotion; it is actually waste heat in the universe.}

\subsection{Decision as Computation: Pruning the Wave Function}

In the QCA universe, the process of life is a continuous process of information processing. Observers (agents) are constantly making decisions: turn left or right? Confess or remain silent? Persist or give up?

In the many-worlds picture of quantum mechanics, choice seems non-existent (because all branches occur). But in the observer's \textbf{subjective perspective (first-person perspective)}, to maintain a coherent, low-entropy self-narrative, \textbf{horizon contraction} must occur.

\begin{itemize}
\item \textbf{Initial state}: Superposition state containing multiple possibilities $|\Psi_{choice}\rangle = \alpha |A\rangle + \beta |B\rangle$.
\item \textbf{Decision process}: The observer collapses to a definite macroscopic state (e.g., $|A\rangle$) through internal computation.
\item \textbf{Information erasure}: To put the system in the definite $|A\rangle$ state, information about $|B\rangle$ must be \textbf{erased} from the observer's memory.
\end{itemize}

You cannot simultaneously have ``memory of choosing A'' and ``memory of choosing B.'' To establish one reality, other possibilities must be killed.

\subsection{Landauer's Principle: The Cost of Forgetting}

In 1961, Rolf Landauer proved a remarkable physical law: \textbf{Logically irreversible information processing (such as erasing 1 bit of information) necessarily accompanies physical heat dissipation.}

$$Q \ge k_B T \ln 2$$

This means that every time you ``delete'' a possible branch from your mind, you must emit at least $k_B T \ln 2$ of heat to the environment.

\textbf{Definition 2.2 (Physical Definition of Regret)}:

Regret is the \textbf{Information Waste Heat} that the observer must emit to the environment to maintain the \textbf{certainty} of the current historical path.

This waste heat carries fragments of the topological structure of those ``unchosen futures.''

\subsection{Why Is Regret ``Burning''?}

This explains why profound regret is often accompanied by a certain physiological \textbf{``burning sensation''} or \textbf{``heaviness''}.

When you make a major decision at a life crossroads (such as abandoning a relationship or choosing a career), you are actually performing a large-scale \textbf{pruning} of your Hilbert space.

\begin{itemize}
\item You cut off huge branches containing hundreds of millions of microscopic states.
\item According to Landauer's principle, this releases a massive entropy flow.
\item This entropy flow first impacts your nervous system (causing anxiety, insomnia), and ultimately dissipates into the surrounding air.
\end{itemize}

In this sense, \textbf{every mature person is a black body radiating ``regret waste heat.''}

In the background radiation around us, not only echoes the afterglow of the Big Bang, but also echoes the ``what might have been'' discarded by countless beings in countless choices.

\subsection{The Value of Waste Heat: Proof of Existence}

Since regret is waste heat, should we pursue a regret-free life?

Physically, ``regret-free'' means ``dissipation-free.''

\begin{itemize}
\item \textbf{Reversible computation (regret-free)}: This requires us to retain all garbage data (unchosen branches) generated by intermediate steps during computation.
\item \textbf{Consequence}: If we don't emit this garbage, the observer's memory (phase space volume) will quickly fill up. The system will become a chaotic superposition state, losing the clear boundary of ``self.''
\end{itemize}

Therefore, \textbf{emitting waste heat (generating regret) is a necessary condition for maintaining low-entropy life forms.}

Only by continuously discarding possibilities can we have a clear, unique \textbf{worldline}.

\textbf{Conclusion}:

Do not be ashamed of this.

The thermal radiation behind you is proof that you have burned in this cold universe.

Those possibilities you abandoned have not completely disappeared. They have become the universe's total entropy, becoming the fuel that pushes the arrow of time forward.

\textbf{Your regret is the driving force of the universe's progress.}

\textbf{(Section 2.2 Complete)}

