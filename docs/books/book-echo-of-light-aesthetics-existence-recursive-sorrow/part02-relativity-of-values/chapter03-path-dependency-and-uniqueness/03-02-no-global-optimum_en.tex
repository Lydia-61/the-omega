\section{No Global Optimum: Optimization Problems in Complex Systems Are Multimodal. What Is ``Optimal'' for Observer A (Following Their Gradient) May Be a ``Disaster'' for Observer B (Against Their Gradient).}

In the previous section, we established the ``path dependency'' of personal identity. If each person is a unique worldline, how do we evaluate the ``goodness'' of these paths? Society and culture often attempt to impose a single evaluation standard (such as wealth, status, IQ), implying the existence of a global optimum toward which everyone should converge.

This section will falsify the existence of a ``single optimal solution'' from the perspectives of \textbf{Free Energy Landscape} and \textbf{Complex Systems Dynamics}. We will prove: \textbf{In a sufficiently complex system, optimization objectives are necessarily multimodal.} Each local minimum (Attractor) represents an effective survival strategy, and the incommensurability between different attractors is the physical root of value conflicts.

\subsection{Complex Topology of Free Energy Landscape}

Recall Chapter 8, observer dynamics are driven by minimizing variational free energy $F$.

For simple physical systems (such as harmonic oscillators), the potential energy surface has only one lowest point (the bottom of the parabola). All trajectories eventually converge there. This is \textbf{Convex Optimization}.

However, for observers with complex internal models $\mathcal{M}$, their free energy function $F(\mathcal{M}, S)$ is an extremely rugged surface defined on a high-dimensional parameter space.

\begin{itemize}
\item \textbf{Non-convexity}: There exist countless \textbf{local minima}.
\item \textbf{High dimensionality}: The parameter space has extremely high dimensions (neural connections $\sim 10^{14}$), meaning it is almost impossible to traverse all states to find the global minimum.
\end{itemize}

\textbf{Definition 3.2 (Value Attractor)}:

Each local minimum is a \textbf{stable behavioral pattern} or \textbf{value system}. Once the system falls into it, surrounding free energy barriers prevent it from easily leaving.

For example:

\begin{itemize}
\item \textbf{Attractor A (Adventurous)}: High risk, high reward, high information throughput.
\item \textbf{Attractor B (Stable)}: Low risk, low reward, low energy consumption.
\end{itemize}

\subsection{Relativity of Gradients: One's Meat Is Another's Poison}

Each observer attempts to move along the \textbf{negative gradient} of free energy $-\nabla F$ (this is the direction of ``happiness'').

However, due to the landscape's complexity, gradient directions differ drastically at different positions.

Consider two observers Alice and Bob, located in two different attractor basins.

\begin{itemize}
\item \textbf{Alice's optimal direction}: Sliding down her basin, which may mean ``pursuing change.''
\item \textbf{Bob's optimal direction}: Sliding down his basin, which may mean ``pursuing stability.''
\end{itemize}

If we forcibly pull Alice onto Bob's path (social conditioning), or Bob tries to imitate Alice (blind following):

\begin{itemize}
\item For Bob, this is \textbf{counter-gradient movement}.
\item \textbf{Physical consequence}: Free energy $F$ rises sharply (prediction error increases).
\item \textbf{Psychological experience}: Anxiety, pain, self-doubt.
\end{itemize}

\textbf{Conclusion}:

There is no ``objectively better'' lifestyle.

\textbf{``Good'' is the inner product between the gradient vector and your current position vector.} If your position changes, the optimal direction also changes.

\subsection{Nash Equilibrium and Niche Differentiation}

What happens if everyone competes to occupy the same ``global optimum'' (assuming it exists, e.g., becoming a billionaire)?

Under Red Queen dynamics, competitive pressure (dissipation) near that position increases exponentially, causing the \textbf{effective free energy} at that position to rise instead (because survival becomes harder).

\textbf{Evolutionarily Stable Strategy (ESS)}:

The system automatically differentiates into multiple niches.

\begin{itemize}
\item Some become lions (predator strategy), some become antelopes (escape strategy), some become fungi (decomposer strategy).
\item In the lion's value system, ``killing'' is optimal; in the antelope's value system, ``alertness'' is optimal.
\end{itemize}

\textbf{Physical Corollary}:

The diversity of social values (pluralism) is not a moral tolerance, but a requirement of \textbf{thermodynamic stability}.

If the entire society has only one value system (unimodal), the system becomes extremely unstable (everyone engaged in vicious competition in the same dimension), ultimately leading to collapse (deadlock or oscillation).

\textbf{Only multimodal distribution can maximize the entire civilization's information processing capacity.}

\subsection*{Conclusion: Respect Others' Gradients}

This physical picture provides a foundation for tolerance in ethics:

When we see others' choices differ from ours, don't rush to judge ``they're right'' or ``I'm wrong.''

We are simply in different valleys of the free energy landscape.

\begin{itemize}
\item Their ``decadence'' may be the ``optimal solution'' in their local region.
\item My ``progress'' may appear to them as ``futile counter-movement.''
\end{itemize}

\textbf{True wisdom is not forcing everyone to climb the same mountain, but understanding the scenery of each mountain.}

\textbf{(Section 3.2 Complete)}

