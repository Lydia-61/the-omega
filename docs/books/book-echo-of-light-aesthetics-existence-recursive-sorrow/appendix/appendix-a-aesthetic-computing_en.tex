\appendix
\chapter{Aesthetic Computing --- A Metric for Beauty based on Kolmogorov Complexity}

In Chapters 4 and 8 of this book, we proposed that ``aesthetics is a heuristic guide to truth.'' This appendix aims to provide a quantitative mathematical framework explaining why certain structures (such as fractals, physical laws, artworks) are judged as ``beautiful'' by consciousness networks.

\section{The Dilemma of Aesthetic Measurement}

Traditional Shannon Entropy cannot measure beauty.

\begin{itemize}
\item \textbf{Crystal (low entropy)}: $S \to 0$. Completely ordered, but dull.

\item \textbf{White noise (high entropy)}: $S \to max$. Completely random, but meaningless.
\end{itemize}

Beauty seems to exist in a critical region between ``order'' and ``disorder.''

\section{Birkhoff-Bennett Formula}

George Birkhoff once proposed $M = O/C$ (aesthetic measure = order/complexity). In computation theory, we upgrade this to a formula based on \textbf{Kolmogorov Complexity ($K$)} and \textbf{Logical Depth ($D$)}.

Let object $X$'s description be a binary string.

\begin{itemize}
\item \textbf{$K(X)$}: Length of the shortest program generating $X$ (compressed information content).

\item \textbf{$D(X)$}: Number of logical steps required to run that shortest program to output $X$ (computation time).
\end{itemize}

\textbf{Definition A.1 (Aesthetic Value Function $\mathcal{A}$)}:

$$\mathcal{A}(X) = \frac{\mathcal{D}(X)}{K(X)} \times \text{Resonance}(X, \mathcal{M}_{observer})$$

\begin{enumerate}
\item \textbf{Simplicity Benefit ($1/K$)}:

Our brains prefer \textbf{high compression ratios}. If a complex phenomenon $X$ can be explained by a short law (like $F=ma$), the brain saves enormous storage energy, producing a sense of ``elegance.''

\begin{itemize}
\item \textit{Example}: Fractals are beautiful because the code generating them $z \to z^2+c$ is extremely short ($K$ small), but the generated images are infinitely rich.
\end{itemize}

\item \textbf{Profundity Benefit ($D$)}:

If an object has short code but an extremely trivial decompression process (like printing a million ``A''s), it is boring.

Only when the decompression process involves \textbf{non-trivial computation} (like life evolution, story development) does it have \textbf{logical depth}.

\begin{itemize}
\item \textit{Example}: A Taihu stone weathered over hundreds of millions of years, its form contains a long history of fluid dynamics computation ($D$ large).
\end{itemize}

\item \textbf{Resonance Correction}:

The $\text{Resonance}$ term depends on the observer's internal model $\mathcal{M}$. Aesthetics are activated only when object $X$'s topological structure undergoes \textbf{Homology} with the observer's mental structure.
\end{enumerate}

\textbf{Conclusion}:

\textbf{Beauty = Encapsulating the longest history with the fewest bits.}

