\section{5.1 The Revelation of Weber-Fechner}

\begin{quote}
``Why in childhood, a summer vacation feels as long as a lifetime; while in old age, ten years pass like sand through fingers? This is not just psychology; this is geometry. Our consciousness is a logarithmic ruler; it compresses not only light and sound, but also time itself. It is precisely this compression that allows us to maintain a linear dignity when facing an exponentially exploding universe.''
\end{quote}

\subsection{The Mathematical Signature of Physiology}

19th-century psychophysicists Weber and Fechner discovered an iron law governing all senses:

\[S = k \ln I\]

Sensory intensity ($S$) is the logarithm of physical stimulus intensity ($I$).

\begin{itemize}
\item Decibels of hearing.

\item Magnitude of vision.

\item Pressure of touch.
\end{itemize}

They all follow this law.

In the third book of \textbf{Vector Cosmology}, we interpreted this as ``information decoding.'' But from the \textbf{``axiology''} perspective of the fifth book, this law reveals life's ultimate defense against \textbf{``time inflation''}.

\subsection{Countering Exponential Explosion: Fighting Poison with Poison}

Recall our evolution equation: The universe's total budget (speed of light/computational power) grows exponentially with intrinsic time $\tau$:

\[I(\tau) = c(\tau) \propto e^{\lambda \tau}\]

If our perception were linear ($S \propto I$), we would face a disaster:

\begin{itemize}
\item \textbf{At birth}: The world is slow.

\item \textbf{In adulthood}: The world is 10 times faster.

\item \textbf{In old age}: The world is 100 times faster. We would be unable to think or react, because the frequency of external events exceeds our processing bandwidth.
\end{itemize}

But evolution (or that higher-dimensional us) was extremely clever in choosing \textbf{logarithmic perception}.

We put the exponentially growing physical input $I(\tau)$ into the logarithmic function $\ln(\cdot)$:

\[S(\tau) = k \ln(e^{\lambda \tau}) = k \lambda \tau\]

\textbf{A miracle occurs.}

The exponential ($e$) and logarithm ($\ln$) cancel each other out.

That violently accelerating physical time is restored in our subjective consciousness to a \textbf{smooth, uniform, linearly flowing river}.

We feel that ``every second is equal in length'' because we are using a constantly shrinking ``psychological ruler'' to measure a constantly expanding ``physical universe.''

This is the highest manifestation of \textbf{life's homeostasis} in the time dimension.

\subsection{The Relativity of Time Density}

This also explains why \textbf{``new experiences''} can lengthen life.

Weber's law has another corollary: \textbf{Just Noticeable Difference (JND)} is proportional to the current total.

$\Delta I \propto I$.

This means that the more you have (the larger $I$), you need \textbf{more} increment ($\Delta I$) to feel ``change.''

Mapping to the sense of time:

\begin{itemize}
\item \textbf{Childhood}: Your total memory ($I$) is small. Each new day ($\Delta I$) accounts for a large proportion relative to the total. The increment $\ln(I + \Delta I) - \ln(I)$ is significant. So time is \textbf{slow}, \textbf{thick}.

\item \textbf{Adulthood}: Your total memory ($I$) is large. The same day ($\Delta I$) thrown into the ocean creates no waves. The logarithmic increment approaches zero. So time \textbf{flies}.
\end{itemize}

\textbf{Conclusion:}

Physical time is fair to everyone (Chronos).

But \textbf{Logarithmic Time} is different for everyone.

If you want to ``live long,'' extending physical lifespan through health preservation is only a lower strategy.

\textbf{The upper strategy is ``resetting the base.''}

Explore unknown fields, learn completely new skills, love strangers.

These actions create \textbf{heterogeneous $v_{int}$} that cannot be simply added to the old $I$; they open new logarithmic coordinate axes.

\textbf{Only innovation can reverse the accelerated depreciation of psychological time.}

\subsection{Conclusion: We Are the Deceleration Glass of Time}

The Weber-Fechner law is not just a mechanism of senses; it is the \textbf{shield of consciousness}.

The universe ($e$) tries to throw us out, tries to tear us apart with exponential acceleration.

And we ($\ln$) create an \textbf{``inertial reference frame''} by bending our perception.

In this reference frame, we don't feel the universe's gallop. We only feel gentle breeze and peaceful years.

We are \textbf{the deceleration glass of time}.

Through this glass, we filter that bizarre, ever-changing high-energy universe outside into a warm, gentle \textbf{``human world''} suitable for human habitation.

Since our perception is logarithmic, what properties will the logarithmic function exhibit when we approach that infinitely distant endpoint ($\Omega$ point)?

Does it imply that for subjective consciousness, \textbf{the endpoint can never be reached}?

This leads to the theme of the next section: \textbf{Zeno's Eternity}. We will see why, in the logic of logarithms, life is an endless flight.

