\section{Chapter 6: Dynamic Block Universe}

In the previous chapter, we stretched the length of subjective time through the logarithmic law, but this did not solve the question of where objective time goes. When the clock's second hand jumps to the next tick, where did the previous second go?

In common sense, the past is like sand slipping through fingers, disappearing into nothingness after falling. But in physics' \textbf{Block Universe} theory, past, present, and future exist simultaneously like a huge block of ice. This view guarantees eternity but sacrifices freedom (everything is predetermined).

\textbf{Vector Cosmology} proposes a compromise, more revolutionary model: \textbf{Dynamic Block Universe}.

The universe is indeed a growing crystal.

\begin{itemize}
\item \textbf{Future} is liquid (wave function superposition state), full of possibilities.

\item \textbf{Present} is the crystallization frontier (collapse interface), full of energy release.

\item \textbf{Past} is solid (geometric topology), full of definite structures.
\end{itemize}

This chapter will reveal that history is not smoke blown away by wind, but \textbf{immortal textures} carved on four-dimensional manifolds.

