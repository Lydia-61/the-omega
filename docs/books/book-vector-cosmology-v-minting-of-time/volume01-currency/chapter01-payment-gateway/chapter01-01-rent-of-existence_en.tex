\section{1.1 The Rent of Existence}

\begin{quote}
``There is no free `I.' You think you own your body, you own your consciousness. But in reality, you are only renting them. Every heartbeat is a rent payment you make to the thermodynamics counter. Once you stop paying, your structure will be reclaimed by the universe and melted into background noise.''
\end{quote}

\subsection{The Maintenance Fee of Structure}

In classical physics, objects seem to have a kind of ``eternality.'' A stone placed there, if left untouched, seems to remain a stone forever.

But from the microscopic perspective of \textbf{FS geometry}, maintaining a complex \textbf{$v_{int}$ (internal structure)} requires continuous consumption of \textbf{$c_{FS}$ budget}.

Recall our core formula:

\[v_{ext}^2 + v_{int}^2 + v_{env}^2 = c_{FS}^2\]

\begin{itemize}
\item \textbf{$v_{int}$} is your asset (your body, memories, order).

\item \textbf{$v_{env}$} is your liability (environmental erosion, thermal noise, forgetting).
\end{itemize}

The second law of thermodynamics tells us that the universe's natural tendency is to transform $v_{int}$ into $v_{env}$.

Like a sandcastle on the beach, the waves (environmental noise) constantly want to flatten it.

To keep the castle (you) standing, you must continuously do work---that is, continuously allocate funds from the total $c_{FS}$ budget and inject them into the $v_{int}$ sector to offset the pull of $v_{env}$.

This is \textbf{``the rent of existence''}.

Living is not a static property; living is an \textbf{expensive dynamic equilibrium}.

You must pay energy (ATP) and information (error correction) every moment to barely maintain the geometric shape of ``I am I'' from collapsing.

\subsection{The Physical Essence of Aging: Inflation}

This explains why we \textbf{age}.

Biologically, aging is telomere wear or mitochondrial damage.

In vector economics, aging is \textbf{``rising rent''} and \textbf{``declining purchasing power''}.

As intrinsic time $\tau$ progresses, although you are still paying rent, the \textbf{transaction fees ($v_{env}$ loss)} of the transaction are getting higher and higher.

\begin{itemize}
\item \textbf{When young}: Your metabolic system is very efficient. You pay 1 unit of $c_{FS}$ and can buy back 0.9 units of $v_{int}$ (tissue repair).

\item \textbf{When old}: Your system has accumulated too many microscopic damages (geometric wrinkles). You pay 1 unit of $c_{FS}$ and can only buy back 0.5 units of $v_{int}$. The remaining 0.5 all become waste heat.
\end{itemize}

\textbf{Aging is not the passage of time; aging is the exponential rise of ``maintenance costs.''}

When one day, all the energy you consume (eating/breathing) is insufficient to pay the minimum rent required to maintain your structure, the system will trigger the \textbf{``bankruptcy liquidation''} procedure.

This is \textbf{death}.

\subsection{The Subscription Universe}

This perspective, though harsh, is extremely clear.

We are not shareholders of the universe; we are \textbf{tenants} of the universe.

\begin{itemize}
\item We rent \textbf{atoms} to build our bodies.

\item We rent \textbf{photons} to obtain energy.

\item We rent \textbf{logic} to run our thoughts.
\end{itemize}

This tenancy relationship has a mandatory clause: \textbf{``Pay-as-you-go''}.

The universe does not accept credit. You must pay for your current existence with current actions (breathing, metabolism, thinking).

This brings us a profound sense of \textbf{urgency}.

Since every second is paid for, we cannot waste it.

If we spend expensive $c_{FS}$ budget on meaningless daydreaming, internal friction, or repetition, we are \textbf{``burning money''}. We are paying high rent but leaving the house vacant.

\textbf{Conclusion:}

Time is not free air.

Time is every coin that is burning.

Realizing this is the first step to becoming a \textbf{``time minter''}.

Since we know that every second is ``spending money,'' where exactly are we spending it? How do we transform flowing budget into fixed history through this transaction?

This leads to the theme of the next section: \textbf{The Moment of Transaction}. We will see that the real ``sense of time'' does not come from the ticking of clocks, but from \textbf{the frequency of transactions}.

