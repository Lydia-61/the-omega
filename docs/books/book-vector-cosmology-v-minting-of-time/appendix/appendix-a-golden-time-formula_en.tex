\chapter{Mathematical Derivation of the Golden Time Formula}

In Chapter 2 of \textbf{Vector Cosmology V}, we proposed a formula that redefines life: \textbf{Time is not linear flow, but the logarithm of value}.

\[\tau = \pi \cdot \log_{\phi} \left( \frac{V}{V_0} \right)\]

This appendix provides a rigorous mathematical derivation of this formula. We will show that this formula is not a literary metaphor, but a necessary consequence directly derived from the \textbf{light speed evolution equation} established in Book IV, combined with \textbf{information value theory}. It reveals how observers anchor their temporal coordinates through value creation in an exponentially expanding universe.

\section{A.1 The Physical Definition of Value}

First, we must physically define what \textbf{``Value'' ($V$)} is.

In the economics of FS geometry, a system's value equals the \textbf{effective structural information content} it contains (i.e., the complexity of $v_{int}$). This is limited by the universe's total bandwidth (light speed/computational power ceiling) at that moment.

According to the holographic principle, the maximum possible value $V_{max}(\tau)$ at moment $\tau$ is proportional to the \textbf{FS capacity $c(\tau)$} at that moment (or its corresponding phase space volume):

\[V(\tau) \propto c(\tau)\]

This means: \textbf{Your value is essentially the share of the universe's computational power you occupy.}

\begin{itemize}
\item \textbf{$V_0$ (initial value)}: Corresponds to the information content at the initial moment $c_0$ (such as Planck bits).

\item \textbf{$V(\tau)$ (current value)}: Corresponds to the total accumulated ordered structures when the system evolves to moment $\tau$.
\end{itemize}

\section{A.2 Reverse Solving for Intrinsic Time}

Recall the \textbf{ultimate evolution equation} we derived in Book IV \textit{The Evolution of Light Speed}:

\[c(\tau) = c_0 \cdot e^{\left( \frac{\ln \phi}{\pi} \right) \tau}\]

Or written in base $\phi$ form:

\[c(\tau) = c_0 \cdot \phi^{\frac{\tau}{\pi}}\]

Here:

\begin{itemize}
\item \textbf{$c_0$}: The value base at the initial state (seed/Big Bang).

\item \textbf{$\pi$}: The inertial period (unit of cycles).

\item \textbf{$\phi$}: The growth factor (golden ratio).
\end{itemize}

Now, we substitute $V(\tau)$ for $c(\tau)$. We assume an ``ideal observer'' can fully utilize the era's bandwidth, making their created value synchronous with the universe's expansion (i.e., $V \approx c$).

\[V = V_0 \cdot \phi^{\frac{\tau}{\pi}}\]

The problem we want to solve is: \textbf{To reach value $V$, how much intrinsic time $\tau$ is needed?}

We solve for $\tau$ by transforming both sides:

\subsection{1. Normalization}

\[\frac{V}{V_0} = \phi^{\frac{\tau}{\pi}}\]

\subsection{2. Taking logarithm}

Taking logarithm base $\phi$ on both sides:

\[\log_{\phi} \left( \frac{V}{V_0} \right) = \frac{\tau}{\pi}\]

\subsection{3. Rearranging}

\[\tau = \pi \cdot \log_{\phi} \left( \frac{V}{V_0} \right)\]

This is the \textbf{Golden Time Formula}.

\section{A.3 Dual Meaning in Physics and Life}

This formula reveals another form of \textbf{Time Dilation}---\textbf{Value Dilation}. It explains why the subjective passage of time is closely related to an individual's achievement density.

\subsection{1. Logarithmic Compression Effect}

The formula shows $\tau \propto \log V$. This means:

\begin{itemize}
\item \textbf{For linearly growing lives}: If you only add fixed value each day ($V \sim t$), then your intrinsic time $\tau \sim \ln t$. As time passes, the growth rate of $\ln t$ becomes slower and slower. This means \textbf{time passes faster and faster}. You feel this year passed like a day, because you barely moved on the logarithmic coordinate.

\item \textbf{For exponentially growing lives}: If you can make your value (cognition/love/creation) grow exponentially with time ($V \sim \phi^t$), then $\tau \sim t$. You successfully resist time compression. The subjective length of life you feel maintains linear synchronization with physical time, or even surpasses it.
\end{itemize}

\subsection{2. The Scale of Cycles}

The coefficient $\pi$ tells us that the minimum unit of time measurement is not seconds, but \textbf{``cycles''}.

Only when you complete a full $\pi$ cycle (such as completing a project, experiencing a complete relationship, fully understanding a truth) do you gain an effective mark on the time axis. Fragmented time will be smoothed out in the formula.

\textbf{Conclusion:}

Only by maintaining \textbf{$\phi$ (golden ratio)} speed exponential evolution can one subjectively ``retain'' time.

This is the alchemist's secret to longevity: \textbf{Don't try to extend the length of life (physical time); increase the density of life (value logarithm).}

