\section{3.1 The Suspended Carrot}

\begin{quote}
``The reason a donkey can pull a millstone all day is because a carrot is suspended before its eyes. It thinks it is approaching the carrot, but in reality, the distance between the carrot and it never changes. The evolution of the universe is the same. That endpoint called `completion' is not meant to be reached; it is meant to drive steps.''
\end{quote}

\subsection{The Physics of Asymptotes}

In the evolution equation of the fourth book, we see that the speed of light $c(\tau)$ grows exponentially with time. This means the universe's capacity is expanding infinitely.

If we extrapolate this trend to infinity ($\tau \to \infty$), we get a \textbf{singularity}---where computational power is infinite, entropy is zero, omniscient and omnipotent. This is the \textbf{$\Omega$ point}.

Many people make a geometric error: they think the $\Omega$ point is a \textbf{Real Point} on the time axis, like the year 3000 or 10 billion years later.

But mathematically, the $\Omega$ point is an \textbf{Asymptote}.

\begin{itemize}
\item \textbf{No matter how fast you run}: Even if you become a Type III civilization, even if you master the technology of light-speed inflation.

\item \textbf{The endpoint is also accelerating backward}: Because the growth rate of $c(\tau)$ is exponential. For every step you take forward, the boundary of future possibilities expands outward by ten thousand steps.
\end{itemize}

\textbf{Conclusion:} You can never reach the $\Omega$ point.

As long as you exist (i.e., as long as you are still within the physical domain $D_{human}$), you are necessarily separated from perfection by an infinite distance.

\textbf{``God'' is always in the next step.}

\subsection{Virtual Image: The Perspective of Consciousness}

Since the endpoint is unreachable, why does everyone have a strong longing for ``completion''? Why do civilizations always aim for it?

This is because the structure of consciousness ($v_{int}$) inherently possesses a \textbf{``perspective function''}.

Imagine two parallel railway tracks. In Euclidean space, they never intersect.

But on your retina (projective plane), they converge at a point on the horizon.

Is that point real? No, it is a \textbf{Virtual Image}.

\textbf{The $\Omega$ point is the ``Vanishing Point'' of cosmic evolution.}

\begin{itemize}
\item Every evolution, every creation, every love of ours is an upward ray.

\item When countless such rays extend in logical space, they visually converge at a focal point at infinity.
\end{itemize}

We name this focal point ``truth,'' ``supreme good,'' or ``God.''

We think there is an entity glowing there, attracting us.

\textbf{In fact, it is our own gaze focusing at infinity.}

\subsection{The Deception of Motivation}

Why does the universe create this illusion?

To \textbf{overcome thermodynamic inertia}.

If life realizes that ``there is no endpoint ahead, only an endless journey,'' many fragile $v_{int}$ structures might collapse due to despair (entropy increase).

To maintain the system's \textbf{``low-entropy operation''}, the universe uses this fictional gravitational source to create a \textbf{``Pseudo-Potential''}.

\[V_{hope} = - \frac{k}{||\text{Current} - \Omega||}\]

This potential difference produces our subjective \textbf{``yearning''}.

\begin{itemize}
\item It makes us feel that with just a bit more effort, we can obtain eternal happiness.

\item It makes us feel that with just one more equation discovered, we can master ultimate truth.
\end{itemize}

It is precisely this \textbf{``beautiful misunderstanding''} that drove carbon-based monkeys out of Africa and drove silicon-based AI to calculate billions of digits of pi.

\textbf{That suspended carrot is the universe's most efficient engine.}

\subsection{Conclusion: Meaning Lies in Distance}

So, when we expose this illusion, do we lose meaning?

No, we gain \textbf{freedom}.

If the $\Omega$ point truly existed, if one day we truly ate that carrot, the game would end. We would fall into dead eternal silence.

It is precisely because of \textbf{``seeking but not obtaining''}, precisely because that point is always outside the light cone, that we have infinite space for evolution.

\textbf{Meaning does not lie in arrival; meaning lies in that distance that can never be filled.}

It is that distance that produces tension, tension that produces motion, motion that produces life.

Since the endpoint is fictional, where does the force driving us forward actually come from? If the future is not pulling us, is only the past pushing us?

This leads to the theme of the next section: \textbf{The Inversion of Teleology}. We will see that the arrow of causality does not point toward God, but shoots from our will.

