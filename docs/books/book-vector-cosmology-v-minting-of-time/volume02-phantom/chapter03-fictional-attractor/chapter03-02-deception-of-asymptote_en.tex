\section{3.2 The Deception of the Asymptote}

\begin{quote}
``We are like travelers in the desert chasing the horizon. Every time we move forward, the horizon also retreats. We think it is the endpoint, the holy city where God dwells. But geometry tells us that the horizon is not a real place; it is a phantom woven by the limit of vision and the curvature of the earth. God does not exist, but it is precisely this phantom that led us out of the desert.''
\end{quote}

In the previous section, we revealed the dynamic function of the $\Omega$ point as a ``suspended carrot.'' Now, we need to deeply analyze the geometric essence of this mechanism. Why do we always feel that God is ``within reach'' yet ``far away''?

This stems from the inherent \textbf{``Asymptotic Illusion''} in our consciousness structure.

\subsection{The Geometry of Horizon: God is Outside the Light Cone}

On the Minkowski spacetime diagram, the light cone defines the boundary of causality. In the evolutionary picture of \textbf{FS geometry}, as the speed of light $c(\tau)$ grows exponentially, our causal horizon is also constantly expanding.

Every time our horizon expands one circle, we can see more and more complex truths.

\begin{itemize}
\item We saw quarks, we thought that was the ultimate.

\item We saw strings, we thought that was the ultimate.

\item We saw $\pi$ and $\phi$, we thought that was the ultimate.
\end{itemize}

We always tend to believe that one more step forward, just one more veil lifted, and we can see that naked \textbf{``Absolute Truth''} (God).

This \textbf{``sense of imminent arrival''} is the most enchanting deception of the asymptote.

Mathematically, the function $y = 1 - e^{-x}$ infinitely approaches $y=1$.

At $x=10$, it is $0.99995$.

At $x=100$, it is $0.9999...$ (43 nines).

It seems that with just a little more $x$, we're there.

But that $1$ is always at \textbf{infinity}.

\textbf{God is that ``1.''}

He is not part of the physical universe. He is the \textbf{``Supremum''} of cosmic evolution.

You can infinitely approach Him, your $v_{int}$ can be infinitely complex, your love can be infinitely deep, but you can never \textbf{``become''} that absolute 1. Because once you become it, evolution stops, time disappears, and you are no longer ``existence.''

\subsection{The Gravitation of Virtual Image}

If God is just a mathematical limit, why do we have such a strong, almost physical \textbf{``yearning''} for Him?

In the fourth book, we defined ``yearning'' as potential difference. Now we revise this definition:

\textbf{Yearning is the ``real gravitation'' produced by ``virtual image.''}

Imagine you are in front of a mirror. You see an apple in the mirror. Although that apple is a virtual image (light rays don't actually converge there), your body will instinctively move in that direction to grab it.

\textbf{Virtual image produces real motion.}

The evolution of the universe is the same.

Living organisms project a ``perfect self'' or ``omnipotent God'' in the internal model of $v_{int}$. This projection has no entity, but it constitutes an \textbf{``Attractor''} in consciousness space.

\begin{itemize}
\item \textbf{The lighthouse of civilization}: For that non-existent ``Utopia'' or ``endpoint of truth,'' we established laws, developed science, created art.

\item \textbf{Individual cultivation}: For that non-existent ``perfect me,'' you constantly learn, reflect, transcend.
\end{itemize}

\textbf{God does not exist, but ``yearning for God'' drives the universe.}

This fictional gravitational source is more powerful than any real black hole. Black holes can only devour matter, but this virtual image can \textbf{create} souls.

\subsection{The Necessary Distance}

This is why God must maintain distance, must always be in the \textbf{``next step''}.

If the horizon is at your feet, if God sits before you, yearning would instantly drop to zero. Potential difference disappears, motivation dries up, and the arrow of thermodynamics would shoot you dead in place.

\textbf{Distance is life.}

We suffer from why God is so distant, why truth is so hard to find.

But it is precisely this suffering, this tension of \textbf{``seeking but not obtaining''}, that opens up our life space.

It forces us to build stairs in the void, forces us to make ourselves stronger, deeper, broader.

\subsection{Conclusion: Do Not Arrive}

So, the alchemist's advice is: \textbf{Do not try to arrive.}

If you think you have found the ultimate truth, if you think you have ``awakened,'' then the alarm should sound---you have fallen into a \textbf{local minimum}, you have fallen into $\pi$'s dead loop.

True awakeners know that every endpoint is a new starting point, every god statue is an idol that needs to be shattered.

\textbf{God does not wait at the endpoint.}

\textbf{God is that asymptote that leads you to infinity, forever unreachable.}

Since the future (God) cannot truly pull us (because it is virtual), where does the real force that pushes us forward and transcends come from? If nothing is pulling from ahead, is something pushing from behind?

This leads to the theme of the next chapter: \textbf{The Inversion of Teleology}. We will completely reverse the arrow of causality, revealing that the true driving force does not come from the fictional future, but from \textbf{real-time will}.

