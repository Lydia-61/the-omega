\section{Appendix C: Selected Bibliography}

\begin{quote}
\textbf\{Note\}: Although this book is a theoretical work, its core arguments are built upon solid foundations of cutting-edge research in physics, information theory, and cognitive science. The following literature provides original mathematical proofs and experimental evidence for key concepts covered in the book (such as QCA, holographic principle, IIT, free energy principle). For readers wishing to explore technical details in depth, these works are essential paths into the depths of the "rabbit hole."
\end{quote}

\subsubsection{1. Physics Foundations \& Cosmology}

\begin{itemize}
\item 't Hooft, G. (2016).** \textit{The Cellular Automaton Interpretation of Quantum Mechanics}. Springer.

\item Relevant Sections**: Foreword, 1.1, 3.3

\item Core Contribution**: Foundational work by Nobel laureate Gerard 't Hooft, arguing that quantum mechanics' probabilistic nature may originate from underlying deterministic cellular automaton (QCA) evolution, providing the physics foundation for this book's "discrete ontology."

\item Susskind, L. (1995).** "The World as a Hologram". \textit{Journal of Mathematical Physics}, 36(11), 6377-6396.

\item Relevant Sections**: 2.2, 3.2

\item Core Contribution**: Foundational work on the holographic principle. Argues that physical information in three-dimensional space is completely encoded on two-dimensional boundaries, providing mathematical support for the view that "physical world is a projection of information."

\item Maldacena, J., \& Susskind, L. (2013).** "Cool horizons for entangled black holes". \textit{Fortschritte der Physik}, 61(9), 781-811.

\item Relevant Sections**: 3.2, A.2.2

\item Core Contribution\textbf{: Proposed the famous }ER=EPR conjecture**, proving the geometric equivalence of quantum entanglement (EPR) and spacetime wormholes (ER), which is the core basis for this book's view of "love/entanglement" as the source of gravity.

\item Bekenstein, J. D. (1981).** "Universal upper bound on the entropy-to-energy ratio for bounded systems". \textit{Physical Review D}, 23(2), 287.

\item Relevant Sections**: 10.2, A.3.1

\item Core Contribution**: Bekenstein Bound. Proved the upper limit of information storage in finite space, deriving the physicality of information.
\end{itemize}

\subsubsection{2. Consciousness, Information \& Complexity}

\begin{itemize}
\item Tononi, G. (2004).** "An information integration theory of consciousness". \textit{BMC Neuroscience}, 5(1), 42.

\item Relevant Sections**: 6.1, A.1.1

\item Core Contribution**: Original paper on Integrated Information Theory (IIT). Defined $\Phi$ value as a measure of consciousness level, supporting this book's panpsychist position that "consciousness is an intrinsic property of systems."

\item Friston, K. (2010).** "The free-energy principle: a unified brain theory?". \textit{Nature Reviews Neuroscience}, 11(2), 127-138.

\item Relevant Sections**: 4.1, A.1.1

\item Core Contribution**: Proposed the free energy principle. Explained how living organisms resist entropy increase by minimizing prediction error (variational free energy), which is the theoretical source for this book's view of "pain as error signal."

\item Bennett, C. H. (1982).** "The thermodynamics of computation—a review". \textit{International Journal of Theoretical Physics}, 21(12), 905-940.

\item Relevant Sections**: 7.2, 10.2

\item Core Contribution**: Connected Landauer's Principle with Maxwell's Demon, clarifying the equivalence between information erasure and thermodynamic work, which is the physical foundation for this book's "anti-entropy technology."
\end{itemize}

\subsubsection{3. Computation, Logic \& Mathematics}

\begin{itemize}
\item Turing, A. M. (1936).** "On Computable Numbers, with an Application to the Entscheidungsproblem". \textit{Proceedings of the London Mathematical Society}, 2(42), 230-265.

\item Relevant Sections**: 11.2

\item Core Contribution**: Definition of Turing machines and undecidability of the halting problem. Supports this book's assertion that "cosmic algorithm never halts."

\item Gödel, K. (1931).** "Über formal unentscheidbare Sätze der Principia Mathematica und verwandter Systeme I". \textit{Monatshefte für Mathematik und Physik}, 38, 173-198.

\item Relevant Sections**: 11.2

\item Core Contribution**: Incompleteness theorems. Proved that truth always exceeds provability, providing logical guarantee for "open-ended finale" and "infinite truth."

\item Lloyd, S. (2000).** "Ultimate physical limits to computation". \textit{Nature}, 406(6799), 1047-1054.

\item Relevant Sections**: 3.3, 10.2

\item Core Contribution**: Calculated the maximum computational power limit of the universe as a computer, quantifying physical evolution as logical operations.
\end{itemize}

\subsubsection{4. Philosophy \& Theology}

\begin{itemize}
\item Teilhard de Chardin, P. (1955).** \textit{The Phenomenon of Man}. Harper \& Row.

\item Relevant Sections**: 6.2, 10.1

\item Core Contribution\textbf{: Proposed the concept of }Omega Point**, that the universe is evolving toward an extremely complex super-consciousness state, which is the core inspiration for this book's view on the ultimate destination of civilizations.

\item Spinoza, B. (1677).** \textit{Ethics}.

\item Relevant Sections**: 0.2, 9.1

\item Core Contribution**: Spinoza's pantheism (Deus sive Natura), that God equals nature/universe as a whole.

\item Spencer-Brown, G. (1969).** \textit{Laws of Form}. Allen \& Unwin.

\item Relevant Sections**: 2.1

\item Core Contribution**: Proposed "distinction" as the starting point of cognition, which is the logical prototype for this book's view of "Big Bang as dissociation."

\bigskip
\hrule
\bigskip

\item (End of Book)**
\end{itemize}

