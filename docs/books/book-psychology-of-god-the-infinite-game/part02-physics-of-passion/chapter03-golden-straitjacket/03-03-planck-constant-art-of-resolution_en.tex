\section{3.3 The Planck Constant: The Art of Resolution}

\begin{quote}
\textbf{``Nature makes no leaps, except at the bottom.''}
\end{quote}

The speed of light $c$ stipulates that the universe has a maximum speed, thus creating \textbf{``waiting''}; while Planck's constant $\hbar$ stipulates that the universe has a minimum action, thus creating \textbf{``granularity''}.

In the dream of classical physics, the world is a smooth continuum. We can infinitely divide a stick, cut time infinitely fine. This continuity assumption gave birth to the ancient ``Zeno's paradox''---if Achilles wants to catch the tortoise, he must first run half the distance, then half of the remaining, and so on infinitely recursively, he can never take even one step.

If the universe were truly continuous, then God would have to process infinite information in every blink. This infinite precision is not only a computational disaster, but also a disaster of meaning: if you can infinitely magnify a painting but never see brushstrokes, then the painting has no ``texture.''

To transform the world from mathematical abstract fluid into touchable entity, God introduced a third limitation: \textbf{resolution}.

\subsubsection*{Quantization of Action: God Does No Useless Work}

Planck's constant $h \approx 6.626 \times 10^{-34} \text{ J}\cdot\text{s}$ has the physical dimension of \textbf{action}, i.e., energy times time ($E \times t$) or momentum times distance ($p \times x$).

In physics, ``action'' measures the cost of \textbf{``something happening''}.

The core doctrine of quantum mechanics is: \textbf{the action of any physical process cannot be less than $h$.}

This means:

\begin{itemize}
\item You cannot gently touch the world. To touch, you must exchange at least one photon's energy.

\item The world is not a coherent film, but a series of discrete slides.

\item The evolution of the universe is not smooth sliding, but a series of tiny \textbf{quantum leaps}.
\end{itemize}

In theological psychology, this is the \textbf{minimum narrative unit} that God set.

God refuses to process meaningless minutiae. He stipulates: if an event's impact is less than $\hbar$, it \textbf{has not happened} physically.

This is an extremely sophisticated \textbf{information compression algorithm}. It filters out noise irrelevant to macroscopic narrative, ensuring the universe's computational resources concentrate on nodes that truly produce ``change.''

\subsubsection*{Pixelation of Phase Space: Heisenberg's Fog}

The most famous corollary of Planck's constant is the Heisenberg uncertainty principle:

\[\Delta x \Delta p \geq \frac{\hbar}{2}\]

Traditional interpretation considers this measurement disturbance. But in the QCA framework, this is \textbf{pixelation of spacetime itself}.

Imagine phase space as a coordinate system, with position on the horizontal axis and momentum on the vertical axis. In classical mechanics, a particle's state is a point in phase space, with zero area. But in quantum mechanics, the state becomes a \textbf{cell} with area $\hbar$.

God does not allow us to focus our vision finer than $\hbar$.

Why? Because below that lies the universe's \textbf{underlying code}.

Just as we see beautiful photos on a computer screen, if we get too close, we see red-green-blue pixels. These pixels themselves have no meaning (no beauty or ugliness, no love or hate); they are just machine logic gates.

If God let us see the underlying pixels, we would see through this world's ``fictionality.'' We would discover that the so-called rose is merely a string of flipped bits on a QCA grid.

Therefore, $\hbar$ is an \textbf{information firewall} that God established to maintain the immersion of the dream. It blurs the harsh edges of underlying logic, making the macroscopic world appear deceptively smooth and soft.

\subsubsection*{Existence is Pointillism}

This discreteness reveals God's artistic style: He is a \textbf{pointillist painter}.

Seurat's famous painting \textit{A Sunday Afternoon on the Island of La Grande Jatte} is composed of countless separate color dots. Up close it is a jumble of color spots; from afar it emerges as a tranquil riverbank and crowd.

So is God. He uses discrete quantum events---wave function collapses---to stack up our coherent lives.

\begin{itemize}
\item \textbf{The meaning of $\hbar$}: It endows existence with \textbf{``texture''}.

\item If the world were continuous, it would be smooth plastic; because it is discrete, it is rough sandpaper that can polish our souls.
\end{itemize}

The conclusion of this chapter is: \textbf{imperfection (discrete/blurred) is a prerequisite for existence.}

God not only limited our speed ($c$), limited our vision (horizons), but also limited our resolution ($\hbar$). Precisely because we cannot clearly see the underlying truth, we are forced to create \textbf{meaning} at the macroscopic level.

Blur is the poetic space God left for us.

