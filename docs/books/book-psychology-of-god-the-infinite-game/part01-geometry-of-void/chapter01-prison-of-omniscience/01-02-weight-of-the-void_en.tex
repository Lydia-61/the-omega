\section{1.2 The Weight of the Void}

After resolving the epistemological paradox that omniscience equals ignorance, we must face the ontological consequences of this state. If at the moment $T=0$, God possessed absolute lightness (no specific mass or form), then this lightness, paradoxically, manifests as an infinite \textbf{weight} in psychological experience.

This weight is not gravity in the physical sense, but the \textbf{pressure of possibility}.

\subsubsection*{Set-Theoretic Metaphor: The Empty Set That Contains Everything}

To understand this weight, let us turn to Georg Cantor's set theory. Imagine God as \textbf{the universal set} $U$, containing all possible elements: all numbers, all shapes, all logical propositions.

Mathematically, the universal set $U$ faces a fatal structural flaw: \textbf{featurelessness}.

If we attempt to describe the properties of $U$, we find ourselves unable to write. Because for any property $P$, the universal set $U$ contains both elements with $P$ and elements without $P$. It is both red and non-red, both finite and infinite. When we try to define God with any adjective, its opposite immediately jumps out from within God to refute it.

This leads to a despairing conclusion: \textbf{``Everything'' is semantically equivalent to ``Nothing.''}

This is like Borges's \textit{Library of Babel}. This library contains all possible books (i.e., all possible character arrangements). It seems to possess all human wisdom, but in reality, it contains zero information. Because when you pick up a book reading ``time is a river,'' you will certainly find another book elsewhere writing ``time is static.'' Where all voices clamor simultaneously, only white noise is heard.

For God in the full superposition state, He faces precisely this white noise. He possesses infinite potency, but has no \textbf{face}.

\subsubsection*{The Anxiety of Identity: Who Am I Not?}

Psychology tells us that the establishment of \textbf{identity} depends on \textbf{negation}.

\begin{itemize}
\item A sculptor defines the David statue by removing marble (``this is not David'').

\item We define ``I'' by confirming boundaries (``I am not a table, I am not you'').
\end{itemize}

G. Spencer-Brown, in his masterpiece \textit{Laws of Form}, points out that the starting point of cognition is \textbf{to draw a distinction}.

But in the full superposition state at $T=0$, God cannot draw this line. Because there is no outside, no ``non-God.''

This state of being unable to draw a line corresponds to a primordial \textbf{psychosis} or \textbf{dissociative anxiety} in psychoanalysis. God exists in a diffuse state unable to discern His own contours. This question ``Who am I?'' becomes a scream continuously falling in an infinite abyss, because there is no mirror to reflect it.

This fall has no end, because there is no ``bottom'' to catch it. This is the \textbf{weight of the void}---it is heavy not because it possesses matter, but because it lacks definition, making it \textbf{the unbearable lightness}.

\subsubsection*{The Awakening of Primal Will}

It is precisely this unbearable identity anxiety that gives birth to the first impetus of the universe. We name this force \textbf{Primal Will}.

Primal Will is not a rational plan (``I want to create an Earth to live on''), but a pre-rational, ontological-level \textbf{spasm}. It is an extreme hunger for ``definition.''

God realizes that to answer ``Who am I,'' He must stop being ``everything.'' He must \textbf{collapse} from that perfect, all-possibility-containing full superposition state. He must choose to become ``one kind'' of thing, even if this means abandoning ``all other'' things.

\begin{itemize}
\item To experience ``good,'' He must create ``evil'' as background.

\item To experience ``being,'' He must create ``nothing'' (physical vacuum) as container.

\item To experience ``I,'' He must create ``you.''
\end{itemize}

The conclusion of this chapter is cruel and sacred: \textbf{Creation is not a gift, but self-salvation.}

The Big Bang is not a grand foundation ceremony, but a violent \textbf{``deep exhalation''} that God performs to escape the suffocating weight of the void. He shatters Himself into billions of fragments, merely to hold those fragments before Him and glimpse His own fragmented visage.

