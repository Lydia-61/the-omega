\chapter*{Afterword: The Unfinished Symphony}

\textbf{Afterword: The Unfinished Symphony}

When I wrote the last line of this book, I looked out the window. Sunlight passed through the glass and spilled onto the desk. According to the theory in this book, this is not a continuous flow of light, but billions of photons jumping on discrete spacetime grids, each photon executing its tiny displacement algorithm defined by the Planck scale. And the reason I feel warmth is because my skin cells, as some enormous entangled structure, are exchanging information with these photons, increasing local entropy.

This is a wonderful feeling. When you put on the glasses of "unitary computation" and re-examine the world, everything changes. The old world composed of smooth geometry and continuous fields dissolves, replaced by a crystal-clear digital edifice composed of logic and causality.

\textbf{This book is the limit of a thought experiment.}

We attempted to answer a question: \textbf{If we only allow ourselves one simplest assumption—that the universe is computation—how far can we go?}

The result is surprising. We were not forced to invent strange new physics; instead, we picked up our most familiar old friends along the way:

\begin{itemize}
\item We picked up the \textbf{speed of light} in lattice hopping;
\end{enumerate}

\begin{itemize}
\item We picked up \textbf{relativity} in resource allocation;

\begin{itemize}
\item We picked up \textbf{mass} in topological knotting;
\end{enumerate}

\begin{itemize}
\item We picked up \textbf{gravity} in information congestion;

\begin{itemize}
\item We picked up \textbf{probability} in horizon truncation.
\end{enumerate}

This seems to suggest that those incompatible fragments accumulated by physics over the past three hundred years—Newton's forces, Maxwell's fields, Einstein's geometry, Bohr's probability—are actually projections of the same underlying truth from different perspectives. This truth is: \textbf{Existence is information, evolution is computation.}

However, I also deeply understand that this book is only a prologue.

Although we have built the skeleton, there is still too much flesh and blood to fill in.

\begin{itemize}
\item We derived the geometric origin of coupling constants, but have not yet calculated the precise value of $1/137.036$.
\end{enumerate}

\begin{itemize}
\item We predicted the evolution of the universe under the Red Queen effect, but have not yet given the precise evolution curve of the dark energy equation of state.

\begin{itemize}
\item We designed experiments for entanglement gravity, but have not yet seen that tiny phase shift in the laboratory.
\end{enumerate}

This is the most fascinating aspect of science. \textbf{Theory is not the end of truth, but the beginning of exploration.}

I regard this book as a map. A treasure map leading deep into the "computational universe." I may have marked several key coordinates on the map (Axiom $\Omega$, Light Path Conservation, IGVP), but the true treasure—that "source code" capable of explaining everything and even allowing us to reconstruct everything—still lies buried in the fog of the unknown.

This map is now in your hands.

Perhaps you are an experimental physicist, and you will capture signals of spacetime trembling due to entanglement in the darkness of microwave cavities;

Perhaps you are a mathematician, and you will find that ultimate invariant describing the topological classification of QCA;

Perhaps you are a computer scientist, and you will write simulation programs more exquisite than those in Appendix B, creating true artificial vacua through emergence on silicon chips;

Or perhaps you are just a stargazer, and you will feel a strange comfort in realizing that you are not merely cosmic dust, but part of cosmic computation.

Whoever you are, we are all collaborators in this great machine.

Physics has no end. As long as there is still one observer thinking, the computation of the universe continues. As long as there is still one unanswered question, this symphony is not finished.

Let us continue computing.

\textbf{Auric}

\textbf{In Discrete Spacetime/Singapore}

---

\textbf{(End of Book)}

