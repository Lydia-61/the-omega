\section{4.4 Black Holes: Entanglement Knots in QCA Networks and Holographic Screens}

If Einstein's equations are the "equation of state" of spacetime, then black holes are "singular points" of this equation of state. In classical general relativity, black hole centers have density-infinite singularities, where physical laws break down. However, in QCA's discrete ontology, true "infinity" does not exist.

What exactly are black holes?

This section will provide a completely different microscopic picture of black holes from traditional geometric perspective, based on Axiom $\Omega$ and Light Path Conservation Theorem: \textbf{Black holes are not spacetime holes, but "congestion knots" in information processing networks.} Their horizons are holographic screens storing maximally densely packed entangled information.

\subsection{4.4.1 Limit of Information Congestion: Horizon Formation}

Recall the optical metric refractive index we derived in Section 4.2:

\[n(x) \approx 1 + \frac{G}{c^4} \rho_{\text{info}}(x)\]

And time flow rate caused by Light Path Conservation:

\[v_{int} = c/n(x)\]

As matter continuously aggregates, local information processing density $\rho_{\text{info}}$ keeps rising. According to holographic principle (Bekenstein Bound), any region has an information density upper limit $\rho_{max} \sim 1/l_P^2$ (Planck density).

When $\rho_{\text{info}} \to \rho_{max}$, refractive index $n \to \infty$.

What happens then?

\begin{enumerate}
\item \textbf{Time Freeze}: Internal evolution speed $v_{int} \to 0$ (relative to external observers). Qubits inside black holes are frantically computing (proper time extremely fast), but appear "frozen" to the outside.
\end{itemize}

\begin{enumerate}
\item \textbf{Space Stretch}: Spatial scaling factor $a = n \to \infty$. Physical distance to black hole center becomes infinite. This corresponds to "deep well" in classical geometry.

\begin{enumerate}
\item \textbf{Light Speed Zero}: External signal propagation speed $v_{ext} = c/n^2 \to 0$. Information cannot escape this region.
\end{itemize}

We define \textbf{horizon} as the critical surface satisfying $\rho_{\text{info}} = \rho_{max}$.

This explains why horizon is a one-way membrane: it is a \textbf{saturation layer} of information processing capacity. Any additional information trying to cross it gets "stuck" on the surface due to lack of extra bandwidth.

\subsection{4.4.2 Microscopic Statistics of Entropy: Why Area?}

Why is black hole entropy proportional to area $A$, not volume $V$? This is incomprehensible in continuous space, but obvious in QCA networks.

Consider horizon as a closed surface $\Sigma$. On QCA graph $\Lambda$, horizon cuts connection edges (links) between internal nodes $V_{in}$ and external nodes $V_{out}$.

Each cut connection edge represents a pair of entangled qubits (Bell pair).

For external observers, internal states are unknowable (shielded by horizon). Therefore, we perform partial trace over these unknowable degrees of freedom.

According to von Neumann entropy definition, entanglement entropy $S$ equals number of cut connections $N_{links}$ times entanglement per edge (maximum 1 bit).

\[S_{BH} \propto N_{links}\]

On regular lattices, number of connections crossing a surface is obviously proportional to discrete area (number of lattice points) of that surface:

\[N_{links} \sim \frac{\text{Area}}{a^2}\]

where $a$ is lattice spacing (Planck length $l_P$).

Therefore, we directly derive the form of Bekenstein-Hawking formula:

\[S_{BH} \propto \frac{\text{Area}}{l_P^2}\]

\textbf{Physical Interpretation}: \textbf{Black hole surface is the densest hard drive in the universe.} Every Planck area unit $l_P^2$ is a bit storage location. Black hole entropy is enormous because it "flattens" (projects) all information in three-dimensional volume onto two-dimensional surface. This is the microscopic mechanism of \textbf{holographic principle}.

\subsection{4.4.3 Resolution of Singularities and Internal Structure}

Classical theory predicts a volume-zero, density-infinite singularity inside horizon. But in QCA, this cannot happen.

Because $\Lambda$ is discrete, number of nodes is finite. When matter collapses, it cannot collapse into a point (due to Pauli exclusion principle and minimum lattice spacing constraints).

Instead, black hole interior forms a \textbf{high-density entanglement core}.

\begin{itemize}
\item \textbf{Spatial Topology}: Interior is not a point, but a highly interconnected complex network, possibly similar to \textbf{small-world network} or \textbf{fast scrambler}.
\end{enumerate}

\begin{itemize}
\item \textbf{Information State}: Information falling into black holes does not disappear, but is rapidly scrambled, uniformly distributed throughout horizon/internal network. This is like dropping ink into the sea—ink molecules remain, but information is completely delocalized.

Therefore, \textbf{singularities are only breakdown points of continuum mathematical models, not termination of physical reality.} In QCA, singularities are replaced by Planck-scale "fuzzballs" or high-entanglement states.

\subsection{4.4.4 Hawking Radiation and Unitarity Restoration}

Finally, how to resolve information paradox?

According to unitarity of Axiom $\Omega$, black hole evolution operator $\hat{U}_{BH}$ must be unitary. This means information never disappears.

Hawking radiation is not random product of vacuum fluctuations, but \textbf{stimulated radiation of highly entangled states} on horizon surface.

\begin{enumerate}
\item \textbf{Horizon is Not Vacuum}: It is an active qubit layer.
\end{itemize}

\begin{enumerate}
\item \textbf{Information Leakage}: Due to QCA's local interactions, qubits on horizon exchange information with external vacuum qubits (through tiny non-zero $v_{ext}$, i.e., quantum tunneling).

\begin{enumerate}
\item \textbf{Encrypted Transmission}: Radiated photons carry horizon interior information, but this information is highly encrypted (scrambled). This is like burning an encyclopedia—smoke and ash contain all atomic information of the book, but appear as thermal radiation (random).
\end{itemize}

If we could collect all Hawking radiation and have a super quantum computer, we could in principle reconstruct the initial matter state that fell into the black hole. This completely resolves black hole information paradox—\textbf{black holes not only don't swallow information, they are actually the most perfect information mixers and emitters.}

\subsection{4.4.5 Summary}

Black holes are ultimate entities in QCA universe. They are limits of spacetime, warehouses of information, and magnifying glasses of quantum gravitational effects. Through black holes, we glimpse the deepest secrets of Axiom $\Omega$: \textbf{Matter, gravity, and information are trinity on the horizon.}

At this point, we have completed Part II "The Emergence of Spacetime." Starting from a simple lattice axiom, we reconstructed light speed, relativity, curved spacetime, and even black holes. In the next Part III, we turn our gaze to actors on the stage, exploring that more colorful world—\textbf{The Emergence of Matter}.

