\section{4.3 Entropic Derivation of Einstein Field Equations: Proof of IGVP Principle}

In the previous two sections, we established the concept that "geometry is entanglement" and derived the specific form of optical metric through "local information volume conservation." This solves the \textbf{kinematics} problem: if gravity exists, what should it look like (how does metric deform)?

Now, we must solve the \textbf{dynamics} problem: \textbf{Why} does matter distribution determine spacetime curvature? Or, why must the equation of state be Einstein's field equations $G_{\mu\nu} = 8\pi G T_{\mu\nu}$?

In standard general relativity, this equation is introduced as an axiomatic assumption (Einstein-Hilbert action). But in our discrete ontology, gravity is not a fundamental force, but an \textbf{entropic force}. It is similar to gas pressure or rubber elasticity—a macroscopic statistical tendency of systems to maximize microscopic state numbers.

This section will propose and prove the \textbf{Information-Gravity Variational Principle (IGVP)}, thereby deriving Einstein's field equations from first principles.

\subsection{4.3.1 Microscopic Mechanism of Entropic Force}

In thermodynamics, if a system's entropy $S$ depends on some macroscopic parameter $x$, the system experiences a statistical force $F = T \frac{\partial S}{\partial x}$, driving it toward entropy increase.

In QCA universe, the macroscopic parameter is \textbf{spacetime geometry (metric $g_{\mu\nu}$)}.

We need to consider two parts of entropy:

\begin{enumerate}
\item \textbf{Geometric Entropy $S_{geom}$}: This is the entropy of the holographic entanglement network itself, corresponding to complexity of spacetime connections. According to Ryu-Takayanagi formula and black hole entropy formula, it is proportional to area.
\end{itemize}

\begin{enumerate}
\item \textbf{Matter Entropy $S_{matter}$}: This is entanglement entropy or information content carried by matter excitations (Qubit states) distributed on the network.

\textbf{Core Assumption}: The universe is always in \textbf{maximum entanglement equilibrium state} within local causal diamonds. That is, for any given matter distribution, spacetime geometry automatically adjusts to maximize total entropy.

\[S_{tot} = S_{geom} + S_{matter} \to \text{Max}\]

\subsection{4.3.2 Construction of IGVP Action}

To formalize this idea, we construct total entropy functional (equivalent to action $I$).

\textbf{1. Geometric Entropy Term}

In continuous limit, area is proportional to integral of curvature scalar $R$ (this is generalization of Wald entropy to Einstein gravity, or derived from deficit angles in Regge calculus).

\[S_{geom} \propto \frac{1}{l_P^2} \int d^4x \sqrt{-g} R\]

Coefficient $1/l_P^2$ originates from Planck scale discreteness: curvature is actually macroscopic average of lattice defect density.

\textbf{2. Matter Entropy Term}

Matter's load on the network manifests as information processing density. Macroscopically, this is Lagrangian density $\mathcal{L}_{m}$ (or more accurately, it derives stress-energy tensor).

\[S_{matter} \propto \int d^4x \sqrt{-g} \mathcal{L}_{m}\]

\textbf{3. Variational Principle}

We define total action (as negative of entropy or its Legendre transform):

\[I_{IGVP}[g] = \frac{1}{16\pi G} \int d^4x \sqrt{-g} R + \int d^4x \sqrt{-g} \mathcal{L}_{m}\]

This looks like standard Einstein-Hilbert action. But in our theory, each term has clear information-theoretic origin:

\begin{itemize}
\item $\frac{1}{16\pi G}$ is not an arbitrary coupling constant; it directly relates to QCA's maximum information density (bits/Planck area).
\end{enumerate}

\begin{itemize}
\item $R$ is a measure of network connection complexity.

\begin{itemize}
\item $\mathcal{L}_{m}$ is matter entanglement's occupation of network resources.
\end{enumerate}

\subsection{4.3.3 Derivation of Field Equations}

To find equilibrium geometry, we vary metric $g^{\mu\nu}$, seeking stationary points $\delta I_{IGVP} = 0$.

\begin{enumerate}
\item \textbf{Geometric Term Variation}:
\end{itemize}

   Using Palatini identity:

   \[\delta (\sqrt{-g} R) = \sqrt{-g} (R_{\mu\nu} - \frac{1}{2} R g_{\mu\nu}) \delta g^{\mu\nu} + \sqrt{-g} \nabla_\sigma (\dots)\]

   (Boundary terms handled at holographic screen, ignored here).

   Result: $\frac{1}{16\pi G} (G_{\mu\nu})$.

\begin{enumerate}
\item \textbf{Matter Term Variation}:
\end{itemize}

   By definition, stress-energy tensor $T_{\mu\nu}$ is matter action's response to metric:

   \[T_{\mu\nu} \equiv -\frac{2}{\sqrt{-g}} \frac{\delta (\sqrt{-g} \mathcal{L}_{m})}{\delta g^{\mu\nu}}\]

   Or more intuitively, in information theory, $T_{\mu\nu}$ represents \textbf{information cost required to change geometry (stretch spacetime)}.

   Result: $-\frac{1}{2} T_{\mu\nu}$.

\begin{enumerate}
\item \textbf{Equilibrium Equation}:
\end{itemize}

   Setting total variation to zero:

   \[\frac{1}{16\pi G} (R_{\mu\nu} - \frac{1}{2} R g_{\mu\nu}) - \frac{1}{2} T_{\mu\nu} = 0\]

   Rearranging:

   \[R_{\mu\nu} - \frac{1}{2} R g_{\mu\nu} = 8\pi G T_{\mu\nu}\]

\textbf{Q.E.D.}

\subsection{4.3.4 Reconstruction of Physical Meaning: Equation of State of Spacetime}

Through IGVP, we not only derived Einstein's field equations, but more importantly changed our understanding of them.

\begin{itemize}
\item \textbf{Traditional View}: Mass tells spacetime how to curve. This is a dynamical causal relationship.
\end{enumerate}

\begin{itemize}
\item \textbf{IGVP View}: Field equations are \textbf{equations of state}, like $PV = nRT$.

\item $G_{\mu\nu}$ (geometric tensor) corresponds to system's "elastic modulus" or "restoring force" for geometric deformation.
\end{enumerate}

\item $T_{\mu\nu}$ (energy-momentum) corresponds to system's internal "thermal pressure" or "information flow."
\end{itemize}

\item $8\pi G$ corresponds to "Boltzmann constant," connecting microscopic bit numbers with macroscopic geometric quantities.
\end{enumerate}

\textbf{Gravity exists because spacetime network tries to maintain maximum entropy distribution.} When matter aggregates, it reduces local entanglement degrees of freedom (occupies channels). To compensate this entropy decrease, spacetime geometry must curve (increase surface area/connection number), thus restoring thermodynamic equilibrium.

This is why gravity is always attractive (at least under positive energy conditions): because matter always tends to aggregate to minimize occupation of total network information capacity, or, the network tends to contract to maximize connection density, until pushed apart by matter's "exclusion principle."

At this point, we have completed the complete logical closed loop from microscopic QCA to macroscopic general relativity. Gravity is no longer mysterious; it is an inevitable product of quantum information statistical mechanics.

