\section{1.3 Information Realism: Bits and Qubits as the Atoms of Matter}

After revealing the pathology of continuum and the revelation of black hole entropy, we face an unavoidable ontological question: If the universe is not composed of continuous fields or matter, what are its "atoms"?

This section will establish the core viewpoint of this book—\textbf{Information Realism}. We will argue that the most fundamental constituent units of the physical universe are not electrons, quarks, or strings, but \textbf{Bits} and \textbf{Qubits}. Matter, energy, space, and time are all macroscopic phenomena emerging from interactions of these underlying information units.

\subsection{1.3.1 "It from Bit"}

John Wheeler, in his visionary paper "Information, Physics, Quantum: The Search for Links," proposed the famous slogan: "It from Bit." He wrote:

\begin{quote}
"Every 'it'—every particle, every force field, even spacetime itself—derives its function, its meaning, its very existence entirely... from an apparatus-elicited answer to yes-or-no questions, binary choices, bits."
\end{quote}

This view was radical at the time, but today it has become a cornerstone of quantum information physics. If we view the universe as a physical system, then the most fundamental description of this system's state is answering a series of "yes/no" questions (e.g., is spin up or down? Is lattice point empty or full?).

However, classical bits (0 or 1) are insufficient to describe the interference and entanglement we observe in the microscopic world. Therefore, we must upgrade Wheeler's dictum to: \textbf{"It from Qubit."}

\subsection{1.3.2 Qubits: The Minimal Units of Physical Reality}

In our QCA model, the universe is discretized into tiny cells. The most fundamental physical quantity carried by each cell is a \textbf{qubit}.

A qubit state $|\psi\rangle$ is a unit vector in two-dimensional complex Hilbert space $\mathbb{C}^2$:

\[
|\psi\rangle = \alpha |0\rangle + \beta |1\rangle
\]

where $|\alpha|^2 + |\beta|^2 = 1$.

Why do we regard Qubits as more fundamental "atoms" than electrons?

\begin{enumerate}
\item \textbf{Universality}: Any finite-dimensional quantum system (regardless of whether its physical carrier is photons, ions, or superconducting circuits) can be decomposed into tensor products of Qubits. Qubits are the universal currency of quantum information.
\end{itemize}

\begin{enumerate}
\item \textbf{Non-locality and Entanglement}: Two Qubits can form entangled states (such as Bell states), and this non-local correlation cannot be simulated by classical particles. As we will see in subsequent chapters, it is precisely this entanglement that "stitches" discrete cells together, emerging as continuous spatial geometry.

\begin{enumerate}
\item \textbf{Holography}: A system containing $N$ Qubits has a maximum information capacity strictly limited by $N$. This perfectly matches the "finite capacity" we saw in black hole entropy.
\end{itemize}

Therefore, we define the bottom layer of physical reality no longer as $x, y, z, t$, but as state vectors $|\Psi\rangle$ in Hilbert space. Spatial position $x$ is merely the \textbf{index} of qubits in the lattice network, while time $t$ is merely the \textbf{counter} of logic gate operations.

\subsection{1.3.3 From Information to Matter: Reconstruction of Elementary Particles}

If Qubits are bricks, how are the familiar electrons and photons "built"?

In traditional physics, particles are viewed as point-like entities moving in spacetime. But in the QCA framework, \textbf{particles are excitation patterns in information networks}.

Imagine a two-dimensional grid filled with Qubits.

\begin{itemize}
\item \textbf{Vacuum} corresponds to some low-entanglement ground state (e.g., all spins down $|00...0\rangle$).
\end{enumerate}

\begin{itemize}
\item \textbf{Photons} correspond to a "flip" wave propagating on the grid (e.g., $|010...0\rangle \to |001...0\rangle$). Since there is no mechanism preventing this flip's transmission, it propagates at maximum speed (speed of light).

\begin{itemize}
\item \textbf{Electrons} correspond to a complex \textbf{topological knot}. Like a knot on a rope, it is a local information structure that not only contains flips but also phase winding. This winding prevents it from dissipating at light speed, forcing it to maintain its existence in place—this is the origin of \textbf{mass}.
\end{enumerate}

In this picture, matter is not a "thing" but a "process." An electron is not a small ball named "electron"; it is a self-maintaining, self-referential information vortex in the Qubit ocean.

\subsection{1.3.4 Conclusion: Computational Ontology}

At this point, we have completed a thorough reconstruction of physics ontology:

\begin{enumerate}
\item \textbf{Dematerialization}: There is no "hard" matter, only soft information.
\end{itemize}

\begin{enumerate}
\item \textbf{De-backgrounding}: There is no a priori spacetime stage, only interrelations (entanglement) between qubits.

\begin{enumerate}
\item \textbf{Discretization}: There are no infinitesimals, only finite logical steps.
\end{itemize}

We no longer ask "what is the universe made of," but "how does the universe compute."

After establishing this ontological foundation, we can finally take the most crucial step—writing down the single law governing the operation of all these Qubits. This is the theme of the next chapter: \textbf{The Ultimate Axiom $\Omega$}.

