\section{Appendix D: Glossary of Key Terms}

\textbf{Appendix D: Glossary of Key Terms}

This book constructs a completely new discourse system for physics. Many traditional physics terms (such as mass, gravity, time) are given new definitions based on information theory and discrete geometry in this book, while also introducing some proprietary new concepts.

To facilitate reader reference and eliminate ambiguity, this appendix collects core terminology from the entire book and provides their strict definitions within the framework of "unitary QCA ontology."

---

\subsection{A}

\begin{itemize}
\item \textbf{Algorithmic Turmoil}
\end{enumerate}

\item \textbf{Definition}: Refers to the universe's evolution being in a self-organized critical state that can never relax to thermal equilibrium (heat death).

\item \textbf{Mechanism}: Driven by the "Red Queen game" between agents (observers). To survive in competition, systems continuously increase computational complexity, leading to an eternal cycle of collapse of old structures and emergence of new structures.
\end{itemize}

\item \textbf{Source}: Chapter 8, Section 8.3.
\end{enumerate}

\begin{itemize}
\item \textbf{Agent}

\item \textbf{Definition}: A subsystem in QCA networks possessing a \textbf{Markov blanket} (boundary) and \textbf{internal model} (self-referential structure).
\end{enumerate}

\item \textbf{Characteristics}: Agents actively consume free energy to resist environmental thermalization, exhibiting seemingly "purposeful" behavior (minimizing prediction error).
\end{itemize}

\item \textbf{Source}: Chapter 8, Section 8.1.
\end{enumerate}

---

\subsection{C}

\begin{itemize}
\item \textbf{Causal Locality}
\end{enumerate}

\item \textbf{Definition}: Structural property of QCA evolution operator $\hat{U}$, requiring that the next-moment state of any node depends only on the states of nodes within its finite neighborhood.

\item \textbf{Corollary}: Directly leads to light cone structure and existence of maximum signal velocity $c$, prohibiting action at a distance.
\end{itemize}

\item \textbf{Source}: Chapter 2, Section 2.3; Chapter 3, Section 3.1.
\end{enumerate}

\begin{itemize}
\item \textbf{Connection Field / Link Variable}

\item \textbf{Definition}: Unitary operator $U_{yx}$ defined on lattice connection edges (links), used to "translate" differences in local Hilbert space bases between adjacent nodes $x$ and $y$.
\end{enumerate}

\item \textbf{Physical correspondence}: Corresponds to gauge potential $A_\mu$ in gauge field theory. The holonomy on closed loops corresponds to field strength (such as electromagnetic fields, gluon fields).
\end{itemize}

\item \textbf{Source}: Chapter 6.
\end{enumerate}

---

\subsection{I}

\begin{itemize}
\item \textbf{Information-Gravity Variational Principle (IGVP)}
\end{enumerate}

\item \textbf{Definition}: A thermodynamic variational principle asserting that equilibrium states of spacetime geometry correspond to maximum holographic entanglement entropy.

\item \textbf{Formula}: $\delta (S_{geom} + S_{matter}) = 0 \implies G_{\mu\nu} = 8\pi G T_{\mu\nu}$.
\end{itemize}

\item \textbf{Significance}: Interprets Einstein's field equations as the equation of state of the information manifold, rather than fundamental dynamical laws.
\end{enumerate}

\item \textbf{Source}: Chapter 4, Section 4.3.
\end{itemize}

\begin{itemize}
\item \textbf{Information Mass ($M_I$)}
\end{enumerate}

\item \textbf{Definition}: Physical quantity measuring the "weight" of internal structure of complex systems (such as observers).

\item \textbf{Formula}: $M_I \propto \Phi \cdot \mathcal{D}$, where $\Phi$ is integrated information and $\mathcal{D}$ is logical depth.
\end{itemize}

\item \textbf{Effect}: Systems with high $M_I$ have enormous "information inertia," tending to be stationary in external space (motionless), and produce significant gravitational effects.
\end{enumerate}

\item \textbf{Source}: Chapter 8, Section 8.2.
\end{itemize}

\begin{itemize}
\item \textbf{Light Path Conservation (Conservation of Information Celerity)}
\end{enumerate}

\item \textbf{Definition}: Core theorem of the entire book. States that the total information update amount of physical entities within Planck time is constant.

\item \textbf{Formula}: $v_{ext}^2 + v_{int}^2 = c^2$.
\end{itemize}

\item \textbf{Significance}: Unifies special relativity ($v_{ext}$) with quantum mechanics ($v_{int}$), explaining the complementary relationship between time dilation and mass.
\end{enumerate}

\item \textbf{Source}: Chapter 3, Section 3.2.
\end{itemize}

---

\subsection{L}

\begin{itemize}
\item \textbf{Local Information Volume Conservation}
\end{enumerate}

\item \textbf{Definition}: Geometric constraint of unitary evolution in the continuous limit. Requires that deformation of macroscopic metrics cannot change the effective number of microscopic degrees of freedom contained per unit coordinate volume.

\item \textbf{Formula}: $\eta_t \cdot \eta_x^3 = 1$ (in 3+1 dimensions).
\end{itemize}

\item \textbf{Application}: Corrects scalar gravity theory, derives correct optical metric, solves the coefficient problem of light deflection angle.
\end{enumerate}

\item \textbf{Source}: Chapter 4, Section 4.2.
\end{itemize}

---

\subsection{O}

\begin{itemize}
\item \textbf{Optical Metric}
\end{enumerate}

\item \textbf{Definition}: Effective metric describing propagation of light (and matter) in gravitational fields, generated by refractive index $n(x)$.

\item \textbf{Form}: $ds^2 = -n^{-2} c^2 dt^2 + n^2 dl^2$.
\end{itemize}

\item \textbf{Essence}: Reflects that gravity is not curvature of spacetime, but non-uniform distribution of information processing density in the medium (QCA network).
\end{enumerate}

\item \textbf{Source}: Chapter 4, Section 4.2.
\end{itemize}

---

\subsection{T}

\begin{itemize}
\item \textbf{Topological Impedance}
\end{enumerate}

\item \textbf{Definition}: Microscopic mechanism of inertial mass. Refers to the lagged response exhibited by an information flow structure with non-trivial topology (winding number $\neq 0$) when changing its motion state.

\item \textbf{Mechanism}: When $v_{ext} \to c$, internal refresh rate $v_{int} \to 0$, causing the system unable to timely process external perturbations, manifesting as inertial divergence.
\end{itemize}

\item \textbf{Source}: Chapter 5, Section 5.3.
\end{enumerate}

\begin{itemize}
\item \textbf{Unified Time Identity}

\item \textbf{Definition}: Equivalence relation between microscopic time flow rate and local density of states.
\end{enumerate}

\item \textbf{Formula}: $\kappa(E) = 2\pi \rho(E)$.
\end{itemize}

\item \textbf{Significance}: Explains gravitational redshift—density of states is higher deep in potential wells, requiring longer time to traverse states, hence time slows down.
\end{enumerate}

\item \textbf{Source}: Related papers and Chapter 4 background.
\end{itemize}

---

\subsection{U}

\begin{itemize}
\item \textbf{Unitarity}
\end{enumerate}

\item \textbf{Definition}: Evolution operator $\hat{U}$ satisfies $\hat{U}^\dagger \hat{U} = \mathbb{I}$.

\item \textbf{Physical meaning}: Conservation of information. Past, present, and future contain strictly equal amounts of information; no information is completely erased or created from nothing. It is the mathematical root of Light Path Conservation and Born's rule.
\end{itemize}

\item \textbf{Source}: Chapter 2, Section 2.2.
\end{enumerate}

---

\subsection{Z}

\begin{itemize}
\item \textbf{Zitterbewegung (Trembling)}
\end{enumerate}

\item \textbf{Definition}: Rapid oscillatory flipping between positive and negative chirality occurring at microscopic scales for massive particles.

\item \textbf{New interpretation}: Not a mathematical artifact, but an "internal cycle" particles are forced to perform when unable to move at full speed, to maintain light path conservation. Its frequency is the measure of rest mass.
\end{itemize}

\item \textbf{Source}: Chapter 5, Section 5.1.
\end{enumerate}

---

\textbf{(End of main text and appendices)}

