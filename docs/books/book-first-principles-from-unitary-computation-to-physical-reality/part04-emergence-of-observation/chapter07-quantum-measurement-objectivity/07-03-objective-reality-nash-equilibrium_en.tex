\section{7.3 The Achievement of Objective Reality: Nash Equilibrium and Consensus Geometry in Multi-Agent Systems}

From the combinatorial proof of Born's rule in the previous section, we got a startling conclusion: So-called "randomness" is merely subjective feeling of local observers due to information truncation. This seems to push physics toward solipsism—if every observer is in their own "information bubble," each bubble has its own historical branch, then what is the shared, unbreakable "objective reality" we all perceive?

This section will unify quantum mechanics with game theory by introducing \textbf{multi-agent game} models, proving: \textbf{Objective reality is not a priori physical background, but Nash equilibrium emerging from a vast, decentralized communication network.}

Spacetime geometry is essentially a "consensus protocol" reached by all observers.

\subsection{7.3.1 Wigner's Friend and Consensus Puzzle}

To understand fragility of "objectivity," let's revisit "Wigner's Friend" thought experiment.

\begin{enumerate}
\item Friend $F$ measures a qubit in a sealed room, gets result $|0\rangle$ or $|1\rangle$. For $F$, wave function collapsed, world is definite.
\end{itemize}

\begin{enumerate}
\item Wigner $W$ stands outside the room. For $W$, entire room (including friend) is in superposition $|\Psi\rangle = \alpha |0\rangle |F_0\rangle + \beta |1\rangle |F_1\rangle$.

\begin{enumerate}
\item At this point, for $F$, fact is "I saw 0"; for $W$, fact is "friend is in superposition."
\end{itemize}

If $F$ walks out and tells $W$: "I saw 0," then $W$'s wave function also "collapses."

This reveals definition of objectivity: \textbf{Objectivity = mutual consistency of information between different observers.}

In QCA framework, we no longer discuss abstract "wave functions," but \textbf{consistency checks}.

If $W$ and $F$ want to reach consensus, they must physically \textbf{communicate} (exchange photons or matter).

Communication itself is a physical process (interaction), causing $W$ and $F$'s states to entangle.

\textbf{Theorem 7.3.1 (Consensus Entanglement Theorem)}:

Under unitary evolution, two observers $A$ and $B$ can only reach consensus on measurement result of observable $\hat{O}$ when their quantum states are completely entangled about $\hat{O}$ (i.e., in same branch of Schmidt decomposition).

\[|\Psi_{AB}\rangle \approx \sum_k c_k |A_k\rangle |B_k\rangle\]

If state is $|A_k\rangle |B_j\rangle$ ($k \neq j$), it's called "illusion" or "disagreement." Unitary dynamics (interaction Hamiltonian) tends to eliminate disagreement terms (through decoherence), driving system into diagonalized consensus state.

\subsection{7.3.2 Nash Equilibrium: Why Do We See the Same Moon?}

Einstein once asked: "Does the moon exist only when I look at it?"

From QCA perspective, this question becomes: "Why does everyone see the moon at the same position?"

We model the universe as a game network composed of $N$ agents. Each agent $i$ maintains an internal model $M_i$ (beliefs about external world).

Agent's goal is to minimize \textbf{prediction error} (i.e., free energy $F$).

\[F_i = \text{KL}(q_i(\text{state}) || p(\text{sensory data}))\]

If each agent independently constructs models, their worldviews might be completely different (like schizophrenic patients).

But agents have \textbf{communication} between them (light, sound, gravity).

Communication causes \textbf{coupling}. If agent $A$ thinks moon is on left, and agent $B$ thinks moon is on right, when they exchange information, both produce huge prediction errors (surprise).

To minimize this error, they must adjust their respective internal models to be \textbf{mutually compatible}.

\textbf{Definition 7.3.2 (Objective Reality as Nash Equilibrium)}:

Let total free energy of entire system be $F_{total} = \sum_i F_i(M_i, \{M_{j \neq i}\})$.

System's evolution dynamics $\dot{M}_i \propto -\nabla_{M_i} F_{total}$ drives system to find minimum.

When system reaches steady state $\frac{dM_i}{dt} = 0$, all agents' models $\{M^*_i\}$ reach a \textbf{Nash equilibrium}.

\textbf{Common part} contained in this equilibrium state $\{M^*\}$ is what we call "objective reality."

\textbf{Conclusion}: Moon is there because all observers (including photons, air molecules) have "negotiated" its position through continuous interactions. Any observer deviating from this consensus will be severely "punished" by environmental information (decoherence), until they correct their model or are assimilated by environment.

\subsection{7.3.3 Consensus Geometry: Spacetime as the Largest Public Ledger}

Pushing this view to the extreme: \textbf{Spacetime itself is the largest consensus structure.}

In Chapter 4, we defined geometry as entanglement. Now we can say: Geometry is \textbf{causal relationship protocol recognized by all observers}.

\begin{itemize}
\item \textbf{Distance}: Not absolute ruler length, but consensus on "if I send you a signal, how long do you need to receive it."
\end{enumerate}

\begin{itemize}
\item \textbf{Position}: Not absolute coordinates, but "my network relationship relative to all other objects."

In QCA networks, there is no absolute $x, y, z$. But due to rigidity of local interactions, all observers eventually agree on network's \textbf{topological structure} by exchanging photons (measuring light paths).

This network-wide consistent topological skeleton is the \textbf{continuous spacetime manifold} we perceive.

In this sense, physical laws (such as relativity) are not merely natural laws; they are \textbf{underlying algorithms for distributed networks to maintain data consistency}.

\begin{itemize}
\item Special Relativity (Lorentz transformations) is \textbf{data conversion protocol between different reference frames}.
\end{enumerate}

\begin{itemize}
\item General Relativity (covariance) is \textbf{consensus protocol ensuring physical truth is independent of node coordinates}.

\subsection{7.3.4 Layers of Reality}

We arrive at a layered reality picture:

\begin{enumerate}
\item \textbf{Ontic Level}: Underlying QCA network, unitary, deterministic, but unknowable to single observers (hidden).
\end{itemize}

\begin{enumerate}
\item \textbf{Consensus Level}: Macroscopic classical world (spacetime, matter) emerging through multi-agent games. This is "objective reality."

\begin{enumerate}
\item \textbf{Subjective Level}: Quantum branches private to single observers (illusions, dreams, uncollapsed wave functions).
\end{itemize}

Science's task is to extract that hard consensus layer from this subjective fog through experiments (forcing games with environment).

\subsection{7.3.5 Summary}

Objectivity is not a gift from heaven, but an \textbf{evolutionary achievement}.

The universe weaves classical consistency from quantum chaos through countless microscopic "handshakes" and "entanglements." We live in the same world because we continuously "measure" each other, and in this process become part of each other's reality.

This concludes Part IV's discussion on observation. We have explained probability, measurement, and objectivity.

In the next Part V, also the final part of the book, we will put these theories to the cruelest battlefield—\textbf{experimental verification}. If all this is not just philosophy, it must be observable.

