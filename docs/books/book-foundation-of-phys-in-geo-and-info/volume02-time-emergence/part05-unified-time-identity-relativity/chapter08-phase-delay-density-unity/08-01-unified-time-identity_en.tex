\section{Derivation of Unified Time Identity: $\kappa(E) = \varphi'/\pi = \rho_{\text{rel}} = \text{Tr}\mathsf{Q}/2\pi$}

Physics has long been accustomed to treating time $t$ as a background parameter external to matter, and mass or energy as entities filling spacetime. However, the discrete ontology of QCA suggests a deeper duality between these two. This section will rigorously derive and establish the \textbf{Unified Time Identity}, proving that the measurement of "time" and the density of "matter" are different manifestations of the same physical quantity at the spectral geometric level.

\subsection{Review of Three Physical Quantities}

First, let us clearly define the three core physical quantities participating in the unification, which respectively represent three different aspects of physical reality:

\begin{enumerate}
\item \textbf{Geometric Side: Scattering Half-Phase Derivative $\varphi'(E)$}

In scattering theory, the determinant of the scattering matrix $S(E)$ is a unimodular complex number. Define the total scattering phase $\Phi(E)$ as:
$$\det S(E) = e^{i\Phi(E)}$$

Define the \textbf{half-phase} $\varphi(E) = \frac{1}{2}\Phi(E)$. Its derivative with respect to energy $\varphi'(E) = \frac{d\varphi}{dE}$ describes the geometric rotation speed of the wave function in energy space.

\item \textbf{Thermodynamic Side: Relative Density of States $\rho_{\text{rel}}(E)$}

Density of states (DOS) is the core of statistical mechanics. After introducing interaction $V$, the change in local density of states is defined as:
$$\rho_{\text{rel}}(E) \equiv \rho(E) - \rho_0(E) = \frac{d}{dE} \text{Tr} \left[ \Theta(E-H_0) - \Theta(E-H) \right]$$

It quantifies the number of new microscopic degrees of freedom (or information capacity) near the energy shell $E$.

\item \textbf{Dynamical Side: Wigner-Smith Time Delay Trace $\text{Tr}\mathsf{Q}(E)$}

The EWS operator $\mathsf{Q} = -i\hbar S^\dagger \frac{dS}{dE}$ describes the dwell time of wave packets in the scattering region. Its trace $\tau_{tot} = \text{Tr}\mathsf{Q}$ represents the total time delay of all scattering channels, i.e., the "time impedance" of the system as a whole to external probes.
\end{enumerate}

\subsection{Rigorous Derivation of the Identity}

We will use the Birman-Kreĭn theory established in Chapter 7 to complete this proof.

\begin{theorem}[Unified Time Identity]
\label{thm:unified-time}
For scattering systems satisfying the relative trace-class condition, within the absolutely continuous spectrum region, the following three physical quantities are strictly equal (in natural units $\hbar=1$):
$$\kappa(E) \equiv \frac{1}{\pi} \frac{d\varphi(E)}{dE} = \rho_{\text{rel}}(E) = \frac{1}{2\pi} \text{Tr}\mathsf{Q}(E)$$
We define $\kappa(E)$ as the \textbf{Unified Time Scale Density}.
\end{theorem}

\textbf{Proof}:

\textbf{Step 1: Connecting $\text{Tr}\mathsf{Q}$ with $\varphi'$}

According to the definition of the Wigner-Smith operator and the properties of its trace (Section 6.4):
$$\text{Tr}\mathsf{Q}(E) = \text{Tr}\left[ -i S^\dagger(E) \frac{dS(E)}{dE} \right]$$

Using Jacobi's formula $\text{Tr}(A^{-1} dA) = d(\ln \det A)$ and $S^\dagger = S^{-1}$:
$$\text{Tr}\mathsf{Q}(E) = -i \frac{d}{dE} \ln (\det S(E))$$

Substituting the total phase definition $\det S = e^{i\Phi(E)} = e^{2i\varphi(E)}$:
$$\text{Tr}\mathsf{Q}(E) = -i \frac{d}{dE} (2i\varphi(E)) = 2 \frac{d\varphi}{dE} = 2\varphi'(E)$$

Therefore:
$$\frac{1}{2\pi} \text{Tr}\mathsf{Q}(E) = \frac{1}{\pi} \varphi'(E)$$

\textbf{Step 2: Connecting $\varphi'$ with $\rho_{\text{rel}}$}

Using the Birman-Kreĭn trace formula (Section 7.2):
$$\det S(E) = e^{-2\pi i \xi(E)}$$

This means the relationship between total phase $\Phi(E)$ and Kreĭn spectral shift function $\xi(E)$ is (taking continuous branch):
$$\Phi(E) = -2\pi \xi(E) \implies 2\varphi(E) = -2\pi \xi(E) \implies \varphi(E) = -\pi \xi(E)$$

Differentiating with respect to energy:
$$\varphi'(E) = -\pi \xi'(E)$$

Recalling the relationship between spectral shift function and density of states (Section 7.1):
$$\xi(E) = -(N(E) - N_0(E)) \implies \xi'(E) = -(\rho(E) - \rho_0(E)) = -\rho_{\text{rel}}(E)$$

Substituting this into the above:
$$\varphi'(E) = -\pi (-\rho_{\text{rel}}(E)) = \pi \rho_{\text{rel}}(E)$$

That is:
$$\frac{1}{\pi} \varphi'(E) = \rho_{\text{rel}}(E)$$

\textbf{Step 3: Synthesis}

Combining the results of Step 1 and Step 2, we obtain the trinity identity.

$\square$

\subsection{Physical Interpretation: Ontological Status of $\kappa(E)$}

The unified time identity reveals that $\kappa(E)$ is a more fundamental physical quantity than time, energy, or entropy. It constitutes the \textbf{Master Scale} of the QCA universe.

\begin{enumerate}
\item \textbf{Time is Statistics} ($\text{Time} \sim \text{DOS}$):

The formula $\frac{1}{2\pi} \tau_{tot} = \rho_{\text{rel}}$ tells us that \textbf{the essence of time flow is the system's traversal of microscopic states}. If there are no quantum states in a region ($\rho=0$, such as a band gap), wave packets will instantly pass through (tunneling), experiencing no time delay (or extremely short group delay). Conversely, if the density of states is extremely high (such as inside a black hole), wave packets will be "stuck" by countless states, causing extreme time dilation.

\item \textbf{Geometry is Phase} ($\text{Geometry} \sim \text{Phase}$):

The formula $\kappa = \varphi'/\pi$ shows that this statistical property is completely encoded in the geometric phase on the boundary. We don't need to go deep into the system to count states; we only need to measure the "winding rate" of phase on the boundary.

\item \textbf{Planck Information Density}:

The dimension of $\kappa(E)$ is $[\text{Energy}]^{-1} = [\text{Time}]$ (with $\hbar=1$). It actually measures \textbf{information capacity per unit energy interval}. At the Planck scale, this corresponds to the number of bits per Planck energy.
\end{enumerate}

\subsection{Case Verification: One-Dimensional Potential Box}

To concretize this abstract identity, we examine a one-dimensional potential box (infinite square well) of length $L$.

\textbf{Thermodynamic Side (Density of States)}:

Eigenenergy $E_n = \frac{n^2 \pi^2}{2m L^2}$. Momentum $k_n = \frac{n\pi}{L}$.

Density of states $\rho(k) = \frac{dn}{dk} = \frac{L}{\pi}$.

Energy density of states $\rho(E) = \rho(k) \frac{dk}{dE} = \frac{L}{\pi} \frac{1}{v} = \frac{L}{\pi v}$, where $v$ is the group velocity.

\textbf{Dynamical Side (Time Delay)}:

The time for a classical particle to cross the box back and forth is $T = \frac{2L}{v}$.

In the scattering picture, the wave function reflects at boundaries, experiencing phase shifts. Since this is a bound system, we can consider a small scatterer $L$ in a large box $L_{big}$.

Or directly cite the dwell time concept: $\tau_D = \frac{1}{J} \int |\psi|^2 dx = \frac{1}{v/L_{norm}} \cdot 1 = \frac{L_{norm}}{v}$.

For scattering state $S(k) = e^{-2ikL}$ (transmission coefficient phase, corresponding to free propagation), phase shift $\delta(k) = -kL$.

$\varphi' = \frac{d\delta}{dE} = \frac{d(-kL)}{dE} = -L \frac{dk}{dE} = -\frac{L}{v}$.

Here a sign difference appears because free propagation is usually subtracted as background. If we consider the \textbf{extra} delay caused by potential $V$ (e.g., resonance), $\Delta \rho$ and $\tau_{delay}$ will both be positive.

Consider a resonance scattering (Breit-Wigner resonance):
$$S(E) = \frac{E - E_0 - i\Gamma/2}{E - E_0 + i\Gamma/2}$$

Phase derivative (delay): $\tau(E) = \frac{\Gamma}{(E-E_0)^2 + \Gamma^2/4}$.

Density of states (Lorentzian spectrum): $\Delta \rho(E) = \frac{1}{2\pi} \frac{\Gamma}{(E-E_0)^2 + \Gamma^2/4}$.

Clearly satisfies $\tau(E) = 2\pi \Delta \rho(E)$.

\subsection{Conclusion}

The unified time identity $\kappa(E)$ is the key link connecting Volume I (discrete ontology) with Volume III (entropic origin of gravity).

\begin{itemize}
\item Upward, it explains why gravitational fields (metric $g_{\mu\nu}$) affect time—because gravitational fields change the vacuum density of states $\rho(E)$.

\item Downward, it explains the microscopic origin of time—time is not a flowing river but the process of counting quantum states in Hilbert space.
\end{itemize}

In the next section, we will explore the deep corollary \textbf{"Time is Matter"} based on $\kappa(E)$ and explain why in general relativity, the energy-momentum tensor (matter) directly curves spacetime geometry (time).

