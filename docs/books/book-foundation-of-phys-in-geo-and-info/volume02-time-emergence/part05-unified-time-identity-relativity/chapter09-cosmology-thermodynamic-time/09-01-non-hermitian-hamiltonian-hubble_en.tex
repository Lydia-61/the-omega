\section{Non-Hermiticity of Hamiltonian: Information Dissipation Caused by Cosmic Horizon and Hubble Expansion}

In standard cosmology ($\Lambda$CDM model), Hubble expansion is described as the dynamical evolution of metric scale factor $a(t)$. But from the perspective of information physics, expansion means the boundary between the system (observable universe) and environment (beyond horizon) is constantly changing. This section will establish an \textbf{open quantum system} model of cosmic expansion and derive the equivalence between Hubble parameter and information dissipation rate.

\subsection{Horizon as Information Sink}

Consider an observer $\mathcal{O}$ located in de Sitter space or de Sitter-like expanding universe. Due to finite speed of light and spatial expansion, there exists a \textbf{cosmological horizon} with radius $R_H = c/H$.

For $\mathcal{O}$, the horizon is a \textbf{one-way membrane}:

\begin{enumerate}
\item \textbf{Information Outflow}: Particles or radiation inside the horizon can gradually approach the horizon through redshift, eventually their wavelength exceeds the horizon radius, effectively "leaving" the observable Hilbert space.

\item \textbf{Irreversibility}: Once information crosses the horizon, it is permanently lost for local observers (transformed into horizon entropy in the holographic sense).
\end{enumerate}

Therefore, the observable universe is not a closed quantum system but a \textbf{dissipative system}. The effective Hamiltonian $H_{\text{eff}}$ describing its dynamics must contain an anti-Hermitian part to reflect the loss of probability (or information).

\begin{definition}[Cosmic Effective Hamiltonian]
\label{def:cosmic-hamiltonian}
Let $\mathcal{H}_{\text{obs}}$ be the Hilbert space of the observable universe. The system's evolution is generated by a non-Hermitian operator:
$$H_{\text{eff}} = H_0 - i \Gamma$$
where:

\begin{itemize}
\item $H_0 = H_0^\dagger$ is the Hermitian part, describing local unitary evolution (such as particle interactions).

\item $\Gamma = \Gamma^\dagger \ge 0$ is the dissipator operator, describing the rate at which information crosses the horizon.
\end{itemize}
\end{definition}

\subsection{Imaginary Energy Levels and Hubble Parameter}

Eigenvalues of non-Hermitian Hamiltonians are complex: $E = E_r - i \gamma$.

The corresponding state vector $|\psi(t)\rangle \sim e^{-i E t} = e^{-i E_r t} e^{-\gamma t}$.

Modulus squared $|\psi(t)|^2 \sim e^{-2\gamma t}$ decays with time.

In cosmological context, this "decay" does not mean matter truly disappears, but rather that matter is \textbf{diluted} relative to the expanding background.

For a comoving volume $V$, its physical volume grows with time $V(t) \propto a(t)^3$. Particle number density $n(t) \propto a(t)^{-3}$.

If we define wave function normalization within comoving volume, then probability density $\rho \sim |\psi|^2$ in physical volume satisfies:
$$\frac{d\rho}{dt} = -3H \rho$$

Comparing with probability decay rate $2\gamma$ of non-Hermitian evolution, we obtain a key correspondence.

\begin{theorem}[Hubble-Dissipation Equivalence Principle]
\label{thm:hubble-dissipation}
In the open quantum system description with comoving observers as reference, the cosmic Hubble expansion rate $H(t)$ is equivalent to the imaginary part (dissipation rate) of the effective Hamiltonian:
$$\Gamma \cong \frac{3}{2} \hbar H(t) \mathbb{I}$$
(The coefficient depends on dimension and specific state definition, for $d$-dimensional space it is $d/2$).

This means that \textbf{cosmic expansion is essentially an exponential decay process of observable information} (relative to maximum potential capacity).
\end{theorem}

\textbf{Proof Outline}:

Consider the total Hilbert space dimension $D(t)$ of the QCA universe. In an expanding universe, new degrees of freedom (lattice sites) continuously flow in from the horizon (or the horizon expands to sweep over more lattice sites).

For fixed observers, their accessible number of degrees of freedom $N_{obs}$ is limited by the horizon.

According to the holographic principle, loss/exchange of degrees of freedom on the horizon causes mixing of internal states.

Using Lindblad master equation to describe evolution of density matrix $\rho$:
$$\frac{d\rho}{dt} = -i[H_0, \rho] - \{ \Gamma, \rho \} + \mathcal{J}(\rho)$$

where $\{ \Gamma, \rho \}$ describes decay of probability amplitude (dilution), corresponding to redshift and density decrease caused by expansion; $\mathcal{J}(\rho)$ describes backflow of horizon radiation (Hawking radiation/Gibbons-Hawking radiation).

In the macroscopic limit, decay term dominates, $\Gamma \sim H$. $\square$

\subsection{Redshift as Non-Hermitian Phase Evolution}

Using $H_{\text{eff}} = E - i\Gamma$, we can re-derive cosmological redshift.

In Section 8.3, gravitational redshift originates from rescaling of the real part of $H$. Cosmological redshift originates from the imaginary part.

A photon emitted at time $t_e$ with frequency $\omega_e$. Under non-Hermitian evolution, its phase factor is:
$$U(t, t_e) = \exp\left( -i \int_{t_e}^t (H_0 - i\Gamma) dt' \right)$$

The "imaginary phase" $e^{-\int \Gamma dt'}$ here manifests as decay of wave amplitude. However, in conformal time or appropriate coordinate transformations, decay of wave amplitude and decrease of frequency are conjugate (adiabatic invariant $N_\gamma \sim E/\omega$ conserved).

More intuitively: \textbf{dissipation causes "dragging" of phase}. Since information continuously flows beyond the horizon, the remaining wave function is "stretched."

Wavelength $\lambda \propto a(t)$, frequency $\omega \propto 1/a(t)$.

This is completely consistent with dissipation rate $\Gamma \sim \dot{a}/a$.

\subsection{Physical Meaning: Cosmological Origin of Time Arrow}

This model provides a cosmological origin for the thermodynamic time arrow:

\begin{enumerate}
\item \textbf{Non-Hermiticity is Irreversibility}: Since $H_{\text{eff}}$ is not Hermitian, time-reversal symmetry $t \to -t$ is broken ($i\Gamma \to i\Gamma$ sign unchanged, while $-iH_0$ changes sign, causing evolution equation to change).

\item \textbf{Expansion is Entropy Increase}: Dissipation of information (flowing to horizon) increases horizon entropy. For internal observers, this means pure states evolve into mixed states, local entropy increases.

\item \textbf{Conclusion}: The universe has a time direction (expansion) because it is an open system constantly "forgetting" initial state information.
\end{enumerate}

\begin{corollary}[Preview of Geometric Nature of Dark Energy]
\label{cor:dark-energy-preview}
If $\Gamma$ is constant (corresponding to exponentially expanding de Sitter space), then the universe is in a steady state with constant dissipation rate. This means there exists a non-zero, positive vacuum zero-point energy density. In Section 9.4, we will prove this is precisely the origin of the \textbf{cosmological constant (dark energy)}—it is the minimum energy cost required to maintain balance of horizon information flow.
\end{corollary}

By introducing non-Hermitian Hamiltonians, we incorporate cosmological expansion into the framework of quantum information dynamics. Expansion is no longer arbitrary stretching of metric but an information filtering mechanism caused by horizons. In the next section, we will further combine scattering theory to interpret cosmological redshift as \textbf{phase drift} in adiabatic scattering processes.

