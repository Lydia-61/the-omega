\section{Physical Sampling Theorem: Application of Poisson Summation Formula to Band-Limited Physical Fields}

Shannon-Nyquist Sampling Theorem in information theory tells us that a band-limited signal can be perfectly reconstructed from its discrete samples. This section elevates this mathematical theorem to a physics principle, arguing that in a universe with Planck cutoff, discrete QCA states not only approximate continuous fields, but are \textbf{strictly equivalent} to continuous fields in the \textbf{Band-limited} sense.

\subsection{Band-Limited Nature of Physical Reality}

In standard quantum field theory, momentum spectrum of field operators $\hat{\phi}(x)$ is usually assumed to be unbounded ($k \in (-\infty, \infty)$). But this directly leads to UV divergence and violates the \textbf{Finite Information Axiom} we established in Chapter 1.

In QCA ontology, space is discrete lattice $\Lambda = a\mathbb{Z}^d$ (lattice spacing $a$). This means physical momentum space is no longer $\mathbb{R}^d$, but compact \textbf{Brillouin Zone} $\mathcal{B} = [-\pi/a, \pi/a]^d$.

\textbf{Definition 5.1.1 (Physically Band-limited Field)}

A field function $f(x)$ defined on continuous space $\mathbb{R}^d$ is called ``physically band-limited'' if support set (Support) of its Fourier transform $\tilde{f}(k)$ is strictly contained within Brillouin zone $\mathcal{B}$:

\[
\text{supp}(\tilde{f}) \subseteq \left[ -\frac{\pi}{a}, \frac{\pi}{a} \right]^d
\]

Any high-energy modes beyond this range have no physical meaning in discrete ontology, or they are merely ``Aliases'' of low-energy modes.

\textbf{Axiom Corollary}: Due to existence of minimum physical length $a$ (Planck length), all observable physical fields are essentially band-limited. Therefore, continuous space description does not contain more information than discrete lattice description.

\subsection{Poisson Summation Formula: Bridge between Discrete and Continuous}

The core mathematical tool connecting discrete summation and continuous integration is \textbf{Poisson Summation Formula (PSF)}. It reveals profound duality between discretization of position space and periodization of momentum space.

\textbf{Theorem 5.1.2 (Poisson Summation in Distribution Sense)}

Let $f(x)$ be a rapidly decreasing function (Schwartz function) or more generally a band-limited function, then the following identity holds:

\[
\sum_{n \in \mathbb{Z}} f(an) = \frac{1}{a} \sum_{k \in \mathbb{Z}} \tilde{f}\left( \frac{2\pi k}{a} \right)
\]

where $\tilde{f}(\xi) = \int_{-\infty}^\infty f(x) e^{-i\xi x} dx$ is continuous Fourier transform.

\textbf{Physical Interpretation}:

Left side $\sum f(an)$ represents sum of physical quantities on QCA lattice (e.g., total charge, total energy).

First term on right side ($k=0$) corresponds to integral in continuous limit $\frac{1}{a} \tilde{f}(0) = \frac{1}{a} \int f(x) dx$.

Remaining terms on right side ($k \neq 0$) represent \textbf{corrections introduced by discretization}. These correction terms originate from spectral contributions at integer multiples of $2\pi/a$ in momentum space.

For \textbf{strictly band-limited} physical fields, if $\text{supp}(\tilde{f}) \subset (-\pi/a, \pi/a)$, then for all $k \neq 0$, $\tilde{f}(2\pi k/a) = 0$. At this point:

\[
\sum_{n \in \mathbb{Z}} a f(an) = \int_{-\infty}^\infty f(x) dx
\]

This shows: \textbf{Under band-limited conditions, discrete Riemann sum strictly equals continuous integral}. This explains why we can use calculus tools with extreme precision in discrete universe---because under high-energy cutoff, summation and integration are mathematically indistinguishable.

\subsection{Physical Sampling Theorem and Whittaker-Shannon Reconstruction}

Since discrete sum equals continuous integral, what about local values of the field itself? Can we reconstruct field value $\psi(x)$ at any point in space from QCA lattice data $\psi_n$?

\textbf{Theorem 5.1.3 (Physical Sampling Reconstruction Theorem)}

If continuous field $\psi(x)$ is band-limited (bandwidth $K = \pi/a$), then it is uniquely determined by its values $\{\psi(x_n)\}$ at lattice sites $x_n = an$, and can be perfectly reconstructed via \textbf{Whittaker-Shannon Interpolation Formula}:

\[
\psi(x) = \sum_{n \in \mathbb{Z}} \psi(x_n) \, \mathrm{sinc}\left( \frac{x - x_n}{a} \right)
\]

where $\mathrm{sinc}(z) = \frac{\sin(\pi z)}{\pi z}$ is reconstruction kernel.

\textbf{Proof Outline}:

In momentum space, discrete sampling causes periodic extension of spectrum: $\tilde{\psi}_{sample}(k) = \frac{1}{a} \sum_m \tilde{\psi}(k - \frac{2\pi m}{a})$. If $\psi$ is band-limited, then these extended spectral copies do not overlap. We can extract original spectrum $\tilde{\psi}(k)$ losslessly through a rectangular window function (low-pass filter) $\Pi(k) = \mathbb{1}_{[-\pi/a, \pi/a]}$. Inverse Fourier transform of $\Pi(k)$ is precisely $\mathrm{sinc}$ function. $\square$

\textbf{Physical Meaning: Emergence Mechanism of Continuity}

This theorem is the mathematical guarantee for Chapter 4's ``Field Theory Emergence.'' It tells us that although underlying ontology is discrete $\psi_n$, as long as we only care about low-energy physics ($E \ll 1/a$), we can \textbf{safely} regard $\psi_n$ as sample values of some smooth continuous field $\psi(x)$. This $\psi(x)$ is not a physical entity, but an \textbf{interpolation function} of $\psi_n$. All differential equations (such as Dirac equation) are actually operations on this interpolation function $\psi(x)$.

\subsection{Aliasing and Illusion of High-Energy Physics}

When energy of physical processes exceeds cutoff (e.g., extremely high-energy scattering or near black hole singularities), band-limited assumption fails. What happens then?

\textbf{Definition 5.1.4 (Aliasing and Umklapp Process)}

If field $\psi(x)$ contains high-frequency components $k_{high}$ exceeding Brillouin zone boundary $k_{max} = \pi/a$, spectrum after sampling will undergo aliasing: high-frequency component $k_{high}$ will ``fold'' back to low-frequency region, manifesting as pseudo-low-frequency momentum $k'_{low} = k_{high} - \frac{2\pi}{a}$.

In solid state physics, this is called \textbf{Umklapp Scattering} (reciprocal lattice vector scattering).

\textbf{Ontological Corollary}:

In QCA universe, \textbf{momentum space is toroidal, not flat}. There is no true ``ultra-high momentum'' $k > \pi/a$. So-called ``infinitely high-energy particles'' are mathematical fallacies in discrete ontology.

When we try to accelerate particles beyond Planck energy scale in accelerators, we will not explore finer spatial structures; instead, we will see particles ``pass through'' momentum space boundaries, appearing at low momentum in another equivalent momentum sector (or more physically, triggering black hole formation, cutting off degrees of freedom through holographic principle).

Therefore, Poisson summation formula is not only a computational tool; it delineates \textbf{applicability boundaries of Effective Field Theory (EFT)}:

\begin{enumerate}
\item \textbf{Band-Limited Region ($k \ll \pi/a$)}: PSF guarantees discrete $\cong$ continuous, differential geometry applies.

\item \textbf{Aliasing Region ($k \sim \pi/a$)}: Continuous reconstruction fails, must directly use discrete dynamics of QCA, explicit lattice effects (such as Lorentz violation) dominate.
\end{enumerate}

Through this section, we laid the first mathematical pillar for unification of discrete and continuous. In the next section, we will handle more subtle cases: when physical fields are not strictly band-limited (e.g., with exponentially decaying tails), how to precisely control errors introduced by discretization? This will lead to \textbf{Discrete-Continuous Error Control (DCEC)} and modern applications of Euler-Maclaurin formula.

