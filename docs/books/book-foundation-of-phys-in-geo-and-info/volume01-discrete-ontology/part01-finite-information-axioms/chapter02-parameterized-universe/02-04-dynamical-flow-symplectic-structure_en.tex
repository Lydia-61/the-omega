\section{Dynamical Flow and Symplectic Structure on Projective Hilbert Space $\mathbb{C}P^{N-1}$}

In previous sections, we revealed the dual geometric properties of quantum state space (projective Hilbert space $\mathbb{C}P^{N-1}$): on one hand, Riemannian metric $g_{\mu\nu}$ derived from distinguishability (Fisher information) endows it with ``rigid'' distance structure; on the other hand, Berry curvature $\Omega_{\mu\nu}$ derived from geometric phase (Berry phase) endows it with non-trivial topological structure.

This section will prove that these two structures are not isolated, but tightly coupled through complex structure (Complex Structure), together constituting a rigorous mathematical object---\textbf{Kähler Manifold}. More importantly, we will reveal that the dynamical equation of standard quantum mechanics---Schrödinger equation---is essentially classical Hamiltonian Flow on this curved phase space. This conclusion completely breaks the gap between ``quantum'' and ``classical'' in geometric form, showing that evolution of physical laws is essentially symplectic transformation of information geometry.

\subsection{Kähler Manifold: Perfect Unification of Riemannian and Symplectic}

In the ``It from Bit'' discrete ontology, physical states are described by finite-dimensional Hilbert spaces. We have seen that quantum geometric tensor $\chi_{\mu\nu}$ naturally decomposes into real and imaginary parts. This suggests that the underlying manifold structure has a certain ``trinity'' property.

\textbf{Definition 2.4.1 (Kähler Structure)}

Projective Hilbert space $\mathcal{M} = \mathbb{C}P^{N-1}$ is a Kähler manifold, meaning it simultaneously possesses the following three compatible geometric structures:

\begin{enumerate}
\item \textbf{Complex Structure $J$}: A linear map $J: T_p\mathcal{M} \to T_p\mathcal{M}$ satisfying $J^2 = -\mathbf{1}$, describing the operation of ``multiplying by $i$'' on tangent space.

\item \textbf{Riemannian Metric $g$}: Namely Fubini-Study metric (QFIM), satisfying $g(JX, JY) = g(X, Y)$, ensuring length invariance under complex rotation.

\item \textbf{Symplectic Form $\omega$}: A closed, non-degenerate antisymmetric 2-form, defined as:

\[
\omega(X, Y) \equiv g(JX, Y)
\]

This is precisely the (normalized) Berry curvature $\Omega_{\mu\nu}$.
\end{enumerate}

\textbf{Theorem 2.4.2 (Geometric Unification Theorem)}

On $\mathbb{C}P^{N-1}$, Riemannian geometry (metric) and symplectic geometry (curvature) mutually determine each other through complex structure. That is:

\[
\chi_{\mu\nu} = g_{\mu\nu} - i \omega_{\mu\nu}
\]

This means that as long as we know the ``distance'' (statistical distinguishability) between states, we automatically determine the ``curvature'' (symplectic structure) of space, and vice versa. The \textbf{Metric Rigidity} and \textbf{Symplectic Rigidity} of physical reality are two sides of the same coin.

\subsection{Geometricization of Schrödinger Equation: As Classical Hamiltonian System}

It is usually thought that Schrödinger equation $i\hbar \frac{d}{dt}|\psi\rangle = \hat{H}|\psi\rangle$ describes linear wave mechanics. However, from the perspective of projective geometry, it exhibits a surprising \textbf{nonlinear classical mechanics} appearance.

Consider $\mathbb{C}P^{N-1}$ as the ``true'' phase space of physical system (note: this is not position-momentum phase space, but phase space constituted by quantum states). Any Hermitian operator $\hat{H}$ defines a real-valued function on this space:

\[
h(\psi) = \langle \psi | \hat{H} | \psi \rangle
\]

This function $h: \mathbb{C}P^{N-1} \to \mathbb{R}$ is precisely the system's \textbf{quantum Hamiltonian function} (i.e., energy expectation value).

\textbf{Theorem 2.4.3 (Hamiltonian Nature of Schrödinger Flow)}

Unitary evolution of quantum states is equivalent to classical Hamilton equations with $h(\psi)$ as Hamiltonian and $\omega$ as symplectic form:

\[
\frac{d\xi^\mu}{dt} = \{ \xi^\mu, h \}_{\text{PB}}
\]

where $\xi^\mu$ are real coordinates of the manifold, $\{ \cdot, \cdot \}_{\text{PB}}$ is Poisson bracket defined by symplectic form $\omega$. The corresponding flow field vector $X_h$ satisfies:

\[
\omega(X_h, \cdot) = dh
\]

\textbf{Proof Outline}:

In complex coordinates $Z^k$, the Kähler potential of Fubini-Study metric is $K = \ln(1 + \bar{Z} \cdot Z)$. Symplectic form is $\omega = i \partial \bar{\partial} K$.

Schrödinger equation $i\hbar \dot{Z}^k = \frac{\partial h}{\partial \bar{Z}^k}$ exactly corresponds to Hamiltonian vector field equation $\iota_{X_h} \omega = dh$ on symplectic manifold (taking $\hbar=1$ and appropriate normalization).

This means that quantum evolution trajectories are \textbf{Hamiltonian flows} on phase space $\mathbb{C}P^{N-1}$, preserving energy $h(\psi)$ constant (energy conservation). $\square$

\textbf{Physical Interpretation}:

This conclusion profoundly reveals the essence of quantum mechanics: \textbf{quantum mechanics is not a negation of classical mechanics, but a concrete realization of classical Hamiltonian mechanics on complex projective manifolds}. The only difference is the geometric structure of phase space: classical mechanics occurs on flat $\mathbb{R}^{2n}$, while quantum mechanics occurs on compact curved $\mathbb{C}P^{N-1}$.

\subsection{Geometric Essence of Unitary Evolution: Symplectomorphism and Information Conservation}

In standard form, time evolution operator $U(t) = e^{-i\hat{H}t/\hbar}$ is unitary, i.e., $U^\dagger U = \mathbf{1}$. What does this correspond to in geometric language?

\textbf{Corollary 2.4.4 (Unitarity as Symplectomorphism)}

The one-parameter transformation group $\phi_t: \mathbb{C}P^{N-1} \to \mathbb{C}P^{N-1}$ generated by Schrödinger flow is a \textbf{Symplectomorphism}, i.e., it preserves symplectic form:

\[
\phi_t^* \omega = \omega
\]

At the same time, due to compatibility of Kähler structure, it is also an \textbf{Isometry}, preserving Riemannian metric $g$ unchanged.

\textbf{Quantum Version of Liouville's Theorem}:

Since the flow is symplectic, according to Darboux's Theorem, it necessarily preserves symplectic volume form $dV = \frac{1}{(N-1)!} \omega^{\wedge (N-1)}$ on the manifold.

This is \textbf{Liouville's Theorem} in quantum mechanics:

\begin{quote}
\textbf{``Probability volume'' in physical state space is incompressible during evolution.}
\end{quote}

This geometrically explains \textbf{unitary conservation} of quantum information: information can neither be created nor destroyed; it can only flow in state space. If evolution causes volume contraction (such as non-unitary measurement), it means information flows to the outside (environment); if evolution preserves volume, then the system is closed and information-conserving.

\subsection{Summary: From Information Statistics to Physical Geometry}

At this point, we have completed the construction of Part I of Volume I on ``Geometric and Information Foundations of Physics.'' Starting from the most fundamental finite information axiom, we step by step rebuilt the geometric edifice of physics:

\begin{enumerate}
\item \textbf{Ontology}: Physical reality is described by finite-dimensional Hilbert spaces (Chapter 1).

\item \textbf{Metric Structure}: Distinguishability of states (statistical distance) uniquely determines Riemannian metric $g_{\mu\nu}$ of space (Sections 2.1, 2.2).

\item \textbf{Gauge Structure}: Phase structure of states (geometric phase) uniquely determines symplectic form $\Omega_{\mu\nu}$ and gauge fields of space (Section 2.3).

\item \textbf{Dynamical Structure}: Schrödinger equation is merely Hamiltonian flow preserving information conservation (symplectic volume unchanged) on this geometric structure (Section 2.4).
\end{enumerate}

The core conclusion of this chapter can be summarized as: \textbf{Geometry is not an a priori stage, but statistical properties of information; dynamics is not arbitrary rules, but geometric necessity of information conservation.}

In the next Part II, we will leave static geometric structures and enter the core of discrete dynamics: if both continuous time and space are emergent, how exactly does the most fundamental ``evolution'' occur? We will introduce \textbf{Quantum Cellular Automata (QCA)} as the dynamical engine of discrete ontology.

