\section{Strong CP Problem and Axion: Dynamical Relaxation Mechanism of Topological Phases}

In Sections 17.1 and 17.2, we established the topological origin of matter: fermions are $\mathbb{Z}_2$ knots in spacetime structure, and spin and statistics are geometric manifestations of this topological structure. However, in the Standard Model of particle physics, there exists a more subtle topological problem—the \textbf{Strong CP Problem}.

The vacuum structure of Quantum Chromodynamics (QCD) allows a $\theta$-term that violates CP symmetry (charge-parity). Although $\theta$ can theoretically take any value from $0$ to $2\pi$, experimental observations (such as neutron electric dipole moment) show $\theta < 10^{-10}$. This extreme fine-tuning cannot be explained within the Standard Model framework.

This section will use \textbf{QCA discrete ontology} and topological dynamics of \textbf{self-referential scattering networks} to provide a natural geometric solution. We will prove that the $\theta$ parameter in QCA universes is not a fixed constant, but a \textbf{dynamical phase field}—the \textbf{Axion}. Just as physical systems tend to evolve to lowest energy states, QCA network topological structures tend to drive this phase to zero through \textbf{dynamical relaxation}, thereby naturally restoring CP symmetry.

\subsection{QCD Vacuum and Topological Nature of $\theta$-Angle}

In non-Abelian gauge field theory (such as QCD), vacuum states have non-trivial topological structure. Classical vacua ($F_{\mu\nu}=0$) constitute infinitely many topologically inequivalent classes, labeled by \textbf{Pontryagin Index} $n \in \mathbb{Z}$ (also called winding number).

The true quantum vacuum is superposition of these topological sectors, i.e., \textbf{$\theta$-vacuum}:
$$|\theta\rangle = \sum_{n=-\infty}^{\infty} e^{i n \theta} |n\rangle$$

This parameter $\theta$ appears as a topological term in the Lagrangian:
$$\mathcal{L}_{\theta} = \frac{\theta g^2}{32\pi^2} F_{\mu\nu}^a \tilde{F}^{a\mu\nu}$$

This term violates P and CP symmetry. Since it is a total divergence term (topological term), it does not affect classical equations of motion, but produces observable consequences at quantum level through instanton effects.

\textbf{QCA Perspective}:

In Section 16.3, we interpreted gauge field strength as curvature of total space plaquette operator $\mathbb{W}_{\square}$.

The $\theta$-term corresponds to \textbf{weighted phase of overall topological defects} in QCA networks.

\begin{itemize}
\item $n$ is the total number of "twists" in the network.

\item $\theta$ is the phase cost per additional twist.
\end{itemize}

If $\theta$ is a fixed external parameter (as Standard Model assumes), then QCA networks are "frozen" in a specific CP-violating configuration.

\subsection{Geometrization of Peccei-Quinn Mechanism}

To solve the strong CP problem, Peccei and Quinn (1977) introduced a new global symmetry $U(1)_{PQ}$, whose spontaneous breaking produces a pseudoscalar particle—the \textbf{Axion} $a(x)$. This elevates static parameter $\theta$ to a dynamic field:
$$\theta \to \theta_{eff}(x) = \theta + \frac{a(x)}{f_a}$$

This section will prove that this mechanism is not artificially introduced in QCA universes, but an inevitable manifestation of \textbf{network topological degrees of freedom}.

\begin{theorem}[Dynamicalization of Topological Phases]
\label{thm:dynamicalization}
In QCA universes satisfying the finite information axiom, there are no completely "rigid" global parameters. Any topological phase $\theta$ (such as Berry phase integrals in Section 17.1) corresponds to a \textbf{macroscopic collective mode} of network connection states.

Therefore, $\theta$ must be regarded as a field $\theta(x)$ that slowly varies with time and space. This field is the axion field.
\end{theorem}

\textbf{Microscopic Mechanism}:

Consider self-referential loops in QCA networks. The determinant phase $\arg \det M$ of their effective mass matrix $M$ (arising from left-right chirality coupling) contributes part of the $\theta$ value.

In self-referential scattering networks (SSN), this coupling is dynamic. When fermions propagate in networks, the phase of their chirality flip depends on local connection states of the network. Fluctuations of these connection states are precisely axions.

\subsection{Axion Potential and Topological Relaxation}

If $\theta$ is dynamic, what value will it evolve to? This depends on the effective potential $V(\theta)$ of the system.

In QCD, instanton effects produce a periodic potential:
$$V(\theta) \approx \Lambda_{QCD}^4 (1 - \cos \theta)$$

This potential takes minimum at $\theta = 0$.

\begin{theorem}[Topological Relaxation Theorem]
\label{thm:topological-relaxation}
In QCA universes, total free energy (or negative of generalized entropy) of the system contains penalty terms for network topological complexity.

For configurations carrying topological charge $Q_{top} \propto \int F \tilde{F}$, corresponding spectral shift function $\xi(E)$ undergoes overall offset.

According to trace formulas (Section 7.2) and energy minimization principle (or entropy maximization principle), QCA networks always tend to \textbf{untie unnecessary topological knots}.

This "unknotting" process manifests as \textbf{damped relaxation} of field $\theta(x)$ toward potential minimum $\theta=0$.
\end{theorem}

\textbf{Physical Picture}:

Imagine QCA networks as elastic fabric.

\begin{itemize}
\item $\theta \neq 0$ means the fabric is overall twisted. This twist stores elastic energy.

\item Axion field $a(x)$ is the twist wave of the fabric.

\item As the universe evolves (cools), the fabric releases tension, dissipating topological energy by producing axion waves, eventually settling in untwisted state $\theta = 0$.
\end{itemize}

This is why we do not observe strong CP violation in today's universe—\textbf{the universe has completed topological "annealing" through the axion mechanism}.

\subsection{Axion as Topological Candidate for Dark Matter}

Axion waves produced during relaxation do not disappear; they remain in the universe as \textbf{coherent topological waves}.

Since their interactions with ordinary matter are extremely weak (limited by large scale $f_a$), these axions constitute ideal candidates for \textbf{Cold Dark Matter}.

\begin{corollary}[Geometric Essence of Dark Matter]
\label{cor:dark-matter-geometric}
In QCA theory, dark matter (axions) is not a new type of particle, but \textbf{"twist texture" of vacuum geometry}.

\begin{itemize}
\item Ordinary matter (fermions): Localized, $\mathbb{Z}_2$-protected \textbf{strong knots}.

\item Axion dark matter: Non-local, weak, large-scale \textbf{background twists}.
\end{itemize}

Both are different manifestations of spacetime topological structure.
\end{corollary}

\textbf{Conclusion}

The strong CP problem is no longer a fine-tuning puzzle in QCA discrete ontology.

\begin{enumerate}
\item \textbf{Parameter Dynamicalization}: Topological parameter $\theta$ must be a dynamic field (axion).

\item \textbf{Energy Minimization}: Topological tension of networks drives $\theta$ to relax to $0$, restoring CP symmetry.

\item \textbf{Dark Matter Remnant}: Topological waves left from relaxation process constitute dark matter.
\end{enumerate}

This shows that \textbf{symmetry (CP conservation) is not designed, but evolved}. The universe automatically smooths out initial topological wrinkles through dynamical mechanisms.

