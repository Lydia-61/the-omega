\section{Raychaudhuri Equation: Precise Description of Geometric Focusing Effects}

In Section 11.1, we found that the area deficit $\delta A$ of the causal diamond edge directly encodes spacetime curvature $R_{\mu\nu}$. In Section 11.3, we established the connection between entanglement entropy changes and energy-momentum tensor flux. Now, to complete the derivation of gravitational dynamics, we need a differential equation that describes \textbf{how geometric evolution responds to energy-matter}.

This section will introduce one of the most important geometric identities in general relativity—the \textbf{Raychaudhuri Equation}. This equation not only precisely describes the focusing behavior of geodesic congruences, but is also the mathematical hub connecting geometry (curvature) with causal structure (light cone contraction). We will prove that in the framework of the generalized entropic variational principle (IGVP), the Raychaudhuri equation is the geometric preparatory form of Einstein's field equations, forcing matter with positive energy density to cause gravitational focusing of spacetime.

\subsection{Kinematic Decomposition of Geodesic Congruences}

Consider a set of null (or timelike) geodesic congruences in spacetime, with tangent vector field $k^\mu$ (satisfying $k^\nu \nabla_\nu k^\mu = 0$). This set of geodesics can be viewed as a collection of worldlines of light rays or freely falling observers. To describe the geometric evolution of this bundle, we examine the change in transverse deviation vectors (Deviation Vector) $\eta^\mu$.

\begin{definition}[Expansion, Shear, and Vorticity]
\label{def:expansion-shear-vorticity}
We project the covariant derivative tensor $B_{\mu\nu} = \nabla_\nu k_\mu$ of the tangent vector field onto the two-dimensional transverse cross-section (screen space) orthogonal to $k^\mu$, and decompose it into irreducible components:
$$B_{\mu\nu} = \frac{1}{d-2} \theta h_{\mu\nu} + \sigma_{\mu\nu} + \omega_{\mu\nu}$$

where:

\begin{enumerate}
\item \textbf{Expansion Scalar $\theta$}: Describes the relative rate of change of cross-sectional area $A$, $\theta = \nabla_\mu k^\mu = \frac{1}{A} \frac{dA}{d\lambda}$.

\item \textbf{Shear Tensor $\sigma_{\mu\nu}$}: Describes shape deformation of the cross-section (area-preserving), a traceless symmetric tensor.

\item \textbf{Vorticity Tensor $\omega_{\mu\nu}$}: Describes rigid rotation of the cross-section, an antisymmetric tensor. For hypersurface-orthogonal geodesic congruences (such as wavefronts), $\omega_{\mu\nu} = 0$.
\end{enumerate}
\end{definition}

\subsection{Derivation of the Raychaudhuri Equation}

To obtain the rate of change of $\theta$ with respect to affine parameter $\lambda$, we differentiate $B_{\mu\nu}$ along the geodesic:
$$\frac{d B_{\mu\nu}}{d\lambda} = k^\rho \nabla_\rho (\nabla_\nu k_\mu) = k^\rho (\nabla_\nu \nabla_\rho k_\mu + R_{\mu\rho\nu}{}^\sigma k_\sigma)$$

Using the geodesic equation $k^\rho \nabla_\rho k_\mu = 0$ and the definition of the Riemann tensor, we obtain the evolution equation for $B_{\mu\nu}$. Taking its trace yields the Raychaudhuri equation.

\begin{theorem}[Raychaudhuri Equation]
\label{thm:raychaudhuri}
For vorticity-free ($\omega_{\mu\nu}=0$) null geodesic congruences, the expansion scalar $\theta$ satisfies:
$$\frac{d\theta}{d\lambda} = -\frac{1}{d-2} \theta^2 - \sigma_{\mu\nu} \sigma^{\mu\nu} - R_{\mu\nu} k^\mu k^\nu$$

where $R_{\mu\nu}$ is the Ricci curvature tensor.
\end{theorem}

\textbf{Physical Interpretation}:

The three terms on the right side of this equation determine the second-order change (acceleration) of the beam cross-sectional area:

\begin{enumerate}
\item \textbf{$-\frac{1}{d-2}\theta^2$}: Always non-positive. This represents that once a beam begins to contract, it accelerates contraction (self-focusing).

\item \textbf{$-\sigma^2$}: Always non-positive. Shear (shape deformation) always consumes transverse kinetic energy, causing beam focusing.

\item \textbf{$-R_{\mu\nu} k^\mu k^\nu$}: This term is the \textbf{direct contribution of spacetime geometry to focusing}.
\end{enumerate}

\subsection{Focusing Theorem and Energy Conditions}

The most profound corollary of the Raychaudhuri equation lies in its establishment of an inequality relationship between \textbf{geometric curvature and matter energy}.

\begin{definition}[Null Energy Condition / NEC]
\label{def:nec}
If for any null vector $k^\mu$, the matter energy-momentum tensor satisfies $T_{\mu\nu} k^\mu k^\nu \ge 0$, then matter is said to satisfy the null energy condition. This means that the energy density seen by any local observer (projected onto the light-speed flow) is non-negative.

If Einstein's equation $R_{\mu\nu} k^\mu k^\nu = 8\pi G T_{\mu\nu} k^\mu k^\nu$ holds, then NEC implies the geometric term $R_{\mu\nu} k^\mu k^\nu \ge 0$.
\end{definition}

Combining with the expression for $\frac{d\theta}{d\lambda}$, we can conclude:

\begin{theorem}[Classical Focusing Theorem]
\label{thm:focusing}
If matter satisfies the null energy condition (NEC), then the expansion rate of null geodesic congruences monotonically decreases:
$$\frac{d\theta}{d\lambda} \le 0$$

This means that gravity always manifests as \textbf{attraction}. Beams in gravitational fields always tend to converge, leading to the formation of caustics or singularities (the basis of the Penrose-Hawking singularity theorem).
\end{theorem}

\subsection{Role in Entropic Variational Principle: From Geometry to Dynamics}

In the IGVP framework constructed in this book, we have not yet assumed Einstein's equations. Instead, we use the Raychaudhuri equation as a purely geometric tool to derive the field equations.

Consider a set of null generating lines of a small causal diamond. At $\lambda=0$ (the maximal cross-section $\partial \Sigma$ of the diamond), we can set the initial expansion rate $\theta|_0 = 0$ (extremal surface condition).

According to the Raychaudhuri equation, the expansion rate at $\lambda$ is:
$$\theta(\lambda) \approx - \lambda R_{\mu\nu} k^\mu k^\nu$$

This leads to second-order change in cross-sectional area $A(\lambda)$:
$$\delta A(\lambda) \approx -\frac{1}{2} A_0 \lambda^2 R_{\mu\nu} k^\mu k^\nu$$

This is precisely the differential form of the area deficit formula in Section 11.1.

Now, we combine this geometric fact with the first law of entanglement from Section 11.3:

\begin{enumerate}
\item \textbf{Geometric Entropy Change}: $\delta S_{\text{grav}} \propto \delta A \propto - R_{\mu\nu} k^\mu k^\nu$.

\item \textbf{Matter Entropy Change}: $\delta S_{\text{mat}} \propto \delta \langle K \rangle \propto T_{\mu\nu} k^\mu k^\nu$.
\end{enumerate}

\textbf{Entropic Equilibrium Condition}:

Requiring the second-order variation of total entropy $S_{\text{gen}}$ on any small causal diamond to vanish (or satisfy a specific conservation flow relation):
$$\delta^2 S_{\text{gen}} = \delta^2 S_{\text{grav}} + \delta^2 S_{\text{mat}} = 0$$

This will lead to:
$$-\frac{1}{4G} R_{\mu\nu} k^\mu k^\nu + 2\pi T_{\mu\nu} k^\mu k^\nu = 0$$

Since $k^\mu$ is an arbitrary null vector, stripping $k^\mu$ yields the core part of Einstein's field equations $R_{\mu\nu} \propto T_{\mu\nu}$.

\textbf{Conclusion}

The Raychaudhuri equation is a rigid link connecting \textbf{geometric deficit (area)} with \textbf{curvature tensor}. It ensures that as long as we accept the holographic principle (entropy proportional to area) and the first law of entanglement (entropy change proportional to energy), the gravitational field equations are no longer assumptions, but \textbf{inevitable corollaries of geometric statistical mechanics}.

At this point, all geometric preparations for Part VII are complete. We have:

\begin{enumerate}
\item Probe: Small causal diamonds (11.1);
\item Potential function: Generalized entropy $S_{\text{gen}}$ (11.2);
\item Matter response: First law of entanglement (11.3);
\item Geometric response: Raychaudhuri equation (11.4).
\end{enumerate}

In the upcoming \textbf{Chapter 12: Entropic Variational Principle (IGVP) and Field Equations}, we will formally assemble these components to derive the complete Einstein field equations and their higher-order corrections.

