\section{Horizon as Information Truncation Surface in QCA Networks}

In standard general relativity, event horizons are defined as the past boundary of future null infinity. This is a definition that depends on the global structure of spacetime. In QCA discrete ontology, we need a more operational, local definition of horizons. This section formalizes horizons as \textbf{information flow traps} in discrete causal networks and argues that their physical essence is the \textbf{partial trace} operation on the total system Hilbert space.

\subsection{Horizon Definition in Discrete Causal Networks}

Consider the network graph $G = (\Lambda, E)$ of a QCA universe. Due to the locality of dynamical evolution $U$, information propagation is limited by finite light cone structures.

\begin{definition}[Algebraic Black Hole Region and Horizon]
\label{def:bh-horizon}
Let $\mathcal{A}_{\text{obs}}$ be the von Neumann algebra generated by local operator algebras accessible to external observers (located at "infinity" or asymptotically flat regions).

The total system Hilbert space decomposes as $\mathcal{H}_{\text{total}} = \mathcal{H}_{\text{obs}} \otimes \mathcal{H}_{\text{bh}}$ (assuming tensor product structure approximately holds).

The \textbf{black hole region} $\mathcal{H}_{\text{bh}}$ is defined as the set of all lattice points whose causal future cannot reach $\mathcal{A}_{\text{obs}}$.

The \textbf{discrete horizon} $\Sigma_H$ is defined as the \textbf{interface} between these two regions, i.e., the set of all edges connecting lattice points inside $\mathcal{H}_{\text{bh}}$ with those outside $\mathcal{H}_{\text{obs}}$:
$$\Sigma_H = \{ (x, y) \in E \mid x \in \Lambda_{\text{bh}}, y \in \Lambda_{\text{obs}} \}$$

Or on the dual graph, the horizon is the closed surface cutting these edges.
\end{definition}

\subsection{Information Truncation and Mixed State Generation}

For external observers, the state inside the black hole is unknowable. Physical "unknowability" in quantum mechanics corresponds to the \textbf{partial trace} operation.

Let the entire universe be in pure state $|\Psi\rangle$. The physical state of external observers is described by the reduced density matrix $\rho_{\text{obs}}$:
$$\rho_{\text{obs}} = \text{Tr}_{\text{bh}} (|\Psi\rangle\langle\Psi|)$$

Since $|\Psi\rangle$ is a highly entangled state of the total system (QCA evolution necessarily produces entanglement), the reduced $\rho_{\text{obs}}$ must be a \textbf{mixed state}.

\begin{theorem}[Thermality Induced by Horizon]
\label{thm:horizon-thermality}
Even if the entire universe is in a zero-temperature vacuum pure state, as long as a horizon $\Sigma_H$ severs entanglement bonds, the $\rho_{\text{obs}}$ seen by external observers manifests as a thermal state with non-zero von Neumann entropy:
$$S(\rho_{\text{obs}}) = -\text{Tr}(\rho_{\text{obs}} \ln \rho_{\text{obs}}) > 0$$

This is the \textbf{entanglement origin} of black hole entropy. The horizon is not a physical membrane, but an \textbf{information truncation surface}. The "thermal radiation" of black holes (Hawking radiation) is actually quantum noise caused by this entanglement truncation.
\end{theorem}

\subsection{Holographic Bit Counting and Area Law}

In continuous field theory, entanglement entropy across boundaries is usually UV-divergent. But in QCA discrete ontology, due to finite information density (Chapter 1), the number of edges crossing the horizon is finite.

\begin{lemma}[Microscopic Geometry of Area Law]
\label{lem:area-law-microscopic}
Let the lattice point density of the QCA network be $1/l_P^3$ (one lattice point per Planck volume, or more precisely, one link per Planck area).

The horizon $\Sigma_H$ is a two-dimensional surface with area $A$. The number of "edges" crossing this surface $N_{links}$ is roughly estimated as:
$$N_{links} \approx \frac{A}{l_P^2}$$

Each severed edge represents a pair of entangled qubits (one inside, one outside), contributing $O(1)$ bits of entanglement entropy.

Therefore, total entropy $S$ must be proportional to horizon area $A$:
$$S_{\text{bh}} \propto N_{links} \propto A$$

This directly gives the \textbf{topological interpretation} of the Bekenstein-Hawking entropy formula $S \propto A$: black hole entropy is the number of QCA communication channels severed by the horizon.
\end{lemma}

\subsection{Edge Modes and Surface Algebra}

The severed edges on the horizon carry \textbf{edge modes}—degrees of freedom that cannot be assigned purely to the inside or outside. These edge modes form a surface algebra $\mathcal{A}_{\Sigma}$.

In gauge field theory and gravity, edge modes are necessary to restore factorization of Hilbert space. They carry gauge charges (or diffeomorphism charges) that ensure Gauss's law is satisfied on both sides of the horizon.

The dimension of this surface algebra is:
$$\dim \mathcal{A}_{\Sigma} \sim 2^{N_{links}} \sim 2^{A/l_P^2}$$

Taking the logarithm gives entropy $S \sim A/l_P^2$, consistent with the area law.

\textbf{Summary}

This section established that black hole horizons are information truncation surfaces in QCA networks. Black hole entropy arises from counting severed entanglement bonds, naturally giving the area law. In the next section, we will perform precise microscopic counting to derive the coefficient $1/4$ in $S = A/4G$.

