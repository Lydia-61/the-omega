\section{Time Delay Measurement in Mesoscopic Conductors: Electron Interference and Quantum Capacitance}

In Section 26.1, we used macroscopic microwave chaotic cavities to verify validity of unified time identity $\kappa(E)=\rho(E)$ for boson (photon) fields. However, material foundation in QCA ontology is fermions (Chapter 17). Fermions obey Pauli exclusion principle, their ``density of states'' directly relates to particle number filling and charge accumulation.

This section will advance experimental verification to \textbf{Mesoscopic Electronics} field. We will prove that in quantum dots or mesoscopic interferometers, \textbf{Scattering Phase Shift} of electrons not only determines time delay, but also directly defines \textbf{Quantum Capacitance} of system. This experimental fact provides most direct electrical evidence for ``time is matter'': \textbf{longer residence time, greater system's ability to store charge (matter)}.

\subsection{Quantum Dots as Physical Realization of QCA Nodes}

Mesoscopic quantum dots (Quantum Dot, QD) are artificially manufactured ``artificial atoms,'' connected to external leads (Source/Drain) through tunnel junctions.

In QCA discrete model, quantum dots perfectly correspond to \textbf{Scattering Nodes} in networks.

\begin{itemize}
\item \textbf{Input/Output}: Electron wave functions $\psi_{in/out}$ transmit through leads.
\item \textbf{Internal States}: Discrete energy levels $\{E_n\}$ inside quantum dot correspond to internal Hilbert space of QCA.
\item \textbf{Control Parameters}: Gate voltage $V_g$ can adjust internal potential, equivalent to scanning energy $E$.
\end{itemize}

Our goal is to verify that for fermion systems, Wigner-Smith time delay $\tau_W$ and local density of states $\rho(E)$ still satisfy strict linear relationship:

\[
\tau_W(E) = 2\pi \hbar \rho(E)
\]

\subsection{Theoretical Bridge: Quantum Capacitance and RC Time Constant}

In macroscopic circuits, capacitance $C = dQ/dV$ is purely geometric quantity. But at mesoscopic scale, when we add an electron to quantum dot, must pay two parts of energy: classical Coulomb repulsion energy $E_C$ and quantum level spacing $\Delta E$.

\textbf{Definition 26.2.1 (Electrochemical Capacitance)}

Total capacitance $C_{\mu}$ of system is defined as reciprocal of chemical potential change $d\mu$ required to inject charge $dN$:

\[
\frac{1}{C_{\mu}} = \frac{1}{e^2} \frac{d\mu}{dN} = \frac{1}{C_{geo}} + \frac{1}{C_q}
\]

where $C_{geo}$ is geometric capacitance, and \textbf{Quantum Capacitance} $C_q$ is directly defined by density of states:

\[
C_q(E) = e^2 \rho(E)
\]

\textbf{Corollary 26.2.2 (Time-Capacitance Identity)}

Combining unified time identity $\tau = h \rho$ (here $h=2\pi\hbar$), we obtain a surprising electrical relationship:

\[
\tau_W(E) = \frac{h}{e^2} C_q(E) = R_K C_q(E)
\]

where $R_K = h/e^2 \approx 25.8 \, \text{k}\Omega$ is \textbf{von Klitzing Constant}, i.e., resistance quantum.

\textbf{Physical Meaning}:

This formula shows that time delay $\tau_W$ of microscopic scattering is essentially an \textbf{RC time constant}, where ``resistance'' is universal vacuum impedance $R_K$, and ``capacitance'' is quantum capacitance $C_q$ characterizing material density of states.

This not only verifies Friedel sum rule ($\Delta N = \frac{1}{\pi} \delta$), but transforms abstract ``time'' into measurable ``capacitance.'' \textbf{Objects have mass (can store charge/information) because they can delay time.}

\subsection{Experimental Scheme: Aharonov-Bohm (AB) Interferometer}

To directly measure scattering phase $\delta(E)$ (thereby obtaining $\tau_W \sim d\delta/dE$), we need a phase-sensitive experimental setup.

\textbf{Experimental Setup}:

\begin{enumerate}
\item \textbf{AB Ring}: Construct a mesoscopic-scale conducting ring, electrons can flow from source (S) to drain (D).
\item \textbf{Embedded Quantum Dot}: Embed a quantum dot (QCA node under test) in one arm of ring, other arm as reference path.
\item \textbf{Magnetic Flux Control}: Magnetic flux $\Phi$ through ring introduces controllable Aharonov-Bohm phase $\varphi_{AB} = 2\pi \Phi/\Phi_0$.
\end{enumerate}

\textbf{Measurement Principle}:

Transmission amplitude $t_{QD} = |t| e^{i\delta}$ through quantum dot interferes with reference arm amplitude $t_{ref}$. Total conductance $G$ oscillates with magnetic flux $\Phi$:

\[
G(\Phi) \propto |t_{QD} + t_{ref} e^{i\varphi_{AB}}|^2 = A + B \cos(\varphi_{AB} + \delta)
\]

By measuring phase shift of interference fringes, we can directly read \textbf{transmission phase $\delta(V_g)$} of quantum dot.

\textbf{Experimental Procedure}:

\begin{enumerate}
\item Adjust gate voltage $V_g$, changing alignment of quantum dot energy levels with Fermi surface (scanning energy $E$).
\item At each $V_g$ point, scan magnetic flux $\Phi$, extract phase $\delta$.
\item Calculate phase derivative $\frac{d\delta}{dV_g} \propto \tau_W$.
\item Simultaneously, independently measure density of states $\rho(E)$ or quantum capacitance $C_q$ through width and spacing of Coulomb blockade peaks.
\end{enumerate}

\subsection{Experimental Evidence and Correction of Phase Lapse}

Early AB interference experiments (such as Yacoby et al., 1995) indeed observed continuous evolution of phase with energy, but also discovered a phenomenon called ``Phase Lapse'': between two resonance peaks, phase sometimes undergoes $\pi$ jump.

This was once considered challenge to Friedel rule. But under QCA framework, this receives perfect explanation:

\begin{itemize}
\item \textbf{Multi-channel Effects}: Real quantum dots are not single-channel. According to $\mathsf{Q}$ matrix theory in Section 6.4, measured phase is phase $\theta_t$ of some component of $S$ matrix, while Friedel rule relates to total phase $\Phi = \det S$.
\item \textbf{Universal Verification}: Subsequent more precise experiments (such as Schuster et al., 1997) by controlling channel number confirmed that in single-channel limit, phase evolution $\Delta \delta$ precisely equals $\pi$ when passing through a resonance peak.
\end{itemize}

This provides strong experimental support for unified time identity in fermion systems, and demonstrates that \textbf{time delay and quantum capacitance are two sides of the same coin}---both measure how much ``room'' system has to accommodate additional charge/information.

