\section{Microwave Scattering Networks: Verifying $\kappa(E)=\rho(E)$ Identity Using Chaotic Cavities}

Core conclusion of Volume II of this book is \textbf{Unified Time Identity} (Section 8.1):

\[
\kappa(E) \equiv \frac{1}{\pi} \frac{d\varphi}{dE} = \rho_{\text{rel}}(E)
\]

That is: derivative of scattering phase (time delay) strictly equals local density of states of the system. This identity establishes equivalence between dynamics (time) and thermodynamics (matter). Although this relationship has indirect evidence in nuclear physics, we need a \textbf{macroscopic, controllable, high-precision} experimental platform to directly verify this ontological duality.

This section will introduce how to use \textbf{Chaotic Microwave Cavities} as analog computers for quantum chaotic scattering, reproducing and verifying this cosmological-scale identity in laboratories through precision measurements of Vector Network Analyzers (VNA).

\subsection{Quantum-Microwave Analogy Principle}

In two-dimensional flat microwave cavities, electric field component $E_z$ of electromagnetic waves in cavity of height $d$ satisfies Helmholtz equation:

\[
(\nabla^2 + k^2) E_z(x, y) = 0, \quad k = \frac{2\pi f}{c}
\]

This is completely isomorphic in mathematical form to two-dimensional stationary Schrödinger equation $(\nabla^2 + \frac{2mE}{\hbar^2}) \psi = 0$.

\begin{itemize}
\item \textbf{Microwave frequency $f$} corresponds to \textbf{quantum energy $E$}.
\item \textbf{Electric field strength $E_z$} corresponds to \textbf{wave function $\psi$}.
\item \textbf{Cavity wall boundary conditions} (conductor) correspond to \textbf{infinite potential wells}.
\end{itemize}

Therefore, a macroscopic copper cavity (size $\sim$10--50 cm) can perfectly simulate a nanoscale quantum billiard. By designing cavity shape as \textbf{Stadium} or \textbf{Sinai Billiard}, we can create scattering systems with highly chaotic dynamics, simulating complex entanglement and scattering processes in QCA networks.

\subsection{Experimental Setup and Scattering Matrix Measurement}

\textbf{Experimental Setup}:

\begin{enumerate}
\item \textbf{Chaotic Cavity}: An irregularly shaped (breaking symmetry) flat metal cavity to ensure internal wave dynamics is ergodic.
\item \textbf{Ports (Channels)}: Insert $N$ microwave antennas (coaxial cable probes) into cavity walls. These probes act as input/output ports in QCA networks, connecting ``external world'' with ``internal black box.''
\item \textbf{Measurement Instrument}: Use precision \textbf{Vector Network Analyzer (VNA)} to connect ports.
\end{enumerate}

\textbf{Measured Quantities}:

VNA directly measures complex scattering parameters $S_{nm}(f)$ (scattering matrix elements):

\[
S_{nm}(f) = |S_{nm}(f)| e^{i \phi_{nm}(f)}
\]

where $|S_{nm}|$ is modulus of transmission/reflection coefficient, $\phi_{nm}$ is phase. This not only gives probability amplitudes, but more crucially gives \textbf{phase information}.

\subsection{Experimental Extraction of $\kappa(E)$: Wigner Time Delay}

According to definition in Chapter 6, total time delay (Wigner Time Delay) of system is frequency derivative of phase of scattering matrix $S(f)$.

\textbf{Experimental Step 1: Construct Scattering Matrix}

In wide frequency band (e.g., 1 GHz -- 20 GHz), perform full-port measurements on all $N$ ports, obtaining complete $N \times N$ scattering matrix $S(f)$.

\textbf{Experimental Step 2: Calculate Unified Time Density $\kappa_{exp}$}

Calculate total phase of scattering matrix determinant $\Phi(f) = \text{Im} \ln \det S(f)$.

Through numerical differentiation, obtain experimental unified time scale density:

\[
\kappa_{exp}(f) = \frac{1}{\pi} \frac{d\Phi(f)}{df}
\]

This physical quantity directly corresponds to average residence time of photons in cavity (multiplied by $2\pi$). Near resonance peaks, $\kappa_{exp}$ exhibits sharp Lorentz-type spikes.

\subsection{Experimental Extraction of $\rho(E)$: Eigenmode Counting}

To verify identity, we need to independently measure \textbf{Density of States (DOS)} of system.

\textbf{Method A: Resonance Fitting (Low Frequency Region)}

In low frequency region, resonance peaks are separated. We can identify each eigenfrequency $f_n$ by fitting Breit-Wigner peak shapes of $S_{nm}(f)$.

Density of states is given by broadening of Dirac $\delta$ functions:

\[
\rho_{exp}(f) = \sum_n \delta_\Gamma (f - f_n)
\]

where $\delta_\Gamma$ is Lorentz function with finite width.

\textbf{Method B: Weyl's Law Baseline (High Frequency Region)}

In high frequency region, modes overlap, cannot count individually. At this point utilize \textbf{Weyl's Law} in spectral geometry (Section 8.4.3):

\[
N_{Weyl}(f) = \frac{A \cdot 2\pi f^2}{c^2} - \frac{L \cdot f}{c} + \dots
\]

where $A$ is cavity area, $L$ is perimeter. Density of states is derivative: $\rho_{Weyl}(f) = dN_{Weyl}/df$.

\textbf{Verification Criterion}:

Compare $\kappa_{exp}(f)$ and $\rho_{exp}(f)$ point by point. If unified time identity holds, they should coincide within experimental error:

\[
\kappa_{exp}(f) \approx \rho_{exp}(f) \quad (\text{within } \sim 1\%)
\]

This would be first direct macroscopic verification that \textbf{time delay equals density of states}, proving ``time is matter'' at experimental level.

