\section{Intuitionistic Logic: Logical Duality of Quantum Superposition and Failure of Law of Excluded Middle}

In Section 24.1, we established that mathematical model of physical reality is not static sets in set theory, but dynamic presheaves in topos. This structural change brings fundamental reconstruction of logical laws. In classical physics, we default to using \textbf{Boolean Logic}, whose core pillar is \textbf{Law of Excluded Middle (LEM)}: any physical proposition $P$ (e.g., "electron is at A"), is either true or false ($P \lor \neg P = \text{True}$).

However, in QCA quantum universe, existence of superposition states challenges this binary logic. This section will prove that intrinsic logic of quantum mechanics is \textbf{Intuitionistic Logic}. We will show that superposition states are not logical paradoxes of "both A and B," but natural results of failure of law of excluded middle. In topos logic, $\neg \neg P$ (not-not P) is not equivalent to $P$, and this logical gap is precisely logical origin of \textbf{quantum uncertainty} and \textbf{measurement collapse}.

\subsection{Heyting Algebra of Physical Propositions}

In classical phase space, propositions correspond to subsets (Borel sets), which form a Boolean algebra. But in physical topos $\mathcal{T}_\mathcal{A}$, propositions correspond to \textbf{Subobjects}, which form a \textbf{Heyting Algebra}.

\begin{definition}[Truth Value of Physical Propositions]
\label{def:truth-value}
Let $P$ be a proposition about physical system (e.g., "energy $E \in [E_1, E_2]$"). In topos $\mathcal{T}_\mathcal{A}$, truth value $[\![ P ]\!]$ of $P$ is not simply $\{0, 1\}$, but a \textbf{Sieve} on base category $\mathcal{V}(\mathcal{A})$.

Specifically, $[\![ P ]\!] (V)$ is a set containing all observation contexts (commutative subalgebras) $V$ where proposition $P$ is true in classical approximation.

Key difference between Heyting algebra and Boolean algebra lies in definition of \textbf{Negation}:

\begin{itemize}
\item \textbf{Boolean Negation}: $\neg P$ is complement of $P$.

\item \textbf{Heyting Negation}: $\neg P$ is \textbf{Pseudo-complement}. It is the maximum element of set of all contexts \textbf{incompatible} with $P$.
$$[\![ \neg P ]\!] = \bigvee \{ Q \mid Q \land P = \bot \}$$
\end{itemize}
\end{definition}

\subsection{Failure of Law of Excluded Middle and Superposition}

Consider a spin $1/2$ particle. Let proposition $P_z$ be "spin $z$ component is $+1/2$."

\begin{itemize}
\item In classical logic, $\neg P_z$ means "spin $z$ component is $-1/2$." Therefore $P_z \lor \neg P_z$ covers all cases.

\item In quantum logic (intuitionistic logic), if we are in $x$-direction eigenstate $|\uparrow_x\rangle = \frac{1}{\sqrt{2}}(|\uparrow_z\rangle + |\downarrow_z\rangle)$, the situation is completely different.
\end{itemize}

\begin{theorem}[Failure of Law of Excluded Middle]
\label{thm:lem-failure}
In physical topos $\mathcal{T}_\mathcal{A}$, for non-trivial quantum systems, there exist propositions $P$ such that global truth value:
$$[\![ P \lor \neg P ]\!] \neq \mathbf{1} \quad (\text{absolute truth})$$
\end{theorem}

\textbf{Proof}:

In context $V_x$ (measuring $S_x$), proposition $P_z$ (statement about $S_z$) is \textbf{undefined}, because operator $S_z$ is not in algebra $V_x$.

Therefore, in $V_x$, we can neither assert $P_z$ is true, nor assert $\neg P_z$ is true.
$$[\![ P_z ]\!] (V_x) = \emptyset, \quad [\![ \neg P_z ]\!] (V_x) = \emptyset$$

So $[\![ P_z \lor \neg P_z ]\!] (V_x) = \emptyset \neq \text{True}$.

This means statement "electron spin is either up or down" is logically invalid in contexts where $z$-direction measurement is not performed.

\textbf{Physical Corollary (Logical Essence of Superposition)}:

Superposition state $|\psi\rangle = \alpha |0\rangle + \beta |1\rangle$ is not "simultaneously in 0 and 1," but \textbf{in a logical state where $P_0 \lor \neg P_0$ has not yet obtained truth value}. Superposition is logical "pending zone."

\subsection{Double Negation Does Not Eliminate: $\neg \neg P \neq P$}

Most famous feature of intuitionistic logic is failure of double negation law. In QCA universe, this has profound dynamical meaning.

\begin{definition}[Physical Meaning of Double Negation]
\label{def:double-negation}
\begin{itemize}
\item $P$: System is \textbf{explicitly} in state $P$ (e.g., measurement result confirms $P$).

\item $\neg P$: System is in state orthogonal to $P$.

\item $\neg \neg P$: System is in state where \textbf{$P$ cannot be excluded}.
\end{itemize}

In Hilbert space, for any non-zero state $|\psi\rangle$, as long as its projection onto subspace of $P$ is non-zero, it satisfies $\neg \neg P$.

For example, superposition state $|\psi\rangle = \epsilon |0\rangle + \sqrt{1-\epsilon^2} |1\rangle$.

\begin{itemize}
\item It is not $|0\rangle$ (so $P_0$ is not fully true).

\item It is not $|1\rangle$ (so $\neg P_0$ is also not fully true).

\item But it can never be classified as "not $P_0$," because it contains component of $P_0$. Therefore $\neg \neg P_0$ is true.
\end{itemize}
\end{definition}

\begin{theorem}[Logical Approximation Theorem]
\label{thm:logical-approximation}
In physical topos, $P \subseteq \neg \neg P$ always holds.

$\neg \neg P$ represents \textbf{Logical Closure} or \textbf{Densification} of $P$.

Physical evolution (unitary dynamics) usually occurs at level of $\neg \neg P$ (possibility space), while measurement (topological fusion) is process of forcibly jumping from $\neg \neg P$ to $P$ (deterministic realization). This is precisely dual description of "topological fusion" in Section 21.4 at logical level.
\end{theorem}

\subsection{Curry-Howard-Lambek Correspondence: Proofs as Programs}

Why should physics follow this strange logic? Because \textbf{Physics is Computation}.

According to Curry-Howard-Lambek (CHL) correspondence:

\begin{itemize}
\item \textbf{Proposition} = \textbf{Type} / \textbf{State Space}

\item \textbf{Proof} = \textbf{Program} / \textbf{Physical Process}
\end{itemize}

In constructive mathematics (mathematical foundation of intuitionistic logic), to prove $A \lor B$, you must provide an algorithm that explicitly outputs $A$ or outputs $B$.

In quantum mechanics, without performing measurement (running "collapse program"), system has no definite $A$ or $B$. Therefore, before measurement occurs, $A \lor B$ is \textbf{unprovable} in constructive sense.

\begin{corollary}[Constructivity of Physical Reality]
\label{cor:constructivity}
QCA discrete ontology is a \textbf{Constructive Theory}. Universe does not contain physical quantities that "exist but are uncomputable."

\begin{itemize}
\item Classical physics claims $x(t)$ has value at any moment (law of excluded middle), even if no one measures. This is \textbf{non-constructive} (God's-eye view).

\item QCA physics claims only information computed by $U$ operators and extracted by observers has truth value. This is \textbf{intuitionistic} (constructive perspective).
\end{itemize}
\end{corollary}

\subsection{Conclusion: Geometrization of Logic}

This section proved that quantum strangeness is not collapse of physics, but rigorization of logic.

\begin{enumerate}
\item \textbf{Superposition} is logical gap where law of excluded middle fails.

\item \textbf{Uncertainty} is manifestation that double negation does not eliminate ($\neg \neg P \neq P$).

\item \textbf{Measurement} is \textbf{logical phase transition} from intuitionistic logic ($\neg \neg P$) to Boolean logic ($P$).
\end{enumerate}

By introducing topos and intuitionistic logic, we found the most fundamental logical foundation for physical theory of the entire book. In the next section 24.3, we will specifically demonstrate implementation of CHL correspondence in physics, proving that \textbf{physical laws are type inference rules in type theory}.

