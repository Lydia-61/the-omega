\section{Implementation of Curry-Howard-Lambek (CHL) Correspondence in Physics}

In Sections 24.1 and 24.2, we established topos model of physical systems and intuitionistic logic foundation. This reveals \textbf{Constructivity} of physical reality: physical truths are not static tautologies, but need to be established through observation (operations).

This section will introduce the most profound isomorphism between logic, computer science, and category theory—\textbf{Curry-Howard-Lambek (CHL) Correspondence}—and generalize it to physics. We will prove that physics, logic, and computation theory are \textbf{trinitarian} at deep structural level. In QCA discrete ontology, \textbf{Physical Systems are Types, Physical States are Programs, Physical Processes are Proofs}. This perspective completely eliminates boundaries between "physical laws" and "mathematical logic," interpreting universe evolution as a huge \textbf{Type Inference and Reduction} process.

\subsection{Meta-Isomorphism of Trinity: Physics, Logic, and Computation}

CHL correspondence reveals deep isomorphism of three seemingly independent fields:

\begin{enumerate}
\item \textbf{Logic}: Science of propositions and proofs.

\item \textbf{Computation}: Science of types and programs ($\lambda$-calculus).

\item \textbf{Category}: Science of objects and morphisms.
\end{enumerate}

After introducing physics, this correspondence extends to \textbf{Physics-Logic-Computation-Category} quaternary isomorphism.

\begin{definition}[CHL Dictionary of Physics]
\label{def:chl-dictionary}
In QCA physical theory, this correspondence concretizes as:

\begin{center}
\begin{tabular}{|l|l|l|l|}
\hline
\textbf{Physics} & \textbf{Logic} & \textbf{Type Theory/Computation} & \textbf{Category} \\
\hline
\textbf{Physical System} $A$ (Hilbert space) & \textbf{Proposition} $P$ & \textbf{Type} $T$ (data type) & \textbf{Object} $\text{Obj}$ \\
\hline
\textbf{Physical State} $|\psi\rangle \in A$ & \textbf{Proof} (Witness) $w : P$ & \textbf{Term} (Term) $t : T$ (instance) & \textbf{Morphism} $I \to A$ \\
\hline
\textbf{Physical Process} $U : A \to B$ & \textbf{Implication} $P \implies Q$ & \textbf{Function} $f : T \to S$ & \textbf{Morphism} $A \to B$ \\
\hline
\textbf{Composite System} $A \otimes B$ & \textbf{Conjunction} $P \land Q$ & \textbf{Product Type} $T \times S$ & \textbf{Tensor Product} $A \otimes B$ \\
\hline
\textbf{Interaction} (Hamiltonian) & \textbf{Inference Rule} & \textbf{Function Application} & \textbf{Composition} $\circ$ \\
\hline
\end{tabular}
\end{center}

\textbf{Physical Interpretation}:

When we say "system A is in state $|\psi\rangle$," it is logically equivalent to saying "proposition A has a proof $|\psi\rangle$."

\begin{itemize}
\item \textbf{Vacuum} is tautology (Truth object $\Omega$), always true.

\item \textbf{State Preparation} is constructing a proof.

\item \textbf{Measurement} is verifying whether a proof conforms to specific type (eigenspace).
\end{itemize}
\end{definition}

\subsection{Linear Logic and Quantum Resources: Logical Root of No-Cloning}

Classical logic and standard $\lambda$-calculus allow free copying ($A \implies A \land A$) and discarding ($A \implies \text{True}$) of information. But in quantum physics, no-cloning theorem and unitarity prohibit such operations.

Therefore, logic corresponding to quantum physics is not classical logic, but \textbf{Linear Logic} (proposed by Jean-Yves Girard).

\begin{theorem}[Linear Logic Formulation of Quantum Processes]
\label{thm:linear-logic}
Physical laws in QCA universe follow syntactic rules of \textbf{Linear Logic}:

\begin{enumerate}
\item \textbf{Resource Sensitivity}: Propositions (resources) cannot be arbitrarily copied or destroyed. Premise $A$ is "consumed" after inference $A \vdash B$, transformed into $B$. This precisely corresponds to extreme difficulty of \textbf{non-destructive measurement} of quantum states, and unitarity of state evolution.

\item \textbf{Multiplicative Connectives}:

\begin{itemize}
\item \textbf{Tensor Product ($\otimes$)}: $A \otimes B$ means "simultaneously having resource A \textbf{and} resource B."

\item \textbf{Linear Implication ($\multimap$)}: $A \multimap B$ means "process consuming A to produce B." Physically, this is operator space $\mathcal{L}(\mathcal{H}_A, \mathcal{H}_B) \cong \mathcal{H}_A^* \otimes \mathcal{H}_B$.
\end{itemize}

\item \textbf{Duality}: Negation $A^\perp$ in linear logic corresponds to dual space $\mathcal{H}^*$ or antiparticles. $A^{\perp\perp} \cong A$ corresponds to dagger structure in DCC.
\end{enumerate}
\end{theorem}

\textbf{Corollary}:

Quantum mechanics appears "strange" because it runs on a \textbf{resource-conserving logical system} at bottom level. Wave function collapse is not logical error, but \textbf{resource consumption of linear types}—you read data once, data is "used up" (becomes classical record, no longer original quantum state).

\subsection{Physical Laws as Type Inference Rules}

From CHL perspective, physical laws (such as Schrödinger equation or QCA update rules) are no longer descriptive formulas, but \textbf{constructive Type Inference Rules}.

\begin{definition}[Proof-Theoretic Semantics of Dynamics]
\label{def:proof-semantics}
Consider QCA's local update rule $U : \mathcal{H}_{\text{in}} \to \mathcal{H}_{\text{out}}$.

In type theory, this corresponds to a \textbf{function term}:
$$\text{update} : \text{State}_{t} \multimap \text{State}_{t+1}$$

Time evolution sequence of universe $x_0 \to x_1 \to x_2 \to \cdots$ corresponds to step-by-step construction of a \textbf{Proof Tree}.

\begin{itemize}
\item \textbf{$t=0$}: Axiom (initial conditions).

\item \textbf{$t=n$}: Theorem obtained by applying derivation rules (physical laws) $n$ times.
\end{itemize}
\end{definition}

\begin{theorem}[Physics as Normalization]
\label{thm:physics-normalization}
Running of physical processes is equivalent to \textbf{Reduction (Normalization)} of $\lambda$-terms.

Let initial physical configuration be a complex tensor network (or $\lambda$-term) $M$.

\begin{itemize}
\item \textbf{Interactions} (such as particle collisions) correspond to $\beta$-reduction: $(\lambda x.\, t)\, u \to t[u/x]$.

\item \textbf{Thermal Equilibrium} corresponds to \textbf{Normal Form}: a stable state that cannot be further simplified.
\end{itemize}

Therefore, \textbf{time passage is process of universe computing system executing reduction steps}.
\end{theorem}

\subsection{Universe as Type-Theoretic Universe: From "Existence" to "Construction"}

Finally, this perspective resolves a core ontological divergence in physics: \textbf{Platonism vs. Constructivism}.

\begin{itemize}
\item \textbf{Platonism}: Physical laws exist in an eternal world of ideas, physical world imitates it.

\item \textbf{QCA Constructivism}: Physical reality is \textbf{constructed by computational processes}.
\end{itemize}

\begin{corollary}[Constructive Realism]
\label{cor:constructive-realism}
In QCA universe, if a physical state $|\psi\rangle$ cannot be prepared from vacuum or given initial state through finite QCA steps (finite-length proofs), then it is \textbf{non-existent} physically.

This excludes "mathematical states" in standard Hilbert space with uncomputable amplitudes. Physical Hilbert space is subspace of \textbf{Computable States}.

This echoes Gödel incompleteness discussed in Section 5.4: physical truths are limited by \textbf{Provability}. Universe can only explore states it can "compute to."
\end{corollary}

\textbf{Conclusion}

Section 24.3 establishes \textbf{computational-logical foundation} of physics.

\begin{enumerate}
\item \textbf{Isomorphism}: Physical systems, logical propositions, and computational types are three faces of the same structure.

\item \textbf{Logic}: Quantum physics follows linear logic, emphasizing non-clonability of resources.

\item \textbf{Dynamics}: Evolution is proof, equilibrium is normal form.
\end{enumerate}

At this point, we have completed logical reconstruction of meta-theory of physics. We proved that QCA theory is not only physically self-consistent (unitarity, causality), but also has the most solid mathematical foundation logically (category theory, type theory).

In the upcoming \textbf{Part XIV: Computational Foundations and Encoding}, we will ground these abstract logical principles, exploring how physical world performs \textbf{optimal encoding} (such as bit counting of holographic principle) and \textbf{thermodynamic cost} of computation (Landauer principle).

