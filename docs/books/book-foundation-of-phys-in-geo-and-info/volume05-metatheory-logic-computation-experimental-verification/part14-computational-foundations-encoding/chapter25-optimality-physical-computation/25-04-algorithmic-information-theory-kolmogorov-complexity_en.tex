\section{Algorithmic Information Theory and Kolmogorov Complexity of Physical Laws}

In previous sections of Chapter 25, we explored thermodynamic cost of physical computation (Landauer principle) and optimal encoding under local causal constraints (Fibonacci/golden ratio). These discussions mainly focus on encoding and transmission of \textbf{States}. However, physics has a deeper question: encoding of \textbf{Laws} themselves.

Why are fundamental physical laws describing universe (such as Standard Model Lagrangian or Einstein equations) so concise that they can be written on a T-shirt? Why don't we live in a universe with extremely complex laws full of special cases and patches?

This section will introduce \textbf{Algorithmic Information Theory (AIT)}, using \textbf{Kolmogorov Complexity} to quantify simplicity of physical laws. We will prove that Occam's Razor is not merely human aesthetic preference, but \textbf{statistical necessity for existence of computational universes}. In QCA discrete ontology, universe is a computational process with extremely high \textbf{Logical Depth} generated by extremely short programs (low Kolmogorov complexity).

\subsection{Algorithmic Entropy of Physical Theories: From Equations to Programs}

In traditional physics, laws are differential equations. In QCA discrete ontology, laws are \textbf{update rules of cellular automata}.

\begin{definition}[Kolmogorov Complexity of Physical Laws]
\label{def:kolmogorov}
Let $\mathfrak{U}$ be a physical universe model. Its Kolmogorov complexity $K(\mathfrak{U})$ is defined as length (in bits) of \textbf{shortest program} capable of simulating evolution of this universe:
$$K(\mathfrak{U}) = \min_{p} \{ |p| : U_{TM}(p) = \text{History}(\mathfrak{U}) \}$$

where $U_{TM}$ is universal Turing machine (or universal QCA).

According to parametric definition in Chapter 20, this shortest program is essentially optimal compressed encoding of universe's \textbf{parameter vector} $\Theta = (\Theta_{\text{str}}, \Theta_{\text{dyn}}, \Theta_{\text{ini}})$.
$$K(\mathfrak{U}) \approx I_{\text{param}}(\Theta)$$

Goal of physics is precisely to find $\Theta$ with minimum $K(\mathfrak{U})$.
\end{definition}

\subsection{Physical Origin of Occam's Razor: Algorithmic Probability}

Why do physical laws tend to be simple (low $K$)? Solomonoff's \textbf{Algorithmic Probability Theory} provides the answer.

\begin{theorem}[Prior Probability of Universe]
\label{thm:prior-probability}
Assume all possible computational universes are randomly sampled from "space of all possible programs." According to algorithmic information theory, prior probability $P(\mathfrak{U})$ of a specific universe $\mathfrak{U}$ being "generated" or "existing" has exponential decay relationship with its Kolmogorov complexity:
$$P(\mathfrak{U}) \approx 2^{-K(\mathfrak{U})}$$

This means:

\begin{itemize}
\item \textbf{Simple Universes (Low $K$)}: Such as QCA with translational symmetry, local interactions, their $K(\mathfrak{U})$ is very small (only need few lines of code to define rules), therefore existence probability is extremely high.

\item \textbf{Complex Universes (High $K$)}: Such as universes full of arbitrary non-local connections, laws changing every second, their $K(\mathfrak{U})$ is extremely large, existence probability tends to zero.
\end{itemize}
\end{theorem}

\textbf{Physical Corollary}:

Reason why physical laws we observe have \textbf{symmetries} (spatial translation, time translation, gauge symmetry) is because \textbf{symmetry is best means of compressing information}.

\begin{itemize}
\item Spatial translational symmetry means we don't need to define physical laws separately for each point in universe, only need to define once, then say "same everywhere." This greatly reduces $K(\mathfrak{U})$.

\item \textbf{Occam's Razor is maximum likelihood estimation of universe generation}.
\end{itemize}

\subsection{Emergence of Complexity: Logical Depth vs. Randomness}

If universe tends toward simplicity, why is macroscopic world we see (life, galaxies) so complex? Here we need to distinguish \textbf{Algorithmic Complexity (Randomness)} from \textbf{Logical Depth (Organization)}.

\begin{enumerate}
\item \textbf{Random Sequences} (such as coin toss results): $K(s) \approx |s|$. Incompressible, extremely complex, but no structure.

\item \textbf{Simple Sequences} (such as all 1s): $K(s) \approx \text{const}$. Extremely compressible, no structure.

\item \textbf{Structured Sequences} (such as DNA or QCA evolution patterns): $K(s)$ is small (originating from simple evolutionary laws), but generating it requires long computational process.
\end{enumerate}

\begin{definition}[Bennett's Logical Depth]
\label{def:logical-depth}
Logical depth $D(x)$ of an object (or universe state) is defined as \textbf{computation time (logical steps)} required to run shortest program generating $x$.
$$D(x) = \text{Time}(p^*) \quad \text{s.t. } U(p^*) = x \land |p^*| = K(x)$$
\end{definition}

\begin{theorem}[Universe Depth Theorem]
\label{thm:universe-depth}
Our QCA universe is a system with \textbf{low Kolmogorov complexity, high logical depth}.

\begin{itemize}
\item \textbf{Simple Laws}: $K(U) \ll \text{Size of Universe}$. $\Theta$ is very short.

\item \textbf{Long History}: To obtain current state $\rho(t_{now})$ from $\Theta$, must undergo $10^{60}$ Planck time steps of irreducible computation (computational irreducibility, see Section 5.4).
\end{itemize}

\textbf{Conclusion}: "Beauty" of physics lies in extremely simple rules (low $K$) emerging extremely complex phenomena (high depth). If laws themselves are complex, that's ugly; if phenomena are simple, that's boring.
\end{theorem}

\subsection{Computability of Physical Constants: Taboo of Chaitin Constant $\Omega$}

In standard model, physical constants (such as fine structure constant $\alpha$) are considered real numbers. But in AIT, vast majority of real numbers are \textbf{uncomputable} (random), with $K(\alpha) = \infty$.

If physical constants are uncomputable real numbers, then $K(\mathfrak{U})$ of universe would be infinite, its existence probability zero.

\begin{corollary}[Computability Conjecture of Constants]
\label{cor:computability}
In QCA discrete ontology, all physical constants (including $\alpha, G, \dots$) must be \textbf{Computable Numbers}.

This means they are either rational numbers, or limits of some simple algorithms (such as geometric series, algebraic functions of $\pi$).

For example, in Section 25.3 we saw that dimension of topological quantum computation is $\phi$ (algebraic number). Entropy coefficient of black holes in Chapter 15 is $1/4$ (rational number).

\textbf{Chaitin Constant $\Omega$} (uncomputable number representing halting probability) cannot appear as physical constant in Lagrangian. Physics rejects uncomputability.
\end{corollary}

\subsection{Summary of Part XIV}

Part XIV reveals information-theoretic essence of physical laws by mapping physics to computation and coding theory.

\begin{enumerate}
\item \textbf{Cost}: Landauer principle specifies thermodynamic bottom line of computation (25.1).

\item \textbf{Encoding}: Fibonacci coding demonstrates optimal counting method in local causal networks (25.2).

\item \textbf{Efficiency}: Golden ratio $\phi$ is eigenvalue of most robust information channel (25.3).

\item \textbf{Simplicity}: Kolmogorov complexity explains why physical laws tend toward simple symmetric structures, rejecting random parameters (25.4).
\end{enumerate}

This proves: \textbf{Universe is not only a computer, but an efficient computer that has undergone "code optimization"}. It uses shortest code (laws), computes richest reality (high logical depth) under minimum energy consumption (Landauer lower bound).

In the final part of the entire book—\textbf{Part XV: Experimental Verification and Engineering Prospects}—we will leave ivory tower of theory, discussing how to use precision measurement experiments (such as microwave cavities, gravitational waves) to verify these grand information geometric predictions.

