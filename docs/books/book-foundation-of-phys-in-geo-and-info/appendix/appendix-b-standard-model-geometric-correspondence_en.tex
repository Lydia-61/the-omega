\chapter{Geometric Correspondence Table of Standard Model Particles}

In Chapters 16 (Total Space Geometry) and 17 (Topological Origin of Matter), this book establishes a theoretical framework that unifies gravity, gauge fields, and fermions under QCA discrete ontology. In this framework, elementary particles are not point-like entities, but specific geometric/topological structures on spacetime and internal fiber bundles.

This appendix aims to provide a detailed correspondence table establishing one-to-one mappings between core members of the Standard Model of particle physics and geometric objects constructed in this book. This is not only a summary of the preceding theory, but also an index for future searches for topological fingerprints predicted by QCA in high-energy physics experiments.

\section{Matter Fields (Fermions): Topological Knots and Self-Referential Structures}

In QCA theory, fermions are \textbf{topological solitons} or \textbf{knots} in causal networks, carrying non-trivial topological charges protected by $\mathbb{Z}_2$ holonomy indices.

\begin{table}[h]
\centering
\small
\begin{tabular}{|p{3cm}|p{4cm}|p{4cm}|p{4cm}|}
\hline
\textbf{Standard Model Particle} & \textbf{QCA / Total Space Geometric Correspondence} & \textbf{Topological/Geometric Feature Description} & \textbf{Physical Property Origin} \\
\hline
\textbf{Left-Handed Electron} ($e_L$) & \textbf{Fundamental $\mathbb{Z}_2$ Knot} & Minimal non-trivial self-referential loop on ``shallow'' geometry of spacetime. Its wave function spans two sheets on Null-Modular double cover. & \textbf{Charge}: Momentum map on internal fiber.\\
& & & \textbf{Spin 1/2}: $4\pi$ rotational symmetry of double cover space. \\
\hline
\textbf{Right-Handed Electron} ($e_R$) & \textbf{Dual $\mathbb{Z}_2$ Knot} & Topologically conjugate to left-handed knot, but with opposite orientation in internal space (chirality flip). & \textbf{Mass}: Coupling strength between left/right chiral components through vacuum Higgs condensation (topological background field) (Zitterbewegung frequency). \\
\hline
\textbf{Neutrino} ($\nu_L$) & \textbf{Neutral Topological Tunneling State} & Knot lacking internal charge momentum. Left-right chiral coupling requires traversing high-dimensional topological potential barrier (see Section 17.4). & \textbf{Tiny Mass}: Exponential suppression effect from topological tunneling (geometric seesaw mechanism). \\
\hline
\textbf{Quarks} ($u, d, \dots$) & \textbf{Color-Entangled Triplet} & Open string/knot endpoints that must exist in groups of three (or particle-antiparticle pairs) to close in total space. Single particle is undefined in total space (topologically constrained). & \textbf{Color Charge}: Non-Abelian holonomy generators on internal $SU(3)$ fiber.\\
& & & \textbf{Confinement}: Long-range tension caused by open topological structure. \\
\hline
\textbf{Generations} ($\mu, \tau$) & \textbf{Higher Topological Excitations} & Higher harmonics of fundamental knots on internal manifold or more complex knotting (e.g., trefoil vs. figure-8 knot). & \textbf{Mass Hierarchy}: Higher topological complexity requires higher ``processing frequency'' (energy) to maintain the structure, manifesting as larger rest mass. \\
\hline
\end{tabular}
\caption{Geometric correspondence of fermions in QCA theory}
\end{table}

\section{Interaction Fields (Bosons): Connections and Curvature}

In QCA theory, bosons are mediators of interactions, corresponding to perturbations of the \textbf{unified connection $\mathbb{A}$} or wave packets of \textbf{curvature $\mathbb{F}$} on the total space principal bundle.

\begin{table}[h]
\centering
\small
\begin{tabular}{|p{3cm}|p{4cm}|p{4cm}|p{4cm}|}
\hline
\textbf{Standard Model Particle} & \textbf{QCA / Total Space Geometric Correspondence} & \textbf{Topological/Geometric Feature Description} & \textbf{Physical Property Origin} \\
\hline
\textbf{Photon} ($\gamma$) & \textbf{$U(1)$ Holonomy Packet} & Curvature wave of unified connection $\mathbb{A}$ in internal circle fiber ($S^1$) direction. Maintains topological triviality ($\nu=0$). & \textbf{Massless}: Corresponds to long-range geometric correlation protected by gauge symmetry.\\
& & & \textbf{Spin 1}: Vector property as 1-form field. \\
\hline
\textbf{W/Z Bosons} & \textbf{Massive Gauge Connection} & Connection perturbations accompanied by local ``hardening'' of vacuum geometry (Higgs field). Propagation is impeded by background topological condensation (geometric dual of Meissner effect). & \textbf{Mass}: Arises from short-range geometric rigidity caused by Higgs mechanism.\\
& & & \textbf{Weak Force}: Chiral twisted propagation on internal fiber. \\
\hline
\textbf{Gluons} ($g$) & \textbf{Non-Abelian Curvature Flux} & Self-interacting curvature on $SU(3)$ fiber in total space. They not only transmit curvature but also carry curvature sources (color charge). & \textbf{Asymptotic Freedom}: Curvature localization at high energy; flux tubes from nonlinear superposition of curvature at low energy. \\
\hline
\textbf{Graviton} ($G$) & \textbf{Spacetime Metric Wave} & Quadrupole fluctuation of unified connection $\mathbb{A}$ in spacetime tangent space direction (spin connection $\omega$). & \textbf{Spin 2}: Arises from geometric property of metric tensor (or bilinear of frame fields).\\
& & & \textbf{Universality}: All forms of energy cause spacetime curvature. \\
\hline
\end{tabular}
\caption{Geometric correspondence of bosons in QCA theory}
\end{table}

\section{Higgs Sector and Vacuum Structure}

The Higgs field gives particles mass in the Standard Model. In QCA theory, it corresponds to the \textbf{order parameter of vacuum geometry}.

\begin{table}[h]
\centering
\small
\begin{tabular}{|p{3cm}|p{4cm}|p{4cm}|p{4cm}|}
\hline
\textbf{Standard Model Particle} & \textbf{QCA / Total Space Geometric Correspondence} & \textbf{Topological/Geometric Feature Description} & \textbf{Physical Property Origin} \\
\hline
\textbf{Higgs Boson} ($H$) & \textbf{Amplitude Mode of Geometric Condensate} & Radial fluctuation of vacuum QCA network connectivity or internal fiber ``stiffness''. & \textbf{Mass Generation}: Higgs vacuum expectation value (VEV) establishes background geometric constant for left/right knot coupling (similar to energy gap in superconductors). \\
\hline
\textbf{Vacuum} ($|0\rangle$) & \textbf{Quantum Liquid Crystal} & Ground state of QCA network filled with short-range entangled loops (trivial topology). Has non-zero absolute density of states $\rho_{vac}$. & \textbf{Dark Energy}: Exponential expansion caused by time evolution phase of vacuum density of states ($\kappa = \rho$) (see Section 9.4). \\
\hline
\textbf{Axion} ($a$) & \textbf{Topological Phase Wave} & Dynamical fluctuation of overall QCA network topological parameter $\theta$ (strong CP angle). & \textbf{CP Restoration}: Relaxation of axion field eliminates overall chiral twist of network (see Section 17.3). \\
\hline
\end{tabular}
\caption{Geometric correspondence of Higgs sector in QCA theory}
\end{table}

\section{Summary: Geometric Unification Picture}

Through the above table, we can clearly see the unified picture constructed by \textit{Foundations of Physics in Geometry and Information}:

\begin{enumerate}
\item \textbf{Particles are knots}: Fermions are topological knots on spacetime fabric.

\item \textbf{Forces are curvature}: Bosons are transmission of fabric shape.

\item \textbf{Mass is resistance}: Coupling strength between particles and vacuum geometric background (Higgs condensation).

\item \textbf{Vacuum is sea}: A dynamic medium filled with information processing activity (density of states), whose surface tension manifests as dark energy, and whose vortices manifest as matter.
\end{enumerate}

This correspondence table not only reproduces the classification of the Standard Model, but also provides unified geometric explanations for dark matter (axions), dark energy (vacuum density of states), and neutrino mass (topological tunneling).

