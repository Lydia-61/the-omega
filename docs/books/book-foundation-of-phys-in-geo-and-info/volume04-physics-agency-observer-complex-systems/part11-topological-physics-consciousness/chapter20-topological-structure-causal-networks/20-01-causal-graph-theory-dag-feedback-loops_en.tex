\section{Fundamentals of Causal Graph Theory: Directed Acyclic Graphs (DAG) and Feedback Loops}

In QCA discrete ontology, universe history is described by a huge spacetime network graph $G = (V, E)$, where $V$ are events (lattice updates) and $E$ are causal connections. The core task of physics, in graph-theoretic language, is to study the flow topology of information on this network.

\subsection{Topology of Mechanical Universe: DAG and Causal Order}

In classical, non-intelligent physical processes, causality strictly follows the arrow of time. Past determines future, future never affects past. This structure is mathematically characterized as \textbf{Directed Acyclic Graph (DAG)}.

\begin{definition}[Causal DAG]
\label{def:causal-dag}
Let QCA universe's historical network be $G$. If for any vertex sequence $v_1, v_2, \dots, v_k$, if $(v_i, v_{i+1}) \in E$, then necessarily $v_k \neq v_1$.

This means there are no \textbf{Closed Timelike Curves (CTCs)} in the network.

\begin{itemize}
\item \textbf{Partial Order Relation}: DAG structure naturally induces a partial order relation $\preceq$. If there exists a path from $a$ to $b$, then $a \preceq b$.

\item \textbf{Mechanical Nature}: In DAG, current state of any node is completely determined by parent nodes in its \textbf{Past Light Cone}. This purely feedforward structure corresponds to "zombie" or mechanical automaton behavior patterns. Even if the system is extremely complex (such as weather systems), as long as it can be unfolded as a DAG, it is unconscious.
\end{itemize}
\end{definition}

\subsection{Necessary Condition for Consciousness: Feedback Loops}

Research in cybernetics and neuroscience shows that consciousness is inseparable from \textbf{Re-entry} or \textbf{Recurrence}. To realize this recursion at the physical level, network topology must break DAG limitations in some effective sense, forming \textbf{Feedback Loops}.

\begin{definition}[Information Flow Loop]
\label{def:info-flow-loop}
In physical time $t$, the universe as a whole is a DAG (causal laws do not allow returning to the past). But in \textbf{Functional Connectivity} or \textbf{State Space}, subsystems can form closed loops.

Consider a subnetwork $S \subset G$. If there exists a causal path sequence such that system information flow, after a series of transformations, acts back on itself:
$$\text{State}(t) \to \text{Processing} \to \text{State}(t+\Delta t) \approx \text{State}(t)$$

This manifests as a \textbf{Helix} in spacetime graph, but on the system's \textbf{phase space manifold}, it manifests as a topological circle $S^1$.
\end{definition}

\textbf{Physical Distinction}:

\begin{itemize}
\item \textbf{Simple Feedback (Thermostat)}: Such loops are usually \textbf{dissipative}, information rapidly decays in the loop, system converges to fixed point. Topologically contractible.

\item \textbf{Consciousness Loop (Strange Loop)}: Such loops are \textbf{self-sustaining} and \textbf{information-gaining}. They correspond to what Hofstadter called "Strange Loops." In QCA language, these are closed orbits carrying non-trivial $\mathbb{Z}_2$ holonomy index in \textbf{Self-referential Scattering Networks (SSN)}.
\end{itemize}

\subsection{Graph-Theoretic Decomposition: Components and Hierarchies}

To quantify degree of consciousness, we need to analyze complexity of closed loops in networks. This can be achieved through \textbf{Strongly Connected Component (SCC)} decomposition of graphs.

\begin{definition}[Strongly Connected Component]
\label{def:scc}
In directed graph $G$, a subgraph $C \subseteq G$ is called strongly connected if for any two nodes $u, v$ in $C$, there exist paths from $u$ to $v$ and from $v$ to $u$.

\begin{itemize}
\item \textbf{Condensation}: Contract each SCC in the graph into a super-node. The resulting "graph of graphs" is necessarily a DAG.

\item \textbf{Hierarchical Structure}: This DAG defines causal hierarchy of the system. Lower-level SCCs (such as sensory input) feed information to higher-level SCCs (such as associative cortex).
\end{itemize}
\end{definition}

\begin{theorem}[Consciousness Core Theorem]
\label{thm:consciousness-core}
A necessary condition for a physical system to possess "self" or "unified experience" is: its causal network contains a \textbf{Giant, Irreducible Strongly Connected Component (Giant SCC)}, and this component plays a dominant role in system dynamics (i.e., it is the convergence center of information flow, or the kernel of control manifold).

In Integrated Information Theory (IIT), this Giant SCC is called the \textbf{Complex}, where integrated information $\Phi$ reaches local maximum in this region.
\end{theorem}

\subsection{Physical Realization of Closed Loops: Delay and Memory}

In discrete-time systems like QCA, how are closed loops physically realized? \textbf{Time Delay} must be introduced.

An instantaneous closed loop $x_t = f(x_t)$ is physically ill-posed (or leads to singularities). Physical closed loops must contain memory registers $\mathcal{M}$:
$$x_{t+1} = f(x_t, \mathcal{M}_t); \quad \mathcal{M}_{t+1} = g(x_t, \mathcal{M}_t)$$

Here $\mathcal{M}$ acts as a bridge connecting "present me" with "past me."

Therefore, \textbf{topological structure of consciousness is a hybrid of time and space}. It is not a loop in space, but a \textbf{spiral} on spacetime cylinder. Its "closure" manifests as return of information content (patterns), not return of physical particles.

\textbf{Summary}

This section established graph-theoretic foundations for consciousness.

\begin{enumerate}
\item \textbf{Unconsciousness} corresponds to DAG structure, information flows unidirectionally, no introspective ability.

\item \textbf{Consciousness} corresponds to SCC structure (feedback loops), information reverberates in networks, producing "thickness" of the present.

\item \textbf{Physical Foundation} is delayed circuits in self-referential scattering networks.
\end{enumerate}

In the next section 20.2, we will delve into microscopic structure within these SCCs, define \textbf{Minimal Strongly Connected Components (MSCC)}, and argue they are irreducible "atomic selves."

