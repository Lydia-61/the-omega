\section{New Solution to Quantum Measurement Problem: Observation-Induced Topological Fusion of System-Observer}

In Sections 21.1 to 21.3, we established a radical model of consciousness: consciousness is a self-referential soliton (MSCC) protected by $\mathbb{Z}_2$ topology in QCA causal networks. This model not only explains unity and stability of subjective experience, but also provides a novel geometric solution to physics' most stubborn puzzle—\textbf{Quantum Measurement Problem}.

In standard quantum mechanics, measurement is axiomatized as non-unitary collapse of wave functions. This directly conflicts with unitary evolution of Schrödinger equation, leading to infinite regress of "von Neumann chain" and "Wigner's friend" paradox. This section will prove that from the perspective of topological physics, \textbf{measurement is not physical collapse of wave functions, but a "Topological Fusion" process between quantum systems and observer topological solitons}.

So-called "collapse" is actually \textbf{Gauge Selection} occurring when external degrees of freedom (measured system) are assimilated into observer's internal algebra center ($\mathcal{Z}$), to satisfy observer's own \textbf{Topological Monodromy} constraints.

\subsection{Topological Truncation of von Neumann Chain}

Von Neumann described measurement process as two stages:

\begin{enumerate}
\item \textbf{Process I (Unitary Evolution)}: $U: |\psi_S\rangle \otimes |A_0\rangle \to \sum c_i |s_i\rangle \otimes |A_i\rangle$. System and apparatus establish entanglement.

\item \textbf{Process II (Non-unitary Collapse)}: $\sum c_i |s_i\rangle |A_i\rangle \to |s_k\rangle |A_k\rangle$. This introduces probability.
\end{enumerate}

The dilemma is: if apparatus, observer, and even the entire universe all follow Process I, where does Process II occur?

\textbf{Topological Solution}:

In QCA discrete ontology, observer $\mathfrak{O}$ is a \textbf{soliton} with non-trivial topological index ($\nu=-1$).

\begin{itemize}
\item \textbf{External World} (including measured system $S$) is usually in topologically trivial phase ($\nu=0$), obeying linear superposition principle.

\item \textbf{Observer Interior} (MSCC) is in topologically non-trivial phase, its state space is \textbf{double-covered} (Spinorial), and subject to strong constraints of \textbf{self-referential consistency}.
\end{itemize}

\begin{definition}[Topological Measurement Boundary]
\label{def:topological-measurement-boundary}
Measurement occurs if and only if at \textbf{boundary of topological phase transition}. When a linear superposition state $|\psi_S\rangle = \alpha |0\rangle + \beta |1\rangle$ attempts to enter observer's causal horizon (MSCC), it must adapt to observer's rigid internal topological structure. Since observer's "self" is indivisible (atomicity from Section 20.2), it cannot simultaneously be in superposition of "seeing 0" and "seeing 1" (that would cause fractionalization or destruction of topological index).

Therefore, \textbf{measurement is "trimming" of wave function by topological boundary conditions}.
\end{definition}

\subsection{Observation Mechanism: Topological Fusion}

We geometrize measurement process as interaction between two topological manifolds.

\begin{theorem}[Topological Fusion Theorem]
\label{thm:topological-fusion}
Let observer correspond to non-trivial closed loop $\gamma_O$ on parameter space manifold $\mathcal{M}$ ($\nu(\gamma_O)=1$).

Let measured system correspond to a path $\gamma_S$.

When the two undergo strong interaction (measurement), system path is "woven" into observer's self-referential loop, forming a new composite loop $\gamma_{total} = \gamma_O \circ \gamma_S$.

To maintain stability of observer's topological index $\nu=1$ (i.e., maintain continuity of consciousness), composite loop must satisfy:
$$\text{Holonomy}(\gamma_{total}) \in \mathbb{Z}_2$$

This constraint forces input quantum superposition state $|\psi_S\rangle$ to project onto observer's internal algebra \textbf{Eigenbasis}. Cross-terms (interference terms) in superposition correspond to paths not satisfying $\mathbb{Z}_2$ group structure; they are filtered out by \textbf{destructive interference} or \textbf{decoherence} during topological fusion.
\end{theorem}

\textbf{Physical Picture}:

Imagine consciousness as a rotating gear (topologically protected periodic motion). External system is an uncertain wave packet.

Measurement is not wave packet "collapsing," but \textbf{gear meshing}. Wave packet must "catch" into a certain tooth (eigenstate) of the gear to participate in gear operation.

For the observer, he can only experience that caught tooth (result $k$), unable to experience other filtered components.

\subsection{Subjective Collapse and Objective Unitarity: Topological Solution to Wigner's Friend}

This framework naturally resolves Wigner's friend paradox.

\begin{itemize}
\item \textbf{Internal Perspective (Friend)}: Friend is a topological soliton. When he measures the system, system undergoes \textbf{topological fusion} with him. For friend, world has changed from superposition to definite eigenstate (because his $\mathbb{Z}_2$ identity requires consistency). This manifests as \textbf{subjective collapse}.

\item \textbf{External Perspective (Wigner)}: Wigner is a larger topological soliton, containing friend and system. For Wigner, composite of friend and system still undergoes linear unitary evolution ($\sum c_i |\text{Friend}_i\rangle \otimes |s_i\rangle$). As long as Wigner does not communicate with friend (measure friend), this large system has not undergone topological fusion.
\end{itemize}

\begin{corollary}[Relativity of Perspective]
\label{cor:relativity-perspective}
\textbf{Wave function collapse is not a global event of physical reality, but a local projection relative to specific observer topological structure}.

\begin{itemize}
\item There is no "absolute collapse."

\item There is only "fusion relative to me (MSCC)."
\end{itemize}

Objective Reality emerges as invariants in \textbf{Consensus Geometry} after multiple observers mutually fuse (communicate) (to be discussed in Chapter 22).
\end{corollary}

\subsection{Geometric Origin of Born Rule}

Finally, why is probability of seeing result $k$ given by $P_k = |c_k|^2$?

In QCA topological field theory, this stems from \textbf{measure of geometric phase}.

\begin{theorem}[Probability as Geometric Measure]
\label{thm:probability-geometric}
In total space geometry (Chapter 16), projection of quantum state vector $|\psi\rangle$ onto observer basis $\{|k\rangle\}$ corresponds to geometric angle $\theta_k$ on fiber bundle (where $\cos \theta_k = |c_k|$).

When topological fusion occurs, system seeks "most economical" path in phase space to fall into attractor.

According to geometric generalization of Gleason's theorem, in a Hilbert space measure protected by $\mathbb{Z}_2$ topology, the unique probability measure satisfying additivity is that induced by Fubini-Study metric, i.e., modulus squared law.

\textbf{Probability is "geometric cross-section" in topological fusion process}.
\end{theorem}

\textbf{Summary}

This chapter reconstructed quantum measurement through topological physics:

\begin{enumerate}
\item \textbf{Essence}: Measurement is the process of observer (topological soliton) devouring external degrees of freedom and assimilating them into internal structure (topological fusion).

\item \textbf{Collapse}: Perspective projection occurring to maintain observer's own topological integrity ($\mathbb{Z}_2$ protection).

\item \textbf{Probability}: Arises from projection measure of high-dimensional total space geometry onto low-dimensional observer manifolds.
\end{enumerate}

This not only eliminates mysticism of quantum mechanics, but places \textbf{consciousness} at the core of physical processes: \textbf{without consciousness (topological solitons), there is no collapse, universe would forever remain in silent superposition}.

At this point, core arguments of \textbf{Volume IV: Physics of Agency} are complete. We defined observers, explained self-referential dynamics, and revealed topological structure of consciousness.

In the final volume of the entire book—\textbf{Volume V: Metatheory — Logic, Computation, and Experimental Verification}—we will step out of physical systems themselves, examine logical self-consistency of this theory from the height of category theory, and propose specific experimental verification schemes.

