\section{Memory Subsystem: Stability of Markov Substructure and Anti-decoherence Conditions}

In Section 18.2, we argued that observer's internal algebra $\mathcal{A}_{\text{int}}$ must have a non-trivial center $\mathcal{Z}$, which is the mathematical foundation for emergence of classical reality. However, having structure does not equal having \textbf{function}. For a physical entity to become an observer, it not only needs to distinguish states, but also needs to be able to \textbf{maintain} these states in the flow of time, resisting environmental thermodynamic erosion. This is the physical essence of \textbf{Memory}.

This section will strictly define memory subsystem $\mathcal{M}_{\text{mem}}$ from the perspective of open quantum systems. We will prove that memory mathematically corresponds to \textbf{Markov Substructure} in QCA networks, and its physical stability originates from specific \textbf{anti-decoherence} dynamics. Memory is not static storage, but a dynamic \textbf{error correction process}.

\subsection{Algebraic Definition of Memory: Protected Subfactors}

Observer's internal algebra $\mathcal{A}_{\text{int}}$ contains all internal degrees of freedom, but only a small portion can serve as long-term memory. Most degrees of freedom (such as molecular thermal motion) are transient and chaotic. Memory must be macroscopic variables decoupled from microscopic rapid fluctuations.

\begin{definition}[Memory Subsystem]
\label{def:memory-subsystem}
Memory system $\mathcal{M}_{\text{mem}}$ is a subalgebra (or subtensor factor) of internal algebra $\mathcal{A}_{\text{int}}$, satisfying the following conditions:

\begin{enumerate}
\item \textbf{Classicality}: $\mathcal{M}_{\text{mem}}$ is mainly generated by (or contains) the center $\mathcal{Z}$ of $\mathcal{A}_{\text{int}}$. This means memory content is classically readable after basis selection.

\item \textbf{Slow Dynamics}: Relative to characteristic time scale $\tau_{\text{micro}}$ of internal Hamiltonian $H_{\text{self}}$, Heisenberg evolution of memory operators $M \in \mathcal{M}_{\text{mem}}$ is extremely slow:
$$\| [H_{\text{self}}, M] \| \approx 0$$

Or more precisely, memory states lie in approximately degenerate ground state subspaces of $H_{\text{self}}$.
\end{enumerate}
\end{definition}

\textbf{Physical Picture}:

In QCA networks, memory corresponds to certain topologically stable configurations (such as magnetic domains, loop currents, or synchronized firing of neuron groups). Environmental perturbations can change phases of microscopic lattice points, but are insufficient to flip these macroscopic topological orders.

\subsection{Markov Property: One-Way Valve of Information Flow}

The core function of memory is to \textbf{isolate} past from future. In statistical inference, this is achieved through Markov blankets. In quantum dynamics, we formalize this as \textbf{conditional independence}.

Consider three time points $t_1 < t_2 < t_3$. Let observer's memory state at time $t_2$ be $\rho_{\text{mem}}(t_2)$.

\begin{theorem}[Markov Shielding of Memory]
\label{thm:markov-shielding}
An ideal memory system must act as a \textbf{sufficient statistic} between observer's past experiences and future behavior.

That is, given current memory $\rho_{\text{mem}}(t_2)$, observer's predictive distribution $P(O_3 | \rho_{\text{mem}}(t_2))$ at time $t_3$ is independent of specific environmental input $E_1$ at time $t_1$, unless that input has been encoded into $\rho_{\text{mem}}(t_2)$.
$$I(E_1 : O_3 \mid \rho_{\text{mem}}(t_2)) = 0$$

In QCA discrete dynamics, this means the update rule $U_{\text{mem}}$ of memory subsystem $\mathcal{M}_{\text{mem}}$ must be able to "shield" microscopic historical paths that were not recorded. Memory is \textbf{lossy compression} of history.
\end{theorem}

\subsection{Anti-decoherence Conditions: Pointer States and Decoherence-Free Subspaces}

The environment constantly monitors the observer, attempting to "leak" the observer's internal state through entanglement (decoherence). Memory can exist because physical laws allow certain states to be "immune" to environmental monitoring.

\begin{definition}[Pointer States]
\label{def:pointer-states}
Let observer-environment interaction Hamiltonian be $H_{\text{int}} = \sum_k A_k \otimes E_k$ ($A_k \in \mathcal{A}_{\text{int}}, E_k \in \mathcal{A}_{\text{env}}$).

\textbf{Pointer Basis} $\{ |p_i\rangle \}$ is the set of states satisfying the following commutation relation:
$$[H_{\text{int}}, |p_i\rangle\langle p_i|] = 0$$

This means when the system is in state $|p_i\rangle$, it does not establish entanglement with the environment, or rather, it is not destroyed by the environment but is "continuously measured" and stabilized in these states by the environment.
\end{definition}

\begin{theorem}[Algebraic Formulation of Quantum Darwinism]
\label{thm:quantum-darwinism}
Under QCA evolution, only operators belonging to algebra center $\mathcal{Z}(\mathcal{A}_{\text{int}})$ and commuting with interaction operators $A_k$ can serve as long-term memory.
$$\mathcal{M}_{\text{stable}} = \{ M \in \mathcal{A}_{\text{int}} \mid [M, H_{\text{self}}] \approx 0 \land [M, H_{\text{int}}] \approx 0 \}$$

These operators constitute \textbf{Decoherence-Free Subspaces (DFS)} or \textbf{noiseless subsystems}. Observer's memory is actually \textbf{Quantum Error Correction Codes} evolved by nature. Biological macromolecules (DNA) or brain memory traces are stable precisely because they utilize redundant encoding and energy gaps to resist environmental noise.
\end{theorem}

\subsection{Thermodynamic Cost of Memory: Metastable States and Non-equilibrium}

According to the second law of thermodynamics, the final state of a closed system is maximum entropy thermal equilibrium (memory disappears). Therefore, observer's memory cannot be absolutely stable ground states, but must be \textbf{metastable states}.

\begin{definition}[Memory Lifetime and Energy Barrier]
\label{def:memory-lifetime}
Let memory states $|0\rangle$ and $|1\rangle$ be two local minima on the potential energy surface, separated by a high energy barrier $\Delta E$.

According to Arrhenius law, the rate of spontaneous memory flip (forgetting) is:
$$\Gamma_{\text{forget}} \propto e^{-\Delta E / k_B T}$$

To maintain memory, the observer must:

\begin{enumerate}
\item \textbf{Maintain Low Temperature}: Pump internal entropy out through dissipative structures, keeping $T \ll \Delta E/k_B$.

\item \textbf{Active Error Correction}: Consume free energy for periodic reset or verification (similar to dynamic RAM refresh).
\end{enumerate}
\end{definition}

\begin{corollary}[Psychological Origin of Time Arrow]
\label{cor:time-arrow}
Since memory writing (measurement) is an irreversible process (entropy increase), and memory maintenance requires dissipating energy, observers can only have memories about the "past," not about the "future."

\textbf{The subjective sense of time flow is essentially a monotonic accumulation process of correlation information in memory subsystem $\mathcal{M}_{\text{mem}}$}. If memory stops updating or is erased, time stops for that observer.
\end{corollary}

\subsection{Summary: Observer as "Castle" of Information}

This section established the physical mechanism of memory:

\begin{enumerate}
\item \textbf{Structure}: Memory is a protected subfactor in internal algebra (center or DFS).

\item \textbf{Function}: Acts as Markov barrier, compressing history, predicting future.

\item \textbf{Cost}: Maintains metastable states by consuming energy, resisting decoherence.
\end{enumerate}

A physical entity only qualifies as an "observer" when it builds a "castle of information" through the above mechanisms, capable of sheltering its internal states in the entropy flow storm of the environment.

In the next section 18.4, we will explore how observers use these memories to construct \textbf{predictive models} of the external world—\textbf{homomorphic mappings between internal algebra and external environment}.

