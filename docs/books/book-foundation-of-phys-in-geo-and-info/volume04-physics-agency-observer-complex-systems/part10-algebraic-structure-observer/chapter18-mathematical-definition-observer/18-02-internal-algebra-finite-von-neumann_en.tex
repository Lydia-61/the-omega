\section{Internal Algebra $\mathcal{A}_{\text{int}}$: Finite-Dimensional von Neumann Algebras and Information Processing Capacity}

In Section 18.1, we defined observer $\mathfrak{O}$ as a dissipative structure with clear boundaries in QCA networks. The core of this structure is its \textbf{Internal Algebra} $\mathcal{A}_{\text{int}}$, representing the collection of all physical degrees of freedom that the observer can manipulate, store, and "perceive."

This section will delve into algebraic properties of $\mathcal{A}_{\text{int}}$. We will prove that, based on the finite information axiom established in Volume I, the observer's internal algebra must be a \textbf{Finite-dimensional von Neumann Algebra}. This mathematical fact has profound physical corollaries: it not only limits the observer's maximum information processing capacity (Bremermann's limit), but also naturally explains why the observer's subjective experience always contains "classical" components (such as definite memories and logical states) through the algebra's \textbf{Center} structure, thereby solving the basis selection problem in quantum mechanics interpretation.

\subsection{Structure Theorem of Internal Algebra}

In standard quantum mechanics, systems are usually described by operator algebra $\mathcal{B}(\mathcal{H})$ on full Hilbert space $\mathcal{H}$, which is a factor with trivial center (containing only identity operator). However, for a complex observer capable of carrying memory and logic, its internal structure must be richer.

According to the finite information axiom (Axiom A2), any local subsystem $\mathcal{A}_{\text{int}}$ is finite-dimensional. According to Wedderburn's structure theorem, any finite-dimensional $C^*$ algebra (or von Neumann algebra) is isomorphic to direct sum of matrix algebras.

\begin{theorem}[Observer Algebra Decomposition]
\label{thm:observer-algebra}
The observer's internal algebra $\mathcal{A}_{\text{int}}$ has the following unique canonical decomposition:
$$\mathcal{A}_{\text{int}} \cong \bigoplus_{k=1}^{N} \mathcal{M}_{d_k}(\mathbb{C}) \otimes \mathbf{1}_{m_k}$$

where:

\begin{enumerate}
\item \textbf{$N$ (Number of Classical Sectors)}: Represents the number of macroscopic classical states the observer can distinguish (e.g., number of memory register configurations).

\item \textbf{$\mathcal{M}_{d_k}(\mathbb{C})$ (Quantum Factor)}: $d_k \times d_k$ matrix algebra inside the $k$-th classical sector, representing quantum coherent degrees of freedom in that macroscopic state (such as processor qubits).

\item \textbf{$m_k$ (Multiplicity)}: Represents microscopic degeneracy, usually ignored in coarse-graining.
\end{enumerate}
\end{theorem}

\textbf{Physical Interpretation}:

The observer's world is not purely quantum, but \textbf{Hybrid Quantum-Classical}.

\begin{itemize}
\item \textbf{Between diagonal blocks} (different $k$): Mutually incoherent, obeying classical probability logic. This is the observer's "macroscopic state" or "memory content."

\item \textbf{Inside diagonal blocks} (fixed $k$): Obeying quantum superposition logic. This is the observer's "quantum computational power" or "fuzziness of perceptual moments."
\end{itemize}

\subsection{Algebra Center and Emergence of Classical Reality}

A pure quantum system (such as a single electron) has no "self" because it has no stable properties (unless measured). Observers can become "agents" because they possess \textbf{objective internal properties}. This algebraically corresponds to the non-trivial center of $\mathcal{A}_{\text{int}}$.

\begin{definition}[Algebra Center and Classical Memory]
\label{def:algebra-center}
The center $\mathcal{Z}(\mathcal{A}_{\text{int}})$ of algebra $\mathcal{A}_{\text{int}}$ is defined as the subset commuting with all elements in the algebra:
$$\mathcal{Z}(\mathcal{A}_{\text{int}}) = \mathcal{A}_{\text{int}} \cap \mathcal{A}'_{\text{int}} = \{ Z \in \mathcal{A}_{\text{int}} \mid [Z, A] = 0, \forall A \in \mathcal{A}_{\text{int}} \}$$

According to Theorem 18.2.1, the center is generated by projection operators $P_k$ corresponding to the $k$-th classical sector:
$$\mathcal{Z}(\mathcal{A}_{\text{int}}) \cong \bigoplus_{k=1}^{N} \mathbb{C} \cdot P_k$$
\end{definition}

\textbf{Physical Corollaries}:

\begin{enumerate}
\item \textbf{Stability of Memory}: Information stored in center $\mathcal{Z}$ (classical bits) is not disturbed by internal Hamiltonian $H_{\text{int}} \in \mathcal{A}_{\text{int}}$ (because $[H_{\text{int}}, Z] = 0$). This means the observer can perform quantum thinking (unitary evolution within $\mathcal{M}_{d_k}$) while keeping its classical memory ($k$ value) intact.

\item \textbf{Basis Selection (Pointer Basis)}: Environment-induced decoherence tends to diagonalize the observer's state onto the basis determined by $\mathcal{Z}$. This is why we perceive definite "pointer states" rather than macroscopic superpositions. \textbf{Classical reality is the center of the observer's internal algebra}.
\end{enumerate}

\subsection{Information Processing Capacity: Bremermann's Limit}

Since observers are physical entities, their computational capacity is limited by physical laws.

\begin{definition}[Computational Capacity]
\label{def:computational-capacity}
The observer's \textbf{Information Capacity} $I_{max}$ is determined by the logarithmic dimension of its Hilbert space:
$$I_{max} = \log_2 (\dim \mathcal{H}_{\text{int}}) = \log_2 \left( \sum_{k=1}^N d_k m_k \right)$$

The observer's \textbf{Processing Rate} $\nu_{max}$ is limited by its energy spread (Margolus-Levitin theorem):
$$\nu_{max} \le \frac{2 E_{\text{int}}}{\pi \hbar}$$

where $E_{\text{int}}$ is the observer's average internal energy relative to ground state.
\end{definition}

\begin{theorem}[Bremermann's Limit]
\label{thm:bremermann}
Combining mass-energy equation $E=mc^2$, for an observer with mass $m$, its maximum information processing rate is limited by:
$$R_{max} \le \frac{2 m c^2}{\pi \hbar} \approx 1.36 \times 10^{50} \text{ bits/s/kg}$$

In QCA universes, this limit is strict. It means a finite-mass observer (such as humans or AI) can only process finite computational tasks in finite time.

This is not just an engineering limitation, but an \textbf{epistemological boundary}: observers cannot completely simulate environmental subsystems more complex than themselves through computation in finite time (computational irreducibility).
\end{theorem}

\subsection{Subfactor Structure of Internal Algebra: Containment and Being Contained}

Observer $\mathcal{A}_{\text{int}}$ is not isolated; it is embedded in the total universe algebra $\mathcal{A}$. This embedding relationship can be described using Vaughan Jones's \textbf{Subfactor} theory.

\begin{definition}[Index of the Observer]
\label{def:observer-index}
Let $\mathcal{A}_{\text{int}} \subset \mathcal{A}$. The Jones index $[\mathcal{A} : \mathcal{A}_{\text{int}}]$ measures how "small" the observer is relative to the environment.

\begin{itemize}
\item If the index is infinite (in continuous field theory), the observer is negligible relative to the universe.

\item In QCA discrete ontology, the index is finite. This means there are non-trivial \textbf{quantum correlation constraints} between observer and universe.
\end{itemize}
\end{definition}

\begin{corollary}[Incompleteness of Subjective World]
\label{cor:incompleteness}
Since $\mathcal{A}_{\text{int}}$ is a proper subalgebra, there exist operators $O_{ext}$ in total algebra $\mathcal{A}$ that cannot be represented by any operator in $\mathcal{A}_{\text{int}}$.
$$O_{ext} \notin \mathcal{A}_{\text{int}}$$

This means there are always "hidden variables" in the universe (relative to the observer) that the observer cannot directly perceive or predict. For the observer, these inaccessible degrees of freedom manifest as \textbf{objective randomness} (such as unpredictable results of quantum measurements).

Therefore, \textbf{quantum randomness is not an intrinsic property of the universe, but an inevitable reflection of the incompleteness of observer algebra}.
\end{corollary}

\textbf{Summary}

This section established the algebraic skeleton of observers:

\begin{enumerate}
\item \textbf{Structure}: It is a finite-dimensional von Neumann algebra with $\bigoplus \text{Matrix}$ structure.

\item \textbf{Classicality}: The algebra's center $\mathcal{Z}$ defines the observer's classical memory and macroscopic states, solving the interpretation puzzle of superposition collapse.

\item \textbf{Limits}: Information capacity and processing rate are limited by physical energy (Bremermann's limit).

\item \textbf{Perspective}: The observer's finiteness leads to incompleteness in describing the external world (origin of quantum randomness).
\end{enumerate}

In the next section 18.3, we will explore how this algebraic structure resists erosion by time—the thermodynamic stability and anti-decoherence conditions of the \textbf{memory subsystem}.

