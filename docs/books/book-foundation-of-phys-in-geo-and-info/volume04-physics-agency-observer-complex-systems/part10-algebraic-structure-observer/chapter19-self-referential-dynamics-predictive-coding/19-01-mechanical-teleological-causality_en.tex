\section{Mechanical Causality and Teleological Causality: Dynamical Features Introduced by Self-referential Loops}

Physics usually rejects "teleology," considering it a relic of the Aristotelian era. However, in cybernetics and complex systems theory, teleology acquires a demystified definition: \textbf{negative feedback behavior}. This section will strictly distinguish two causal modes and prove that in QCA discrete networks, through specific topological connections (self-referential loops), mechanical causality can simulate teleological behavior, thereby endowing physical systems with "agency."

\subsection{Two Topological Modes of Causal Structure}

In QCA discrete ontology, causality is defined by connectivity in networks.

\begin{definition}[Mechanical Causality]
\label{def:mechanical-causality}
Let system state be $x_t$, environmental input be $e_t$. Mechanical causality is described by \textbf{feedforward} dynamical equation:
$$x_{t+1} = F(x_t, e_t)$$

This is a \textbf{push} dynamics: past states "push" the system into the future. System evolution trajectory is completely determined by initial conditions $x_0$ and microscopic rules $F$. Entropy usually increases or remains constant over time.
\end{definition}

\begin{definition}[Teleological Causality]
\label{def:teleological-causality}
Teleological causality is described by \textbf{feedback} dynamical equation. The system seems attracted by a future \textbf{target state} $x_{goal}$. Its effective evolution equation can be written in error correction form:
$$x_{t+1} = x_t - \gamma \nabla \mathcal{L}(x_t, x_{goal})$$

where $\mathcal{L}$ is a distance metric (such as relative entropy), $\gamma$ is learning rate or feedback gain.

This is a \textbf{pull} dynamics: future goals "pull" system evolution. Although this is still implemented by $F$ at the microscopic level, at the macroscopic level, system behavior manifests as acting \textbf{in order to} minimize $\mathcal{L}$.
\end{definition}

\subsection{Algebraic Construction of Self-referential Loops: Using Output as Input}

How to construct teleological systems in a purely mechanical QCA universe? The answer lies in \textbf{self-reference}.

In Section 18.4, we defined observer's internal model $\mathcal{A}_{\text{int}}$. If the observer not only simulates the environment but also \textbf{simulates itself}, self-reference is formed.

\begin{construction}[Self-referential Operator]
\label{constr:self-referential}
In observer's internal algebra $\mathcal{A}_{\text{int}}$, there exists a subalgebra $\mathcal{M}_{\text{self}} \subset \mathcal{A}_{\text{int}}$, whose state $\rho_{\text{self}}$ is a coarse-grained mapping of observer's overall state $\rho_{\text{tot}}$ (self-model):
$$\rho_{\text{self}}(t) = \Pi_{\text{self}}(\rho_{\text{tot}}(t))$$

QCA's update operator $U$ is designed to depend on this self-model:
$$x_{t+1} = U(x_t, \rho_{\text{self}}(t))$$

This forms a \textbf{causal closed loop} (Strange Loop):
$$x_t \xrightarrow{\text{map}} \rho_{\text{self}} \xrightarrow{\text{control}} x_{t+1}$$

This structure enables the system to "sense itself sensing," thereby adjusting behavior according to deviation between current state and ideal self-model (goal).
\end{construction}

\subsection{Error-Driven Dynamics: From $F=ma$ to $\dot{x} \propto -\nabla S$}

Self-referential loops introduce a new physical quantity: \textbf{Prediction Error}.

Let observer internally have an expected future state (prior belief) $\rho_{\text{prior}}$. Current perceptual state is $\rho_{\text{sense}}$.

The core mechanism of self-referential dynamics is: system evolution direction always tends to reduce \textbf{relative entropy (KL divergence)} between the two.

\begin{theorem}[Teleological Emergence Theorem]
\label{thm:teleological-emergence}
In a QCA subsystem containing self-referential feedback loops, if update rule $U$ satisfies \textbf{free energy minimization} principle (see Section 19.3 for details), then the subsystem's macroscopic trajectory will follow \textbf{gradient flow} equation:
$$\frac{d}{dt} \mathbf{x}(t) = -\Gamma \frac{\delta F}{\delta \mathbf{x}} + \mathbf{f}_{\text{solenoidal}}$$

where $F$ is variational free energy (as measure of prediction error).
\end{theorem}

\textbf{Physical Significance}:

\begin{enumerate}
\item \textbf{Attractor}: Free energy minima constitute \textbf{strange attractors} of the dynamical system. This attractor is the system's "purpose" or "steady state."

\item \textbf{Teleology}: External observers seeing the system automatically return to steady state under perturbations would think the system has "self-repair" or "goal-seeking" abilities. But this is essentially mechanical response of self-referential loops to error gradients.
\end{enumerate}

\subsection{Physical Origin of Normativity: Bridge Between "Is" and "Ought"}

David Hume proposed the gap between "Is" and "Ought": physical facts cannot derive value judgments.

However, in self-referential dynamics, this gap is filled by \textbf{Prior Models}.

\begin{itemize}
\item \textbf{Is}: Current perceptual state $\rho_{\text{sense}}$.

\item \textbf{Ought}: Internally solidified prior goal $\rho_{\text{prior}}$ (e.g., "body temperature should be 37°C").

\item \textbf{Dynamics}: Physical laws drive the system to eliminate differences between $Is$ and $Ought$ ($Is \to Ought$).
\end{itemize}

\begin{corollary}[Physical Definition of Value]
\label{cor:value-physical}
In physics, "value" or "utility" is not an ethereal concept, but \textbf{negative prediction error} (or negative entropy).

An observer considering some state "good" physically means that state is highly consistent with its internal prior model (low surprise).

Self-referential dynamics explains why living systems always actively seek low-entropy resources (food, information): because this is necessary to maintain predictive success of their internal models.
\end{corollary}

\textbf{Summary}

This section proved:

\begin{enumerate}
\item \textbf{Structure}: Self-reference is a feedback loop where the system uses its own state as input.

\item \textbf{Function}: Self-reference transforms mechanical push into teleological pull (error minimization).

\item \textbf{Philosophy}: This provides solid physical foundation for "intentionality" and "normativity," i.e., \textbf{quantized implementation of cybernetics}.
\end{enumerate}

In the next section 19.2, we will further formalize this dynamics, introduce \textbf{self-referential update operators}, and derive state evolution equations containing predictive feedback. This will directly lead to the core of brain working principles—predictive coding theory.

