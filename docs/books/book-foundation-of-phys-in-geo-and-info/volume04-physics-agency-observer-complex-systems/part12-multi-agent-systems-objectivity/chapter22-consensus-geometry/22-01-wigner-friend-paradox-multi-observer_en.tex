\section{Wigner's Friend Paradox and Multi-Observer Consistency Problem}

Fundamental debates in quantum mechanics often focus on Wigner's Friend paradox. This thought experiment sharply reveals that if we push unitarity of quantum mechanics to extremes, descriptions of the same physical event by different observers may become irreconcilably contradictory. This section will reconstruct this paradox from the perspective of QCA discrete ontology and point out that its resolution lies in abandoning "absolute states" and turning to "relational consistency."

\subsection{Algebraic Conflict of Nested Observers}

Consider two observers: Friend ($\mathcal{F}$) and Wigner ($\mathcal{W}$). Friend measures a spin system $S$ in the laboratory; Wigner observes the composite of friend and system from outside the laboratory.

\begin{enumerate}
\item \textbf{Friend's Perspective}:

Friend measures spin $S$ (in state $\alpha |\uparrow\rangle + \beta |\downarrow\rangle$), observes result (e.g., $|\uparrow\rangle$). For friend, system undergoes non-unitary \textbf{topological fusion} (Section 21.4):
$$\rho_S \xrightarrow{\text{Friend}} |\uparrow\rangle\langle\uparrow|$$

Friend not only "knows" the result, but records it in memory subsystem $\mathcal{M}_{\mathcal{F}}$, which is an irreversible physical fact.

\item \textbf{Wigner's Perspective}:

For Wigner, the laboratory (including friend) is a closed quantum system, following unitary evolution $U$. State described by Wigner is entangled:
$$|\Psi_{\mathcal{W}}\rangle = \alpha |\uparrow\rangle_S \otimes |\text{"up"}\rangle_{\mathcal{F}} + \beta |\downarrow\rangle_S \otimes |\text{"down"}\rangle_{\mathcal{F}}$$

In Wigner's view, no collapse occurred. Friend is in superposition of "seeing up" and "seeing down."
\end{enumerate}

\textbf{Paradox}: Friend believes "a definite fact occurred," Wigner believes "no fact occurred, only superposition." If Wigner subsequently performs interference experiments on the entire laboratory, he can prove existence of superposition, seemingly negating reality of friend's subjective experience.

In QCA framework, this manifests as algebraic containment contradiction: $\mathcal{A}_{\mathcal{F}} \subset \mathcal{A}_{\mathcal{W}}$. Center (classical facts) of internal algebra $\mathcal{A}_{\mathcal{F}}$ may be non-central (quantum operators) in external algebra $\mathcal{A}_{\mathcal{W}}$.

\subsection{Topological Independence Theorem of Observers}

To resolve this contradiction, we must utilize the \textbf{Minimal Strongly Connected Component (MSCC)} concept established in Chapter 20.

\begin{theorem}[Observer Topological Independence]
\label{thm:observer-independence}
In QCA networks, if $\mathcal{F}$ is a consciousness soliton (MSCC) with non-trivial $\mathbb{Z}_2$ topological index ($\nu=1$), then $\mathcal{F}$ cannot be losslessly contained in another observer $\mathcal{W}$'s \textbf{internal predictive model}.

That is, Wigner cannot completely simulate friend's topological state in his finite internal algebra $\mathcal{A}_{\mathcal{W}}$, unless Wigner undergoes \textbf{physical topological fusion} with friend (which would destroy Wigner's external observer status).
\end{theorem}

\textbf{Physical Corollary}:

Entangled state $|\Psi_{\mathcal{W}}\rangle$ written by Wigner is only an \textbf{Effective External Model} of the laboratory. This model is statistically correct (for predicting interference experiments), but \textbf{ontologically incomplete}.

\begin{itemize}
\item $|\Psi_{\mathcal{W}}\rangle$ describes "friend in Wigner's horizon," not "friend as friend experiences himself."

\item Friend's subjective collapse is real (for himself), because it corresponds to topological phase transition of his own QCA subnet.

\item There is no God's-eye-view "absolute wave function" simultaneously containing truth of both. Truth is \textbf{Local} and \textbf{Perspectival}.
\end{itemize}

\subsection{Conditions for Shared Reality: Multi-Observer Consistency}

Since reality is relative, how do we avoid solipsism? The answer lies in \textbf{Communication}.

When Wigner opens the laboratory door and asks friend: "What did you see?", a causal connection is established between the two observers.

\begin{definition}[Consistency Condition]
\label{def:consistency-condition}
Let result recorded by $\mathcal{F}$ be $r_{\mathcal{F}}$, answer recorded by $\mathcal{W}$ be $r_{\mathcal{W}}$.

If in all repeated experiments, joint probability distribution of their records satisfies:
$$P(r_{\mathcal{W}} = r_{\mathcal{F}}) \approx 1$$

then these two observers achieve consistency in the \textbf{classical limit}.

This means, although their descriptions differ at measurement time, after \textbf{causal cone intersection (information exchange)}, their internal algebra centers $\mathcal{Z}_{\mathcal{F}}$ and $\mathcal{Z}_{\mathcal{W}}$ must establish strong correlations (classical entanglement).
\end{definition}

\begin{theorem}[No-Go Theorem for "Facts"]
\label{thm:no-go-facts}
Brukner (2018) and Frauchiger-Renner (2018) proved that the following three assumptions are incompatible:

\begin{enumerate}
\item \textbf{Q}: Quantum mechanics universally applies (unitarity).

\item \textbf{C}: Consistency (observations of different observers do not contradict).

\item \textbf{S}: Single world (each measurement has only one result).
\end{enumerate}

In QCA framework, we abandon \textbf{S (absolute single world)}, replacing it with \textbf{R (relational world)}.

\begin{itemize}
\item "Facts" are not absolute, but defined relative to observers.

\item Objective reality is not the set of all facts, but the subset of information that is \textbf{Shareable} and \textbf{Non-contradictory} among all observers.
\end{itemize}
\end{theorem}

\subsection{Introduction of Consensus Geometry}

This leads us to propose a new geometric concept: \textbf{Consensus Geometry}.

Spacetime itself may not be a bottom-level physical entity, but a \textbf{Public Protocol} emerged from countless local observers (QCA subsystems) to coordinate each other's causal relationships.

\begin{itemize}
\item Each observer $O_i$ has its own private metric $g_i$ (based on its internal clock $\kappa_i$ and measurements).

\item When $O_i$ and $O_j$ communicate, they attempt to calibrate each other's clocks and rulers.

\item \textbf{Objective Spacetime} $g_{\mu\nu}$ is the \textbf{Fixed Point} reached by this vast observer network through continuous calibration.
\end{itemize}

In following sections, we will formalize this process in information geometry language: gravitational interactions are actually "Bayesian flows" generated by observers to minimize \textbf{Relative Entropy (disagreement)}.

\textbf{Summary}

This section established core principles of multi-agent physics through Wigner's friend paradox:

\begin{enumerate}
\item \textbf{Irreducibility of Perspective}: Each observer is an independent topological center, possessing private facts.

\item \textbf{Dynamicity of Consistency}: Objectivity is not presupposed, but established through communication.

\item \textbf{Relationality of Reality}: Physical laws must be expressed as consistency constraints among observers.
\end{enumerate}

In the next section 22.2, we will specifically construct this consistency mechanism, proposing \textbf{Bayesian Gravity} theory.

