\section{Bayesian Gravity: Observer Model Synchronization Driven by Relative Entropy Minimization}

In Section 22.1, we revealed failure of "absolute objectivity" through Wigner's friend paradox: physical reality is not an a priori given single state, but a \textbf{Relational} property of information in observer networks. This raises a profound dynamical question: since each observer has its own private algebra $\mathcal{A}_{\text{int}}$ and subjective model, why don't we live in countless mutually disconnected illusion bubbles? What force "pulls" these discrete subjective perspectives together, forming a coherent, shared objective spacetime?

This section will propose \textbf{Bayesian Gravity} hypothesis. We will prove that gravitational interactions in general relativity, at the bottom level of information geometry, are \textbf{Statistical Flows} generated by observers to eliminate \textbf{Cognitive Dissonance}—i.e., relative entropy. Gravity is not a force pulling objects, but a \textbf{tendency to pull together ideas (models)}.

\subsection{Cognitive Distance and Disagreement Functional}

Consider two neighboring observers $O_A$ and $O_B$ in QCA networks. They each possess internal predictive models $\rho_A$ and $\rho_B$ for the same spacetime region (causal diamond $\Sigma$) (defined on their respective internal algebras, mapped to public boundary algebra $\mathcal{A}_{\partial}$ through communication channels).

\begin{definition}[Cognitive Distance]
\label{def:cognitive-distance}
Cognitive distance between observers $A$ and $B$ is defined as \textbf{Quantum Relative Entropy} of their prediction distributions about public events:
$$D_{AB} \equiv S(\rho_A \| \rho_B) = \text{Tr}(\rho_A \ln \rho_A - \rho_A \ln \rho_B)$$

This quantity is asymmetric, measuring "surprise" or model correction cost when $A$ receives $B$'s data.
\end{definition}

\begin{definition}[Network-Wide Disagreement Functional]
\label{def:disagreement-functional}
For entire observer network $\mathcal{G} = \{O_i\}$, system's \textbf{Total Cognitive Potential} is weighted sum of relative entropy on all connection channels:
$$\mathcal{V}_{\text{cog}}[\{\rho_i\}] = \sum_{(i,j) \in E} w_{ij} S(\rho_i \| \rho_j)$$

where $w_{ij}$ is communication weight (coupling strength), depending on causal distance or channel bandwidth between observers.
\end{definition}

\subsection{Bayesian Update as Gradient Flow}

According to free energy principle (Section 19.3), each observer's dynamical goal is to minimize its own prediction error. In multi-agent environments, this means each observer continuously corrects its model $\rho_i$ to reduce disagreement with neighbors $\rho_j$ (i.e., minimize $\mathcal{V}_{\text{cog}}$).

\begin{theorem}[Bayesian Gravity Flow]
\label{thm:bayesian-gravity}
In continuous time limit, if observers follow optimal rules of Bayesian inference (Bayes' Rule), evolution trajectory of their internal states $\rho_i$ on statistical manifold $\mathcal{S}$ follows \textbf{gradient flow equation}:
$$\frac{d\theta^a_i}{dt} = - g^{ab}(\theta_i) \frac{\partial}{\partial \theta^b_i} \sum_j w_{ij} S(\rho(\theta_i) \| \rho(\theta_j))$$

where $\theta^a_i$ are model parameters, $g^{ab}$ is inverse matrix of \textbf{Fisher Information Metric}.

This equation is isomorphic in form to \textbf{Geodesic Deviation Equation} (or force motion equation) in general relativity:

\begin{itemize}
\item \textbf{Fisher Metric $g_{ab}$} plays the role of \textbf{Spacetime Metric $g_{\mu\nu}$}.

\item \textbf{Relative Entropy Gradient $\nabla S$} plays the role of \textbf{Gravitational Potential Gradient $\nabla \Phi$}.
\end{itemize}
\end{theorem}

\textbf{Physical Interpretation}:

When observer $A$ finds its prediction inconsistent with $B$ ($S(\rho_A \| \rho_B) > 0$), Bayesian update drives $A$ to modify parameters, making its state $\rho_A$ \textbf{"fall"} toward $\rho_B$.

This "mutual attraction" in information geometry space manifests as universal gravitation in macroscopic physics. \textbf{Objects attract each other because information they carry attempts to reach consensus}.

\subsection{Origin of Mass: Inertia of Belief}

Why are some objects (large mass) difficult to move, while others (small mass) easy? In Bayesian gravity, this corresponds to \textbf{Inertia of Belief}.

\begin{definition}[Informational Mass]
\label{def:informational-mass}
Observer's "mass" $M_i$ is defined as \textbf{Precision} or \textbf{Inverse Variance} of its internal prior distribution.
$$M_i \propto \text{Tr}(g_{ab}(\theta_i)) \approx \text{Fisher Information Content}$$

\begin{itemize}
\item \textbf{Large Mass (High Precision)}: Observer possesses vast historical data or extremely strong prior beliefs (such as black holes or stars). When interacting with low-precision observers (test particles), large-mass observer almost does not change its state ($\Delta \theta \approx 0$), forcing the other to significantly correct its model.

\item \textbf{Small Mass (Low Precision)}: Observer is uncertain about its own state (such as electrons or photons). It is easily "pulled" toward large-mass observer's model.
\end{itemize}

This explains \textbf{Equivalence Principle}: inertial mass (ability to resist model updates) and gravitational mass (ability to pull others to update models) both stem from the same statistical quantity—Fisher information content.
\end{definition}

\subsection{Emergence of General Relativity: From Consensus to Curvature}

If all observers mutually "fall" and reach consensus, what is the final state?

\begin{theorem}[Consensus Manifold Theorem]
\label{thm:consensus-manifold}
Under long-term evolution of Bayesian flow, observer network tends toward a \textbf{Nash Equilibrium State}. In this equilibrium state, local metrics $g^{(i)}_{ab}$ of all observers patch together into a globally smooth Riemannian manifold $(\mathcal{M}, g_{\mu\nu})$.

Curvature $R_{\mu\nu}$ of this manifold corresponds to \textbf{residual, irreducible Information Tension} in the network.

\begin{itemize}
\item \textbf{Flat Spacetime ($R=0$)}: All observers completely agree, no information pressure in network.

\item \textbf{Curved Spacetime ($R \neq 0$)}: Due to topological constraints or matter distribution, observers cannot achieve global agreement (e.g., around black holes). Although local consensus is reached, after parallel transporting model parameters around a closed loop, offset occurs (holonomy). This \textbf{Failure of Consensus} is precisely the definition of \textbf{Curvature}.
\end{itemize}
\end{theorem}

\begin{corollary}[Bayesian Interpretation of Einstein Equations]
\label{cor:bayesian-einstein}
Einstein field equations $G_{\mu\nu} = 8\pi G T_{\mu\nu}$ can be restated as:

\textbf{Curvature of Consensus (Geometric Tension) is Proportional to Flux of Information (Material Tension).}

Matter ($T_{\mu\nu}$) is the source of information, continuously injecting new "surprise" into the network, disrupting consensus. Gravity ($G_{\mu\nu}$) is geometric deformation generated by the network to digest this surprise and restore local balance.
\end{corollary}

\textbf{Summary}

This section proposed Bayesian gravity theory, reducing universal gravitation to \textbf{Bayesian model synchronization mechanism} in multi-agent systems.

\begin{enumerate}
\item \textbf{Force as Update}: Gravity is statistical tendency of observers to correct models to minimize relative entropy.

\item \textbf{Mass as Belief}: Inertia is rigidity of prior probability distributions.

\item \textbf{Spacetime as Consensus}: Objective physical world is the greatest common divisor reached by countless subjective worlds through continuous games and calibration.
\end{enumerate}

This view completely eliminates opposition between "subjective" and "objective": \textbf{Objectivity is just the limiting form of Intersubjectivity}.

In the next section 22.3, we will explore stability of this consensus mechanism, i.e., how \textbf{Objective Reality as Nash Equilibrium} is locked in game-theoretic framework.

