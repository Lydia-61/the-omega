\chapter{The Modular Flow Hypothesis}

\section{Who is Ticking?}

\begin{quote}
``We always ask: `What time is it?' as if time were an entity independent of us. But the deep algebra of quantum mechanics tells us there is no universal clock. It is you---as an unbalanced quantum state---who creates time yourself in the process of having to flow to find balance.''
\end{quote}

\subsection{God Has No Clock}

Let us return to the Schrödinger equation $|\dot{\psi}\rangle = -iH|\psi\rangle$. Although elegant, this equation implies a huge assumption: it assumes that a parameter named \textbf{$t$} already exists, and a Hamiltonian named \textbf{$H$} drives the evolution.

But under extreme conditions of quantum gravity (such as the Wheeler-DeWitt equation), $H \Psi = 0$. The entire universe's Hamiltonian is zero, and the time parameter $t$ disappears. The universe appears frozen.

If the bottom is frozen, why do we feel flow?

Who is ticking?

In the paper, we touched the core of this question: the concept of ``modular `thermal' time'' suggests that time is not fundamental but emergent.

From this perspective, \textbf{time originates from ``ignorance''}.

\subsection{State as Clock}

To understand this, we need to introduce \textbf{Tomita-Takesaki Theory} from algebraic quantum field theory. This is a mathematical theorem hailed by many physicists as ``theological level.''

It tells us: As long as you are given a von Neumann algebra (representing observables) and a specific quantum state $|\Omega\rangle$ (representing the system's current state), the mathematical structure will \textbf{automatically generate} a one-parameter unitary evolution group.

This evolution group is called the \textbf{Modular Flow}, denoted $\Delta^{it}$.

\begin{itemize}
\item \textbf{Traditional physics}: First there is time $t$, then there is evolution $U(t)$, and finally this causes state changes.

\item \textbf{Modular flow physics}: First there is state $|\Omega\rangle$, and this state's intrinsic entanglement structure \textbf{defines} what ``evolution'' is, thereby \textbf{defining} what time $t$ is.
\end{itemize}

\textbf{Time is intrinsic to state}.

A system in thermal equilibrium has a static (or trivial) modular flow.

While a system far from equilibrium, containing complex entanglement (such as life or civilization), has an intense modular flow.

\textbf{It is your own quantum state that is ticking.} Every breath you take, every thought you have, is not consuming time but \textbf{generating} your own modular flow.

\subsection{Physical Meaning of Thermal Time}

This hypothesis perfectly explains the ``thermal time'' concept we encountered in the previous volume.

French physicists Alain Connes and Carlo Rovelli proposed the \textbf{Thermal Time Hypothesis}:

\textbf{The time passing we perceive is actually the modular flow of the statistical state we are in.}

\begin{itemize}
\item When we are in a pure state (omniscient), there is no time.

\item When we are in a mixed state (with forgetting, with entropy), modular flow appears.
\end{itemize}

This coincides with the $c_{FS}$ budget allocation we discussed in the first book.

When we cut the system out from the environment (producing $v_{env}$ and entanglement entropy), we not only create the ``self'' but also create the ``time'' that drives self-evolution.

\textbf{Time is the product of ignorance.}

Because we cannot see the full picture of the universe (all entanglement of $|\Psi\rangle$), we can only see local projections. This information deficit (entropy) is mathematically transformed into a generator (Hamiltonian) that drives system evolution.

\subsection{Conclusion: The Objectivity of Subjective Time}

``Who is ticking?''

The answer is: \textbf{You yourself.}

The universe does not have a unified Greenwich time. Every subsystem, according to its entanglement with the environment, has its unique modular flow, that is, a unique \textbf{``private time''}.

\begin{itemize}
\item A photon's modular flow is stagnant, so it has no time.

\item A black hole horizon's modular flow is extremely redshifted, so time freezes there.

\item Our modular flow is linear, so we feel birth, aging, sickness, and death.
\end{itemize}

We do not need to find that Prime Mover who winds the universe's clock.

\textbf{Existence itself is the clockwork.} As long as you exist (in a non-trivial entangled state), you will necessarily experience the modular flow defined by your own state. You are not only the observer; you are your own clock.

Since time is generated by state, how does this generation mechanism operate mathematically? How does that alchemist called the ``Tomita operator'' extract dynamic time flow from static entanglement?

This leads to the theme of the next section: \textbf{Tomita's Alchemy}. We will delve into the mathematical core of this theorem hailed as ``quantum mechanics' most profound theorem'' to witness the alchemical process of time's birth.

