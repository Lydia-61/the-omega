\chapter{The Derivative is the Ontology}

\section{The Power of the Instant}

\begin{quote}
``On the curve of $e^x$, you cannot find a stationary point. Because the height of every point precisely defines its slope. At the foundation of the universe, `what I am' and `where I am going' are different expressions of the same question.''
\end{quote}

\subsection{The Miracle of Derivatives}

Let us return to the birthplace of calculus. Newton and Leibniz invented derivatives to describe the rate of change. For almost all functions, differentiation is a ``transformation'' operation:

\begin{itemize}
\item $x^3$ becomes $3x^2$.

\item $\sin x$ becomes $\cos x$.
\end{itemize}

However, in the pantheon of mathematics, there exists a unique fixed point. That is the natural exponential function $f(t) = e^t$.

$$\frac{d}{dt} e^t = e^t$$

This equation is incredibly simple, but it contains the ultimate secret of \textbf{Self-Driving}: \textbf{The value of the function at this moment (noumenon) directly determines its rate of change at this moment (trend).}

If the universe's wave function follows exponential evolution $|\Psi(t)\rangle = e^{-iHt} |\Psi(0)\rangle$, it must inherit this divinity. Let us rewrite the Schrödinger equation:

$$|\dot{\Psi}\rangle = -iH |\Psi\rangle$$

Please gaze at this formula.

\begin{itemize}
\item \textbf{Left side ($|\dot{\Psi}\rangle$)}: The trend of change, the arrow toward the next second.

\item \textbf{Right side ($|\Psi\rangle$)}: The current state, existence here and now.
\end{itemize}

Between them is only a linear operator $-iH$. This means that \textbf{the rate of change is not something external to the state; the rate of change is isomorphic to the state.}

\subsection{Spinoza's Physics}

This mathematical fact revives an ancient concept from the 17th-century philosopher Spinoza: \textbf{``Conatus'' (endeavor/potential)}.

Spinoza believed that all things have an inherent tendency to ``endeavor to maintain their existence.'' In classical physics, this is understood as inertia. But in \textbf{Vector Cosmology}, this is understood as \textbf{Generation}.

In the universe of $e$, a particle does not need external force to maintain its existence.

\begin{itemize}
\item It exists because it is rotating.

\item It rotates because its vector length in Hilbert space is not zero.
\end{itemize}

As long as $|\Psi\rangle$ is not zero, according to the derivative formula, $|\dot{\Psi}\rangle$ is not zero (assuming $H$ is non-trivial).

\textbf{Existence is flux.}

The universe does not need a Prime Mover. The Big Bang is not God giving a kick; it is the $t=0$ moment of the function $e$. As long as we have $1$ (existence), the derivative mechanism will automatically generate the subsequent infinite sequence.

\subsection{The Tyranny and Freedom of the Infinitesimal}

This mechanism endows the ``instant'' with unparalleled power.

In the macroscopic world, we think a ``moment'' is insignificant. But in the geometry of Lie algebras, the \textbf{Infinitesimal} contains everything.

If we know the system's state $|\Psi(0)\rangle$ at moment $t=0$ and its generator $H$, we know everything about its future.

All history is actually already \textbf{folded} in the tangent space at moment $t=0$.

\begin{itemize}
\item The next 10 billion years are merely the integral trajectory of this instant's tangent, driven by $e$, unfolding along the surface of the manifold.
\end{itemize}

This sounds like a cage of determinism. But this is precisely the beauty of \textbf{$e$}---although it is deterministic, it is \textbf{endogenous}.

Your future is not determined by an external fate planning bureau, but by your own current state ($|\Psi\rangle$) and your internal structure ($H$).

\textbf{You are your own driving force.}

\subsection{Conclusion: Self-Referential Dynamics}

At this point, we understand why $e$ is the natural generator.

It eliminates the binary opposition between ``motion'' and ``rest.''

At the microscopic foundation, there is no absolute rest. What we call ``rest mass'' is actually exponential rotation at the speed of light in the internal dimension ($v_{int} = c_{FS}$).

The universe is a perpetual motion machine. But this does not violate the laws of thermodynamics, because it is \textbf{geometric perpetual motion}---on the great circle of projective space, relying on the mathematical privilege of ``derivative equals self,'' it performs eternal frictionless gliding.

Since the ``instant'' contains all the power, how does this power operate specifically? What does that mysterious operator \textbf{$H$ (Hamiltonian)} that connects state and rate of change actually look like?

This leads to the theme of the next section: \textbf{The Generation of Lie Algebra}. We will see that the seemingly complex forces in the universe (gravity, electromagnetism, strong force) actually all grow from tiny generators through a mathematical language called ``Lie algebra.''

