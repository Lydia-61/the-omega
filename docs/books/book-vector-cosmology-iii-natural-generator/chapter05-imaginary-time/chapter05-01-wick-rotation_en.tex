\chapter{Imaginary Time}

\section{Wick Rotation}

\begin{quote}
``If we rotate the time axis 90 degrees on the complex plane, that quantum world that was oscillating incessantly suddenly quiets down and freezes into a statistical thermodynamic world. This is not a mathematical game; this is the geometric secret passage to grand unification.''
\end{quote}

\subsection{Coincidence of Two Formulas?}

Let us place the two greatest exponential formulas in physics side by side and play a ``spot the difference'' game.

\begin{enumerate}
\item \textbf{Schrödinger Evolution Operator (Quantum Evolution)}:

    Describes the evolution of a quantum state with \textbf{time $t$}.

    $$U(t) = e^{-i \frac{H}{\hbar} t}$$

    Here, $i$ is the imaginary unit, meaning this is a \textbf{rotation factor} (modulus conserved, phase changes).

\item \textbf{Boltzmann Factor}:

    Describes the statistical distribution of a thermodynamic system at \textbf{temperature $T$}. We usually use inverse temperature $\beta = 1/(k_B T)$ to represent it.

    $$\rho(\beta) = e^{- H \beta}$$

    Here there is no $i$, meaning this is a \textbf{decay factor} (modulus changes, probability weight distribution).
\end{enumerate}

Please gaze at these two formulas. Their only difference lies in the variable in the exponent:

\begin{itemize}
\item One is \textbf{$it$ (imaginary time)}.

\item One is \textbf{$\beta$ (real temperature)}.
\end{itemize}

Is this merely a symbolic coincidence?

In \textbf{Vector Cosmology}, we refuse to believe in coincidences. If two formulas look the same, it means the physical entities behind them are \textbf{homologous}.

\subsection{Rotating the Time Axis}

If we perform a bold operation: rotate the variable ``time'' from the real axis to the imaginary axis.

Let $t = -i \tau$ (where $\tau$ is called \textbf{Euclidean time} or \textbf{imaginary time}).

Substituting into the Schrödinger formula:

$$U(t) = e^{-i H (-i \tau)} = e^{-H \tau}$$

A miracle occurs. That quantum formula describing waves instantly becomes the Boltzmann formula describing thermal distribution!

As long as we let $\tau = \beta$ (i.e., imaginary time length equals inverse temperature), quantum mechanics \textbf{becomes} thermodynamics.

This mathematical operation is called \textbf{Wick Rotation}.

It establishes a one-to-one mapping between Minkowski spacetime (spacetime with light cones and causality) and Euclidean space (pure spatial geometry).

\subsection{Geometric Definition of Temperature}

This discovery completely reshapes our understanding of \textbf{temperature}.

In classical physics, temperature is the intensity of molecular motion.

But from the underlying perspective of FS geometry, \textbf{temperature is the time period in the imaginary dimension}.

\begin{itemize}
\item \textbf{Heat (Thermal)} is not a form of chaotic energy.

\item \textbf{Heat (Thermal)} is the ``distance'' traveled by the system evolving along the \textbf{imaginary time axis}.
\end{itemize}

This explains why quantum field theory must use imaginary time techniques when calculating black hole entropy or vacuum fluctuations. Because in that limiting state, real physical time ($t$) and thermodynamic temperature ($\beta$) have become entangled and cannot be distinguished.

\subsection{Hawking's No-Boundary Universe}

Stephen Hawking used this concept to propose the ``no-boundary universe'' model. He believed that in the very early universe (Planck era), time itself underwent Wick rotation, becoming a pure spatial dimension.

\begin{itemize}
\item There is no ``beginning,'' nor is there a ``singularity.''

\item The beginning of the universe is as smooth as the South Pole of Earth.
\end{itemize}

In our \textbf{Natural Generator} framework, this receives a new interpretation:

The noumenon of the universe is a \textbf{Holomorphic Function} defined on the complex domain.

\begin{itemize}
\item When we slice along the \textbf{real axis}, we see \textbf{time passing} (quantum evolution).

\item When we slice along the \textbf{imaginary axis}, we see \textbf{thermal equilibrium} (statistical distribution).
\end{itemize}

The reason we perceive the world as ``cold'' (with a definite temperature) and ``moving'' (with passing time) is because there is a specific angle between our observation slice and this high-dimensional complex manifold.

\subsection{Conclusion: Two Sides of a Coin}

Wick rotation tells us that $e$ is the hub connecting two worlds.

\begin{itemize}
\item \textbf{$e^{ix}$ (wave/circle)}: This is the world of Book I. Conserved, coherent, eternal motion.

\item \textbf{$e^{-x}$ (decay/spiral)}: This is the world of Book II. Dissipative, probabilistic, thermodynamic arrow.
\end{itemize}

They are not opposing physical laws; they are projections of the same exponential function $e^z$ in different directions on the complex plane.

\textbf{Quantum mechanics is imaginary thermodynamics.}

\textbf{Thermodynamics is imaginary quantum mechanics.}

Since temperature and time are mathematically interchangeable, this raises a deeper question: \textbf{Which generates which?}

Does time passing cause heat (entropy increase), or does heat (statistical state) generate time?

This leads to the theme of the next section: \textbf{Unification of Geometry and Heat}. We will see that at that ultimate level ruled by $e$, physics no longer distinguishes between ``evolution'' and ``distribution''; they are unified in the same \textbf{Modular Flow}.

