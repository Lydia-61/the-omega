\section{全局幺正演化模型}

\textbf{(The Global Unitary Model, QTM)}

在第一章中,我们确立了物理实在的有限性与可计算性。基于此,我们可以构建一个描述整个宇宙(包含所有物质、能量及观测者)的完备数学模型。在这一模型中,我们将宇宙视为一台与外界隔离的、自洽运行的\textbf{量子图灵机(Quantum Turing Machine, QTM)}。

该模型代表了物理学追求的理想客观视角——即去除所有主观观测效应后的宇宙本体。我们将看到,在这个视角下,宇宙是一个严格决定论的、信息守恒的、包含所有历史分支的静态结构。

\subsection{全域希尔伯特空间}

根据\textbf{有限信息公理},宇宙的总自由度是有限的。设宇宙包含 $N$ 个基本信息单元(量子比特),则全域希尔伯特空间 $\mathcal{H}_{total}$ 可以定义为 $N$ 个局域二维希尔伯特空间的张量积:

\begin{equation}
\mathcal{H}_{total} = \bigotimes_{x \in \Lambda} \mathcal{H}_x \cong (\mathbb{C}^2)^{\otimes N}
\end{equation}

其中 $\Lambda$ 是离散的时空晶格。这一空间的维数 $D = 2^N$ 虽然巨大,但并非无限。宇宙在任意时刻的\textbf{全域量子态(Global Quantum State)} $|\Psi(t)\rangle$ 是 $\mathcal{H}_{total}$ 中的一个单位向量。

这一状态向量 $|\Psi(t)\rangle$ 包含了宇宙中所有粒子的位置、动量、自旋以及所有复杂的纠缠关系。它是对物理实在的\textbf{完备描述}。根据量子力学的线性叠加原理,它可以被展开为一组正交基底(如所有可能的经典构型)的线性组合:

\begin{equation}
|\Psi(t)\rangle = \sum_{i=1}^{D} c_i(t) |World_i\rangle
\end{equation}

这里,每一个 $|World_i\rangle$ 代表一个特定的经典宇宙快照(Snapshot)。系数 $c_i(t)$ 是复数概率幅,其模方 $|c_i|^2$ 代表该构型在全域波函数中的权重。

\subsection{永恒的动力学:全局幺正算符 $\hat{U}$}

在 QTM 模型中,宇宙是一个封闭系统,不与任何外部环境发生相互作用。因此,其演化严格遵循量子力学的\textbf{幺正性(Unitarity)}。

我们将物理定律编码为一个全局幺正演化算符 $\hat{U}$。宇宙状态随离散时间步 $t$ 的演化方程为:

\begin{equation}
|\Psi(t+1)\rangle = \hat{U} |\Psi(t)\rangle
\end{equation}

这一方程是离散版本的薛定谔方程。算符 $\hat{U}$ 必须满足幺正条件 $\hat{U}^\dagger \hat{U} = I$,这保证了全域波函数的模长守恒:

\begin{equation}
\langle \Psi(t+1) | \Psi(t+1) \rangle = \langle \Psi(t) | \hat{U}^\dagger \hat{U} | \Psi(t) \rangle = 1
\end{equation}

\textbf{物理推论:信息守恒}

幺正演化意味着全域量子态的演化是\textbf{可逆的(Reversible)}。如果我们知道当前时刻的状态 $|\Psi(t)\rangle$ 和物理定律 $\hat{U}$,我们不仅可以完美预测未来的任意状态 $|\Psi(t+n)\rangle$,也可以完美回溯过去的状态 $|\Psi(t-n)\rangle = (\hat{U}^\dagger)^n |\Psi(t)\rangle$。

在 QTM 模型中,\textbf{信息从未被创造,也从未被销毁}。所谓的"熵增"或"遗忘",仅仅是信息从局域自由度扩散到了全局纠缠关联中,对于全知全能的上帝视角(拥有 $\hat{U}^\dagger$ 的计算能力),宇宙的冯·诺依曼熵始终保持为零(纯态)。

\subsection{块宇宙与费曼路径求和}

如果我们考察 $|\Psi(t)\rangle$ 在时间轴上的展开,QTM 模型呈现出一个\textbf{块宇宙(Block Universe)}的图景。

利用费曼路径积分(在离散架构下为路径求和),从初始时刻 $t=0$ 到时刻 $T$ 的演化可以表示为所有可能历史路径的相干叠加:

\begin{equation}
|\Psi(T)\rangle = \sum_{\text{all paths } \gamma} \mathcal{A}[\gamma] |Final_\gamma\rangle
\end{equation}

其中 $\mathcal{A}[\gamma]$ 是路径 $\gamma$ 的复数振幅,由作用量 $e^{iS}$ 决定。

在这个图景中:

\begin{enumerate}
\item \textbf{多重历史并存}:所有符合物理定律的历史路径(例如"猫死了"和"猫活着")都在波函数中拥有非零的振幅。它们是并行存在的实在。

\item \textbf{没有坍缩}:由于系统是封闭的,不存在外部观测者来执行测量,因此波函数永远不会坍缩。薛定谔的猫永远处于生死叠加态。

\item \textbf{静态时空}:时间参数 $t$ 仅仅是希尔伯特空间中的一个索引。整个历史结构 $(|\Psi(0)\rangle, |\Psi(1)\rangle, \dots, |\Psi(T)\rangle)$ 就像一块已经完成的晶体,静态地悬浮在逻辑空间中。
\end{enumerate}

\subsection{主观体验的缺失}

QTM 模型虽然在数学上完美自洽,但它面临一个致命的解释鸿沟:\textbf{它无法推导出"现在"和"我"的概念。}

在一个包含所有可能性的波函数中,所有的时刻都是平权的,所有的历史分支都是平权的。不存在一个特殊的指针来标记"现在是 2025 年"或"我看到了猫活着"。

\begin{itemize}
\item \textbf{没有"现在"}:因为所有 $t$ 的状态都由 $\hat{U}$ 刚性连接,过去和未来在本体论上是等价的。

\item \textbf{没有"选择"}:因为所有分支都发生了,所谓的选择只是幻觉。一个在分叉路口向左走的人和一个向右走的人,都只是全局波函数中的不同分量。
\end{itemize}

这就引出了本书的核心问题:\textbf{为什么我们作为身处宇宙内部的观测者,体验到的不是并行的多重历史,而是单一的、线性的、充满随机性的时间流?}

为了回答这个问题,我们需要引入对偶的第二端点——\textbf{局域交互自动机模型(CITM)},这将在下一节详细阐述。
