\section{局域交互自动机模型}

\textbf{(The Local Interactive Automaton, CITM)}

\begin{quote}
\textbf{"在上帝的视角中,宇宙是一块静态的水晶;但在玩家的视角中,宇宙是一场动态的博弈。如果不引入一个受限于视界的、拥有输入接口的局域模型,物理学就永远无法解释'现在'的流动感,也无法安放观测者的自由意志。"}
\end{quote}

在上一节中,我们建立了全局幺正演化模型(QTM),它描述了一个决定论的、包含所有可能历史的块宇宙。然而,这个模型存在一个致命的缺陷:它无法描述\textbf{身处宇宙内部}的观测者的主观经验。对于我们——也就是被嵌入在这个宇宙中的有限智能体——而言,并没有看到死猫与活猫的叠加,也没有看到未来的剧本。我们看到的是一个不断坍缩的、单一的、充满不确定性的现实。

为了描述这种主观视角,我们需要引入对偶的计算模型:\textbf{经典交互式图灵机(Classical Interactive Turing Machine, CITM)}。本节将严格定义这一模型,并将物理学中的"波函数坍缩"重构为计算机科学中的"外部输入中断"。

\subsection{局域视界与经典状态空间}

首先,我们必须定义观测者的边界。在 QTM 模型中,观测者只是全域波函数 $|\Psi\rangle$ 的一部分。但在 CITM 模型中,观测者 $\mathcal{A}$ 被视为计算的主体。

由于\textbf{有限信息公理}和\textbf{光速限制}(详见第二卷),观测者在任意时刻 $t$ 只能访问宇宙的一个子区域,我们称之为\textbf{局域视界(Local Horizon, $\mathcal{H}_A$)}。

\begin{definition}[经典状态]
虽然视界内的物理本体是量子的(希尔伯特空间),但观测者只能通过特定的\textbf{"指针基底"(Pointer Basis)}来读取信息(例如视网膜上的光子撞击位置,或仪器上的读数)。因此,对于观测者而言,系统的有效状态不是向量 $|\psi\rangle$,而是一个\textbf{经典比特串(Classical Bit String)} $s_t$。

\begin{equation}
s_t \in \mathcal{S} = \{0, 1\}^K
\end{equation}

其中 $K$ 是观测者视界内的最大可区分比特数(受贝肯斯坦界限约束)。
\end{definition}

\textbf{物理意义}:

这一经典化过程对应于物理学中的\textbf{退相干(Decoherence)}。但在计算本体论中,我们将其解释为\textbf{数据类型的强制转换(Type Casting)}:系统为了适应观测者有限的 I/O 带宽,将高维的量子复数向量"压缩"为了低维的经典枚举值。

\subsection{预言机接口:自由意志的数学定义}

QTM 是封闭的,而 CITM 是\textbf{开放}的。这是一个本质的区别。

在经典计算理论中,如果一个图灵机在运行过程中遇到一个无法通过当前算法解决的问题(例如不可判定的命题,或需要真随机数的选择),它可以暂停并查询一个外部的黑箱,这个黑箱被称为\textbf{预言机(Oracle, $\mathcal{O}$)}。

\begin{definition}[物理预言机]
我们定义物理预言机 $\mathcal{O}$ 为一个连接观测者与视界之外(Environment/Exoverse)的\textbf{输入通道}。

\begin{equation}
\mathcal{O}: \mathcal{S} \times \mathbb{N} \to \Omega
\end{equation}

其中 $\Omega$ 是可能的观测结果集合(例如自旋向上/向下)。
\end{definition}

当 CITM 运行到\textbf{分支点(Branching Point)}——即量子力学预测出现叠加态的时刻——系统不会分裂,而是向 $\mathcal{O}$ 发起一次查询(Query)。

\begin{itemize}
\item \textbf{输入}:当前的叠加态系数(概率幅)。

\item \textbf{输出}:一个确定的经典结果(坍缩)。
\end{itemize}

\textbf{本体论推论}:

在 CITM 模型中,\textbf{"自由意志"和"量子随机性"是同义词}。它们都代表了\textbf{非算法的信息流(Non-algorithmic Information Flow)}从系统外部注入到局域系统内部。这不是物理定律的失效,而是系统层级的\textbf{I/O 通信}。

\subsection{交互式动力学:中断与跳转}

有了状态空间和预言机,我们可以写出 CITM 的演化方程。这不再是线性的薛定谔方程,而是一个\textbf{混合动力学系统(Hybrid Dynamical System)}。

系统状态 $s_t$ 的更新遵循以下规则:

\begin{equation}
s_{t+1} = 
\begin{cases} 
\mathcal{L}(s_t) & \text{if } \text{Query}_t = \text{False} \\
\text{Update}(s_t, \mathcal{O}(s_t)) & \text{if } \text{Query}_t = \text{True} 
\end{cases}
\end{equation}

\begin{enumerate}
\item \textbf{默认模式($\mathcal{L}$)}:当没有观测发生时,系统遵循\textbf{经典物理定律}(如牛顿力学或麦克斯韦方程组的离散化版本)。这是低成本的惯性演化,相当于计算机中的\textbf{后台挂起}或\textbf{线性外推}。

\item \textbf{交互模式($\mathcal{O}$)}:当观测者执行测量(Query)时,系统被\textbf{中断(Interrupt)}。预言机注入一个新的值(观测结果),强制系统状态发生非线性的\textbf{跳转(Jump)}。
\end{enumerate}

这完美对应了量子力学中的\textbf{冯·诺依曼假设}:

\begin{itemize}
\item \textbf{过程 II}(幺正演化)对应默认模式。

\item \textbf{过程 I}(波函数坍缩)对应交互模式。
\end{itemize}

在 CITM 视角下,坍缩不是物理上的"毁灭",而是数据的\textbf{写入(Write)}操作。

\subsection{懒加载与单一历史的生成}

QTM 模型计算了所有可能的历史,而 CITM 模型利用\textbf{懒加载(Lazy Evaluation)}机制,只计算被观测的那一条历史。

\begin{itemize}
\item \textbf{未观测时}:山川河流并不作为确定的像素点存在,它们只是作为生成规则(波函数/代码)存储在硬盘里。

\item \textbf{观测时}:只有当观测者的视线(查询)投向那里时,系统才调用预言机,\textbf{即时编译(JIT)}出具体的物理属性。
\end{itemize}

\begin{definition}[动态历史]
在 CITM 中,历史 $H_t = (s_0, s_1, \dots, s_t)$ 不是预先存在的静态数据,而是随着时间 $t$ 的推移,由一系列预言机输入所\textbf{铸造(Minted)}生成的\textbf{日志文件(Log File)}。
\end{definition}

这意味着:\textbf{未来不是被发现的,未来是被生成的。}

\subsection{总结:玩家视角的合法性}

CITM 模型为我们提供了一个符合直觉的物理图像:

\begin{enumerate}
\item \textbf{世界是经典的}(因为我们的接口是经典的)。

\item \textbf{未来是开放的}(因为预言机不断注入新信息)。

\item \textbf{资源是有限的}(因为我们只计算单一历史)。
\end{enumerate}

现在的核心问题是:这个看起来像是"为了省流"而设计的简陋模型(CITM),真的能等价于那个宏大、完美的上帝模型(QTM)吗?

下一节,我们将证明全书最重要的\textbf{斯泰恩斯普林-图灵同构定理},它将从数学上严格证明:\textbf{对于局域观测者而言,这两个模型在统计上是完全不可区分的。}
