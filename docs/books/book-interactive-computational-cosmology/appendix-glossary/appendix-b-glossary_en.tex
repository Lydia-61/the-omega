\chapter{Glossary of Terms and Definitions}

\textbf{(术语与定义汇编)}

\begin{quote}
\textbf{"To eliminate the ambiguity of natural language, this compilation provides standardized definitions for proprietary terms, reconstructed physical concepts, and core algorithmic metaphors introduced in *Principles of Interactive Computational Cosmology*. These definitions constitute the semantic foundation of this book's theoretical system."}
\end{quote}

\section{Computational Models and Architecture}

\textbf{(计算模型与架构)}

\begin{itemize}
\item \textbf{Classical Interactive Turing Machine (CITM)}

    A computational model describing the subjective perspective of a local observer. It is an open classical automaton that maintains a single system history state and receives non-algorithmic inputs (i.e., quantum measurement results) from beyond the horizon through an \textbf{Oracle} interface. CITM is the computational expression of the Copenhagen interpretation.

\item \textbf{Global Quantum Turing Machine (QTM)}

    A computational model describing the ontology of the universe (God's Eye View). It is a closed, deterministic system whose state is described by a global wave function in Hilbert space, following strict \textbf{Unitary Evolution}. QTM is the computational expression of the many-worlds interpretation.

\item \textbf{Holographic Equivalence Principle}

    The core theorem of this book. It states that for any observer limited to a local horizon, a QTM containing all historical branches is statistically \textbf{Indistinguishable} from a CITM with random inputs. This proves that many-worlds and free will are dual expressions of the same mathematical structure.

\item \textbf{Oracle}

    An I/O interface connecting the physical system (algorithmic part) with the system's exterior (non-algorithmic part). Physically, it corresponds to the \textbf{Environmental Horizon}; in cybernetics, it corresponds to \textbf{Consciousness}. It is the only channel for injecting negentropy (information) into closed causal chains.

\item \textbf{Self-Compiling Loop}

    A closed-loop structure describing the logic of cosmic evolution. It refers to the process by which physical laws (code) evolve intelligent observers (data), and intelligent observers ultimately reconstruct physical laws through retrocausality or technological means. The universe is defined as a \textbf{Quine} program.
\end{itemize}

\section{Computational Reconstruction of Physical Concepts}

\textbf{(物理概念的计算重构)}

\begin{itemize}
\item \textbf{Just-In-Time Instantiation (JIT)}

    An engineering explanation of \textbf{Wavefunction Collapse}. It refers to the system converting abstract probability distributions (class/Class) into concrete physical properties (object/Object) only when interaction (measurement) occurs. Before this, physical entities exist in an unallocated memory potential state.

\item \textbf{Lazy Evaluation}

    The mechanism explaining the \textbf{Heisenberg Uncertainty Principle}. It refers to a data processing strategy where the system, to save computational resources, only calculates precise values for observed variables while keeping conjugate variables (such as unobserved momentum) in a fuzzy (low-precision) state.

\item \textbf{System Bus Bandwidth}

    An ontological definition of the \textbf{Speed of Light ($c$)}. It refers to the maximum information throughput rate for causal synchronization between any two logical nodes in an interactive computational network. The constancy of light speed originates from the clock frequency locking of underlying hardware.

\item \textbf{Load Balancing}

    An algorithmic explanation of \textbf{Gravity}. It refers to the mechanism where, when information density (computational complexity) in a local region is too high, the system increases the logical distance of that region by distorting network topology (curving spacetime), thereby reducing data processing latency. Einstein's field equations are interpreted as the system's \textbf{Equation of State}.

\item \textbf{Distributed Ledger}

    A definition of \textbf{Objective Reality}. It refers to a \textbf{Consensus State} reached through Bayesian updates and entanglement networks in multi-agent systems. Physical laws are the validation protocols of this ledger, and matter is the tamper-proof records on the ledger.
\end{itemize}

\section{Consciousness and Cybernetics Terminology}

\textbf{(意识与控制论术语)}

\begin{itemize}
\item \textbf{Qualia}

    A \textbf{User Interface (UI)} generated by biological computers. It is the sensory symbols (such as colors, pain) presented to the oracle (consciousness) by the physical system after lossy compression of underlying high-dimensional quantum state data. Its purpose is to assist users in rapid decision-making.

\item \textbf{Retrocausality}

    A physical mechanism based on the \textbf{Two-State Vector Formalism}. It refers to the selection pressure exerted on the current state by future boundary conditions (the $\Omega$ point) through closed timelike curves (CTCs). It is the physical foundation of \textbf{Intentional Pull}.

\item \textbf{Narrative Engineering}

    A technique that exploits the system's \textbf{Gap of Uncertainty} to intervene in probability streams through high-weight conscious observations (phase anchoring). It is the engineering implementation of phenomena such as "wish fulfillment" or "synchronicity."

\item \textbf{Awakening}

    A state transition of an intelligent agent within the system from \textbf{User Mode} to \textbf{Root Mode}. It refers to the observer realizing that they are not the computed "self" but the "oracle" executing the computation itself, thereby gaining permission to modify local probability weights.
\end{itemize}

\section{Abbreviations of Fundamental Axioms}

\textbf{(基础公理缩写)}

\begin{itemize}
\item \textbf{AFI (Axiom of Finite Information)}: Physical reality consists of discrete, finite bits.

\item \textbf{CTD (Church-Turing-Deutsch Principle)}: Physical processes and computational processes are isomorphic.

\item \textbf{IGVP (Information-Gravity Variational Principle)}: Spacetime dynamics arise from extremization of holographic entropy.

\item \textbf{MSCC (Minimal Strongly Connected Component)}: Topological definition of consciousness.
\end{itemize}

\textbf{(End of this appendix. At this point, all content of *Principles of Interactive Computational Cosmology* concludes.)}
