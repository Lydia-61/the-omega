\section{Part IV: Rupture --- Singularities and Firewalls}

In the first three volumes, we depicted a spacetime woven from tensor networks, protected by error-correcting codes, and manifesting as a quantum superfluid. This sounds like a perfect, self-repairing system.

However, any physical material has its \textbf{Yield Strength}. Any network has its \textbf{Max Bandwidth}.

What happens when we stuff information (mass) exceeding the carrying limit of the spacetime fabric into an extremely small region?

The network will \textbf{congest}. Connections will \textbf{saturate}. Eventually, spacetime itself will \textbf{Rupture}.

This rupture is called a \textbf{Black Hole} in macroscopic physics.

This volume will reveal that black holes are not monsters; they are \textbf{traffic control valves} of the cosmic network. And that fearsome singularity is the \textbf{``blue screen of death''} thrown by physical laws when encountering uncomputable logical errors.

