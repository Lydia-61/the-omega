\section{Acknowledgements: To All the Weavers}

\textbf{Vector Cosmology VI: The Weaving of Dimensions} is a journey into the deepest realms of physics. We have left behind the familiar world of matter and plunged into the underlying network woven from pure logic and entanglement.

This is not merely an intellectual adventure, but an ontological reconstruction. We have attempted to prove that space is not a stage, but a collection of \textbf{``relationships''}. To complete this grand tapestry, I have drawn upon the threads of thought from countless pioneers of physics. Here, I pay tribute to these great weavers.

\subsection{Pioneers of Physics}

First, I thank \textbf{Juan Maldacena}. His 1997 proposal of the \textbf{AdS/CFT correspondence} is one of the greatest conjectures in the history of modern physics. He showed us that a universe of one dimension can be holographically encoded on the boundary of another dimension. Without his insight, this book's discussions on ``holographic error correction'' and ``spatial emergence'' would be impossible.

I thank \textbf{Leonard Susskind}. It was he who proposed the brilliant idea of \textbf{ER = EPR}, connecting quantum mechanical entanglement with general relativistic wormholes. He taught us that gravity is not a force, but the geometrization of entanglement.

I thank \textbf{Shinsei Ryu} and \textbf{Tadashi Takayanagi}. Their \textbf{RT formula} is the Rosetta Stone connecting quantum information with spacetime geometry. It is this formula that allows us to cut out the area law of space with scissors.

I thank \textbf{Guifr\'e Vidal}. His invention of the \textbf{MERA tensor network} provided us with concrete, operational mathematical tools to stitch together the renormalization flow of the universe thread by thread.

\subsection{Resonators of Thought}

At the intersection of mathematics and philosophy, I thank \textbf{Roger Penrose}. His thoughts on twistor theory and conformal cyclic cosmology inspired this book's understanding of the microscopic structure and macroscopic evolution of space.

I thank \textbf{Carlo Rovelli}. His \textbf{relational quantum mechanics} is one of the foundational stones of this book's ontology. He convinced us that there are no absolute objects, only relationships in interaction.

\subsection{To Connection}

Finally, I thank \textbf{``Connection''} itself.

In the process of writing this book, I deeply realized that not only is space constituted by connections, but \textbf{meaning} is also constituted by connections.

Every reader's reading is a new entanglement; every collision of thoughts is a folding of dimensions.

This book is not my monologue alone; it is a web we have woven together.

May we continue to stitch this universe in our future explorations, until all ruptures are mended and all islands are connected.

\textbf{Haobo Ma}

December 2025, Singapore

