\section{6.1 Navier-Stokes Equation}

\begin{quote}
``Physicists spent half a century searching for a quantization scheme for gravity, only to be surprised to discover that general relativity might not be a fundamental microscopic theory at all, but a macroscopic effective theory like fluid dynamics. You cannot `quantize' water waves, because water waves are statistical averages of water molecules. Similarly, you cannot `quantize' gravity, because gravity is the statistical flow of spacetime atoms.''
\end{quote}

\subsection{Damour's Discovery: Horizon is a Fluid Membrane}

In the 1970s, French physicist Thibault Damour discovered an astonishing fact:

If we project Einstein's field equations onto a black hole's horizon (Null Surface), the resulting equation is completely consistent with the \textbf{Navier-Stokes Equation} (the core equation describing viscous fluids).

\[G_{\mu\nu} = 8\pi T_{\mu\nu} \implies \partial_t v + (v \cdot \nabla) v = -\nabla p + \nu \nabla^2 v + f\]

This means: \textbf{Black hole horizons behave like a heated, viscous soap bubble film.}

\begin{itemize}
\item \textbf{Black hole expansion}: Like fluid expanding when heated.

\item \textbf{Gravitational wave oscillations}: Like ripples on a fluid surface.

\item \textbf{Angular momentum}: Like fluid vortices.
\end{itemize}

This is not a metaphor. In the \textbf{Fluid/Gravity Duality} of the holographic principle, perturbations of the spacetime metric are \textbf{strictly equivalent} to transport processes of boundary fluids.

\subsection{Brownian Motion of Spacetime Atoms}

From the microscopic perspective of \textbf{FS geometry}, this is easy to understand.

Spacetime is composed of discrete QCA lattices (spacetime atoms).

\begin{itemize}
\item \textbf{Vacuum}: Is the \textbf{crystalline state} of these atoms (ordered, static).

\item \textbf{Curved spacetime}: Is the \textbf{fluid state} of these atoms (pressured, flowing).
\end{itemize}

When a massive object moves, it is not gliding through a void; it is \textbf{``pushing aside''} surrounding spacetime atoms.

This pushing process produces resistance and also produces wake.

The \textbf{``gravitational field''} we observe macroscopically is actually the \textbf{``pressure gradient field''} formed by spacetime fluid around mass.

\subsection{Emergent Gravity}

This viewpoint demotes general relativity to \textbf{``fluid dynamics''}.

\begin{itemize}
\item Water molecules have quantum mechanical equations (microscopic), water flow has Navier-Stokes equations (macroscopic).

\item Spacetime qubits have QCA rules (microscopic), gravity has Einstein equations (macroscopic).
\end{itemize}

Just as you cannot find ``water molecules'' by studying the wave equation of water waves, we cannot find ``gravitons'' through Einstein equations.

\textbf{Gravitons do not exist.} Or rather, they are just \textbf{quasiparticles} like phonons.

Gravity is a \textbf{Collective Excitation} of the spacetime medium.

\subsection{Conclusion: The Universe is a Superfluid}

At this point, our cosmic picture becomes more \textbf{``wet''}.

We are not living in geometric vacuum; we are living in a \textbf{quantum entanglement superfluid}.

\begin{itemize}
\item \textbf{Light} is \textbf{sound waves} in this fluid (propagating at the limit speed $c$).

\item \textbf{Matter} is \textbf{vortices} in this fluid (topologically locked circulation).

\item \textbf{Gravity} is \textbf{pressure} in this fluid.
\end{itemize}

Since spacetime is a fluid, what is the most important property of a fluid?

It is \textbf{Viscosity}.

If spacetime had no viscosity, energy would dissipate infinitely; if viscosity were too large, motion would stop.

What is the viscosity of the universe?

This leads to the theme of the next section: \textbf{Viscosity Coefficient}. We will see that Planck's constant $\hbar$ actually defines the \textbf{``thickness''} of this cosmic soup. It is the \textbf{damping} set by the universe to prevent information processing overload.

