\section{1.1 The Stitching of Bell Pairs}

\begin{quote}
``Two particles can sense each other instantly even across light-years, not because signals travel too fast, but because in the deeper geometric structure, they were never separated. Entanglement is not a spooky action at a distance; entanglement is the `glue' of space. It is countless such glue molecules that bond discrete pixels into a continuous universe.''
\end{quote}

\subsection{The Smallest Unit of Space}

In classical geometry, the connection between points is axiomatically given (e.g., there is a straight line between two points). But at the QCA bottom layer of \textbf{Vector Cosmology}, points (qubits) are independent.

If nothing connects them, the universe is a pile of sand, without even the concept of ``adjacency.''

The smallest unit that connects this pile of sand is the \textbf{maximally entangled state}, also known as the \textbf{Bell Pair}:

\[|\Phi^+\rangle = \frac{1}{\sqrt{2}} (|00\rangle + |11\rangle)\]

Look carefully at this formula.

\begin{itemize}
\item It is not $|00\rangle$ (both in state 0).

\item It is not $|11\rangle$ (both in state 1).

\item It is the \textbf{coherent superposition} of both. This means that particle A's state and particle B's state are locked together tightly. If you measure A and get 0, B \textbf{must} be 0.
\end{itemize}

\textbf{Geometrically, this is a ``wormhole.''}

Although these two particles may be far apart in physical memory addresses, in logical space (Hilbert space), they share the same wave function. Not only are they connected; they are topologically \textbf{``Stitched''} together.

\subsection{Stitching the Void: Short-Range Entanglement Weaving the Web}

How is our three-dimensional space formed?

It is achieved by filling the vacuum with dense \textbf{Short-range Entanglement}.

Imagine a lattice grid.

\begin{itemize}
\item Each lattice point shares numerous Bell pairs with its left, right, up, and down neighbors.

\item This high-density local entanglement, like stitches, ``sews'' adjacent pixels together.
\end{itemize}

\textbf{Why do you feel space is continuous?}

Because these stitches are too dense.

When you walk through a room, your body is not gliding through a vacuum, but constantly performing \textbf{``Entanglement Swapping''} with Bell pairs in the background field.

With each step, you are untying old threads and sewing new ones.

This microscopic stitching mechanism ensures the \textbf{Connectivity} of space. If entanglement suddenly disappears in some region, it won't become a vacuum; it will become a \textbf{cliff of spacetime}. You will fall or be bounced back when you reach there, because there is no ``road'' (geometric connection) there.

\subsection{The Definition of Distance: The Measure of Mutual Information}

This gives us a revolutionary geometric definition: \textbf{Distance originates from correlation}.

In FS geometry, the \textbf{geometric distance $d(A, B)$} between two regions $A$ and $B$ is inversely proportional to their \textbf{Mutual Information ($I$)}:

\[d(A, B) \sim -\ln I(A : B)\]

\begin{itemize}
\item \textbf{Strong entanglement ($I$ large)}: $\ln I$ large, distance $d$ small. This is why atomic nuclei are extremely dense, because the entanglement between quarks is extremely strong.

\item \textbf{Weak entanglement ($I$ small)}: $\ln I$ small, distance $d$ large. This is why galaxies are far apart, because gravity (long-range entanglement) is much sparser than electromagnetic force (short-range entanglement).
\end{itemize}

\textbf{There is no absolute ``far''.}

``Far'' only means the threads connecting you are stretched very thin and very long.

If you could artificially increase the entanglement between yourself and a star in Andromeda (inject computational power, create Bell pairs), then geometrically, you would find that star \textbf{``flying toward you''} until it touches your nose.

\subsection{Conclusion: Space is Emergent Glue}

This section completely subverts Newton's absolute space view.

\textbf{Space is not a stage; space is glue.}

It is a huge adhesive woven from $10^{120}$ Bell pairs.

\begin{itemize}
\item When we say ``the universe expands,'' we actually mean this glue is being diluted (entanglement density decreases).

\item When we say ``black hole horizon,'' we actually mean the glue there is compressed to the limit, forming an unbreakable hard knot.
\end{itemize}

Since space is stitched by entanglement, if we want to calculate the ``area'' of a piece of space, how should we calculate it?

We don't need a ruler. We need a pair of \textbf{scissors}.

Just count how many threads need to be cut to slice through this piece of space, and we know its size.

This leads to the theme of the next section: \textbf{The Derivation of the Area Law}. We will see that the famous \textbf{black hole entropy formula ($S=A/4$)} is actually a count of the ``number of thread ends'' on the cross-section of the cosmic fabric.

