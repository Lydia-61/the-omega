\section{Appendix A: Geometric Derivation of Holographic Entanglement Entropy}

In Volume I ``The Loom'' of \textbf{Vector Cosmology VI}, we proposed a core viewpoint: \textbf{Space is woven from entanglement}. In particular, we cited the famous \textbf{Ryu-Takayanagi Formula}, stating that the entanglement entropy $S_A$ of a region equals the area $\text{Area}(\gamma_A)$ of the minimal surface in its holographic dual space divided by $4G$.

This appendix will provide the mathematical derivation framework behind this holy grail of physics. We will show how, starting from pure quantum information theory (von Neumann entropy), through the \textbf{Replica Trick} and \textbf{path integrals}, we naturally derive the geometric area law in general relativity.

This is mathematical ironclad evidence that \textbf{``geometry emerges from information''}.

\subsection{A.1 Von Neumann Entropy and the Replica Trick}

First, we define the entropy of a quantum system. For a density matrix $\rho_A$ in a mixed state, its \textbf{von Neumann entropy} is defined as:

\[S_A = -\text{Tr}(\rho_A \ln \rho_A)\]

Directly calculating $\ln \rho_A$ is very difficult (involving matrix logarithms). To solve this problem, physicists introduced the \textbf{Replica Trick}.

Instead of directly calculating the logarithm, we calculate the trace of the $n$-th power of $\rho_A$ (i.e., Rényi entropy), then take the derivative with respect to $n$:

\[S_A = -\lim_{n \to 1} \frac{\partial}{\partial n} \text{Tr}(\rho_A^n)\]

\textbf{Physical Picture:}

\begin{itemize}
\item \textbf{$\text{Tr}(\rho_A^n)$} means we \textbf{``stitch''} $n$ identical copies of the universe (Replicas) together at the boundary of region A.

\item This stitching creates a \textbf{``Conical Singularity''} in spacetime geometry.
\end{itemize}

\subsection{A.2 Geometric Proof of the Ryu-Takayanagi Formula}

Under the framework of \textbf{AdS/CFT duality}, the partition function $Z_{CFT}$ of the boundary field theory (CFT) is equivalent to the partition function $Z_{gravity}$ of the bulk gravitational theory (AdS).

\[Z_{CFT} \approx e^{-S_{gravity}}\]

When we calculate $\text{Tr}(\rho_A^n)$, we introduce a \textbf{Twist Operator} on the boundary.

In the bulk of the holographic dual, this boundary twist condition extends inward, forming a \textbf{``Minimal Surface''} $\gamma_A$.

According to the action principle of general relativity, the system always tends to find configurations with minimum energy (or area).

When we stitch $n$ copies together on the boundary, the spacetime geometry in the bulk responds, trying to connect these $n$ layers at minimum cost.

Mathematical derivation shows that in the limit $n \to 1$, the dominant term of entropy is exactly proportional to the area of this minimal surface:

\[S_A = \frac{\text{Area}(\gamma_A)}{4 G_N}\]

\begin{itemize}
\item \textbf{$\text{Area}(\gamma_A)$}: Is the area of the minimal surface hanging from the boundary of region A in the higher-dimensional bulk space, like a soap film.

\item \textbf{$G_N$}: Is Newton's gravitational constant.

\item \textbf{$4$}: Is a geometric factor.
\end{itemize}

\textbf{Conclusion:}

This formula proves that \textbf{quantum entanglement (entropy)} is not an abstract number; it is \textbf{real spacetime geometric area}.

Each bit of entanglement occupies a spacetime cross-section of size $4G_N$. This is the microscopic fiber density of the \textbf{``spacetime fabric''}.

\subsection{A.3 MERA Networks and Discrete AdS Space}

The Ryu-Takayanagi formula is a result in the continuous limit. In the microscopic QCA model of \textbf{Vector Cosmology}, we focus more on discrete structures.

This involves \textbf{MERA (Multiscale Entanglement Renormalization Ansatz)} tensor networks.

MERA networks construct a fractal tree structure through layered stacking of \textbf{Disentanglers} and \textbf{Isometries}.

If we calculate the entanglement entropy between two regions in a MERA network, we need to cut the connection bonds in the network.

\textbf{Theorem:}

In MERA networks, the number of \textbf{Minimum Cuts} connecting region A with its complement, in the macroscopic limit, precisely corresponds to the \textbf{geodesic length} (for 1+1 dimensional boundaries) or \textbf{minimal surface area} (for higher-dimensional boundaries) in AdS space.

\[S_{MERA}(A) \sim \min \#(\text{Cut Bonds}) \sim \text{Area}(\gamma_A)\]

This not only verifies the holographic principle but also reveals the \textbf{mechanism of dimensional emergence}:

\begin{itemize}
\item \textbf{Horizontal connections}: Constitute the breadth of physical space.

\item \textbf{Longitudinal layers (renormalization steps)}: Constitute the additional \textbf{holographic dimension} (AdS radial direction).
\end{itemize}

\textbf{The ``depth'' ($z$-axis) we perceive is essentially the ``logical steps'' from leaves (microscopic) to roots (macroscopic) in the MERA network.}

