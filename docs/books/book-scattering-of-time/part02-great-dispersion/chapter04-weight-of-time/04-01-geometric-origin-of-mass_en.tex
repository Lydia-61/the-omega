\chapter{The Weight of Time}

\section{The Geometric Origin of Mass}

We usually think that ``mass'' is the most essential property of objects. A stone is heavy because it is filled with ``matter''; an electron has mass because it is a real particle. In our intuition, mass represents the ``weight of existence.''

But in our geometric reconstruction, this intuition is wrong.

Let us return once more to that Pythagorean theorem ruling the universe: $v_{ext}^2 + v_{int}^2 = c^2$.

In the previous chapter, we saw that photons choose $v_{ext} = c$ and $v_{int} = 0$. They spend all their budget on external spatial displacement, so they have no internal evolution and no mass. They are pure ``traveling.''

So, what are massive objects (such as electrons, quarks, or you)?

Massive objects are entities that \textbf{decide to stop} (or run slower) and \textbf{spend their budget internally}.

\subsection{Mass is Imprisoned Time}

When an object is at rest in space ($v_{ext} = 0$), according to the Pythagorean theorem, it must evolve internally at full speed $c$ ($v_{int} = c$). This means its state vector is rotating frantically in the internal dimensions of Hilbert Space.

This frantic internal rotation is what we feel as ``mass.''

In our dictionary, \textbf{mass} ($m$) is not the amount of matter; it is the \textbf{rate of internal information processing} ($v_{int}$).

Imagine a spinning top. From a distance, it appears stationary on the table, occupying a fixed position. But if you look closely, you'll find it full of kinetic energy, spinning at high speed. Matter is the same. A stationary electron is not truly ``stationary''; it is just motionless in space, but in the dimension of time, it is vibrating at extremely high frequency.

This is why Einstein's mass-energy equation $E=mc^2$ is so precise. This equation actually says: energy ($E$) equals mass ($m$) times a constant. From our geometric perspective, this is not just numerical equality; it is \textbf{ontological identity}.

\begin{itemize}
\item Energy $E$ corresponds to total evolution rate (total bandwidth).

\item Mass $m$ corresponds to internal evolution rate (internal bandwidth).

\item In the rest frame, all total bandwidth is converted to internal bandwidth.
\end{itemize}

Therefore, \textbf{mass is imprisoned time}. It is the rate $c$ that the universe should have used to race through space, curled and knotted, locked in a tiny range, becoming an endless internal cycle.

\subsection{The Frequency of Existence}

Quantum mechanics tells us through de Broglie's relation ($mc^2 = \hbar\omega$) that every massive particle corresponds to a specific frequency $\omega$. In traditional physics, this is seen as a mysterious manifestation of wave-particle duality.

But in our framework, this becomes very intuitive: \textbf{frequency is the speed of rotation.}

Greater mass means the object refreshes its own state at this frequency faster internally.

\begin{itemize}
\item An electron is light; its internal clock ticks relatively slowly.

\item A top quark is heavy; its internal clock roars frantically.

\item A black hole is extremely heavy; it is an extremely high-density oscillating knot in spacetime structure.
\end{itemize}

This also explains why we have a ``sense of existence.'' Photons have no sense of existence; they are just fleeting shadows. Matter has a sense of existence because it rewrites itself countless times in Hilbert Space every instant. \textbf{We are heavy because we are very busy at the microscopic level}

We are composed of countless high-speed micro-clocks. This is why we cannot move at light speed---because we are too busy internally, our bandwidth is occupied by life, atomic structures, and the maintenance of existence, leaving no remaining budget to purchase a light-speed ticket.

This ``internal busyness'' brings an inevitable consequence: when you try to push it, it resists. This resistance is another great mystery in physics---\textbf{inertia}.

---

\textit{(Next, we will enter section 4.2 ``The Essence of Inertia,'' to see how this internal rotation transforms into the resistance we feel.)}

