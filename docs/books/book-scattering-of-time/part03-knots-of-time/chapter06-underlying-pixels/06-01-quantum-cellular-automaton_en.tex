\part{The Knots of Time}

\chapter{The Underlying Pixels}

\section{Quantum Cellular Automaton (QCA)}

When you sit in a movie theater, watching smooth images on the screen: cars racing, clouds drifting, protagonists crying. Your eyes tell you everything is continuous and smooth. But if you walk up to the screen and observe with a magnifying glass, you'll find that ``smoothness'' is just an illusion. You see only \textbf{pixels} of three colors: red, green, blue. These pixels themselves don't move; they just change brightness in place according to specific rules.

Our universe is likely the same.

In Part II, we reconstructed relativity using smooth geometric language. We talked about ``flowing rates,'' ``rotating vectors.'' This is perfect for describing the macroscopic world. But when we try to apply this language to extremely tiny scales---Planck scale ($10^{-35}$ meters)---the smoothness of geometry collapses.

This is like trying to draw a perfect circle on a computer screen. No matter how high your resolution, if you zoom in enough, that circle's edge will always become jagged steps.

In our geometric reconstruction framework, this ``jaggedness'' is not error; it is the universe's \textbf{ontological truth}.

\subsection{The Universe's Refresh Rate}

In Chapter 1, we introduced Axiom A1: the universe evolves at a constant rate. Macroscopically, this manifests as continuous flow; but microscopically, we need to introduce a new concept: \textbf{Quantum Cellular Automaton (QCA)}.

Don't be intimidated by this complex term. Its core idea is very simple: \textbf{The universe is composed of countless tiny, discrete ``logic units.''}

Imagine space is not an empty box, but a vast three-dimensional grid. At every grid point there is a tiny quantum system---we can think of it as a ``qubit'' or a ``miniature Hilbert Space.''

This is like the universe's pixels.

This completely changes our understanding of ``motion.'' In classical physics, when an electron moves from point A to point B, we imagine it sliding through intermediate space like a marble. But in the QCA picture, the electron doesn't move. What actually happens is grid point A ``dimmed'' (lost the electron's state), while adjacent grid point B ``brightened'' (gained the electron's state).

So-called motion is actually \textbf{information transmission}.

This explains that mysterious ``evolution rate $c$'' we mentioned in Chapter 2. In the QCA model, time does not flow continuously, but jumps frame by frame. The universe has a fundamental ``clock tick'' (Tick). In every clock tick, every grid point updates its state according to its neighbors' states through a fixed rule (unitary operator).

Therefore, the universe is not ``evolving,'' but \textbf{computing}.

We not only view physics as geometric projection, but further, as a kind of \textbf{computational projection} (Computational Projection). From this perspective, the continuous rotation of Hilbert Space is actually the statistical average of countless tiny logic gate operations at the macroscopic level.

\subsection{The Ghost of Discreteness}

You might ask: ``If the universe is really pixelated, why can't I see the grid? Why don't I feel the world `stuttering'?''

The answer lies in \textbf{scale}.

The pixel density at this layer is astonishing. According to estimates, one cubic meter of vacuum contains approximately $10^{105}$ Planck grids. For comparison, humanity's sharpest displays have only hundreds of pixels per inch.

Because pixels are too small, update frequency too fast ($10^{43}$ times per second), our senses---even our most precise particle colliders---cannot detect the underlying granularity. The ``smooth spacetime'' we see is actually a \textbf{low-resolution approximation} emerging from underlying discrete structures.

Just as water appears as continuous fluid but is essentially discrete water molecules; spacetime appears as a continuous stage but is essentially a discrete qubit network.

Acknowledging the universe's discreteness (QCA nature) solves a problem that has plagued physics for years: \textbf{the elimination of infinities}.

In standard quantum field theory, when we calculate interactions between two particles approaching infinitely close, we often get ``infinity'' results. This is because we assume space can be infinitely divided. But in a QCA universe, you cannot approach infinitely close. Like on a screen, two bright points can only be adjacent at closest, not overlapping. This natural \textbf{geometric cutoff} (Cutoff) makes all physical calculations finite and reasonable.

So when we say ``underlying pixels,'' we are not making a metaphor. We are describing a reality more fundamental than ``strings'' or ``membranes'': \textbf{the minimal unit of information processing}.

But if the universe is composed of fixed grids, a huge problem follows: Since grids are stationary, why does light speed appear the same in all directions? Shouldn't walking diagonally and horizontally on the grid have different distances?

This is the famous ``Lorentz symmetry breaking'' problem. But in the next section we will see that QCA has a magical ability to perfectly disguise itself as isotropic continuous space at the macroscopic level, leaving only extremely tiny traces.

---

\textit{(Next, we will enter section 6.2 ``Causality as Network Speed,'' exploring how the speed-of-light limit naturally emerges as a logical necessity (Lieb-Robinson bound) in this pixelated universe.)}

