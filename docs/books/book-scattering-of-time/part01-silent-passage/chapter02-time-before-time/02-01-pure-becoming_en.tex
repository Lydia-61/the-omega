\chapter{Time Before Time}

\section{Pure Becoming}

In a universe where even space has not yet been ``created,'' talking about time seems absurd.

The time we experience in daily life is always bound to change: the clock hand moves one tick, the sun sets behind the mountain, or a hot cup of coffee cools down. For us, time is a container of events, the order in which all things happen. If the universe were empty, with no events occurring, would time still exist?

Isaac Newton would tell you: ``Yes.'' He believed in an absolute, divine clock that would still tick even if the universe were empty. But Einstein would retort: ``No.'' In relativity, time is a function of matter and motion; without clocks, there is no time.

But in our Hilbert Space picture, the truth lies between these two, yet deeper than both.

Let us return to that lonely universe state vector suspended in mathematical void. According to Axiom A1 we proposed in the previous chapter, this vector is ``rotating.''

Note that this ``rotation'' is not a displacement in space like the Earth orbiting the Sun. Because at this moment, there is no ``space,'' nor is there a ``Sun.'' This is a \textbf{pure internal change}. Mathematically, this is called a phase change (Phase Change).

To understand this, imagine listening to a continuous single tone. This sound has no melody, no volume changes; it is just an eternal ``hum---''. In this sound, no physical object is moving, but the sound itself is \textbf{persisting}. It is not just ``Being''; it is constantly ``Becoming.'' It is constantly updating itself, even though it appears identical.

This is what the universe looks like at its deepest level. It is not a static sculpture, but a flowing potential.

At this stage, time has not yet split into ``past'' and ``future.'' It is just a pure \textbf{update rate}. This update rate is the constant $c$ we defined in the axiom.

The speed of light $c$ we learn in physics textbooks is usually described as ``300,000 kilometers per second.'' This is a definition about \textbf{spatial traversal}. But in the afternoon of Hilbert Space, since there are no kilometers and no seconds, what exactly is this $c$?

It is the universe's \textbf{heartbeat}.

It is the \textbf{clock frequency} of the universe, this supercomputer processing information. Every instant, the universe's state vector rotates a tiny angle in Hilbert Space. The magnitude of this angle represents how much existence the universe has ``experienced.''

If we compare the universe to a game being downloaded, then before the game screen (physical world) appears, $c$ is the \textbf{bandwidth} of that background download process. This bandwidth is finite and constant. It determines how much change can occur in the universe per unit of meta-time.

This is ``time without time.'' It is not the result of measurement; it is an intrinsic property of existence.

The philosophical significance here is profound. Traditional physics often treats ``existence'' as a noun (Objects), but in our geometric reconstruction, ``existence'' is a verb (Process). The universe is not built from pre-made building blocks; the universe is woven from the \textbf{rate of evolution}.

Matter, as we will see later, is just this pure evolution rate knotted locally, creating an illusion of stillness. And energy is just an indicator of how fast this evolution rate is.

So when we say ``the speed of light is an insurmountable limit,'' we are actually saying: \textbf{As part of the universe, your existence rate cannot exceed the universe's overall refresh rate.} You cannot run faster than the creator.

This pure, colorless \textbf{stream of becoming} flowing at rate $c$ is the raw material of all things.

But this stream of light alone is extremely boring. If the universe were just a single tone humming at a constant frequency, there would never be galaxies, nor life. To transform this monotonous ``one'' into the rich ``many,'' this light must be broken. It must be observed, measured, \textbf{projected}.

This leads to a crucial step in our journey: How do we magically conjure the reality of here and now from this abstract mathematical heartbeat?

The answer lies in an ancient and mysterious geometric art---projection.

---

\textit{(Next, we will enter section 2.2 ``The Art of Projection,'' exploring how observers intervene in this light like a prism, and decompose the unified $c$ into the physical laws we are familiar with.)}

