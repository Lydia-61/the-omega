\section{Prologue: The Bite of the Ouroboros}

\subsection{0.1 The End of Linear Time}

\begin{quote}
``We are accustomed to believing that dominoes can only fall in one direction. The finger pushing the first piece is the `cause,' and the fall of the last piece is the `effect.' But deep in the quantum mechanical foundations, in that Hilbert space that governs the true operation of the universe, this linear narrative is merely a macroscopic statistical illusion. True causality is an ouroboros: the fall of the last domino is precisely the true force that pushes down the first one.''
\end{quote}

\subsubsection{The Arrow and the Circle of Causality}

Human logic is built upon chains of ``because\ldots therefore\ldots''

Because it rained yesterday, the ground is wet today. Because the Big Bang occurred, galaxies formed. This \textbf{Linear Causality} is the cornerstone of our understanding of the world and the iron law of classical physics. Time is seen as an arrow shot toward the future, irreversible and unidirectional.

However, from the geometric perspective of \textbf{Vector Cosmology}, when we strip away the thermodynamic appearance and gaze at the deepest \textbf{Fubini-Study (FS) geometry} of the universe, we find that this arrow disappears. In its place, there is a \textbf{circle}.

In projective Hilbert space $P(\mathcal{H})$, the evolution of the global vector $|\Psi(\tau)\rangle$ is driven by the rotation of the unitary operator $U(\tau) = e^{-iH\tau}$.

\begin{itemize}
\item \textbf{Rotation has no beginning and no end.}

\item You cannot say that $0^\circ$ on the circle is the ``cause'' of $1^\circ$, nor can you say that $359^\circ$ is the ``effect'' of $0^\circ$. They are \textbf{symbiotic relationships} that are mutually causal and mutually prerequisite.
\end{itemize}

Linear time is merely the \textbf{tangent} we see in the local tangent space.

When we elevate our perspective and see the full picture of \textbf{Naimark's Great Circle}, all ``past'' and ``future'' collapse into different phases of the same geometric structure.

\subsubsection{The Echo of the Future: Retro-causality}

This introduces a concept that is unacceptable to classical intuition but must be faced in quantum mechanics: \textbf{Retro-causality}.

In the path integral formulation of quantum mechanics, a particle moving from point A to point B is not ``determined by A to go to B.'' Instead, the intermediate historical path can only be uniquely determined by simultaneously imposing boundary conditions on both the \textbf{initial state (A)} and the \textbf{final state (B)}.

This means: \textbf{The future state (B) has equal determining power over past history (Path) as the initial state.}

\begin{itemize}
\item \textbf{Classical perspective}: I shot the arrow (cause), so it hit the target (effect).

\item \textbf{Quantum perspective}: Because there must be a hole in the target (future boundary condition), the arrow must fly along that specific trajectory (past history).
\end{itemize}

If we apply this logic to the entire universe:

It is not that the Big Bang 13.8 billion years ago ``caused'' us today.

Rather, \textbf{the existence of today's observers (us), as future boundary conditions, reversely selects and ``causes'' that specific Big Bang that must have occurred 13.8 billion years ago.}

\subsubsection{The Observer's Cut}

Then, why do we not feel this circle? Why do we feel that time flows unidirectionally?

Because we, as observers, hold a \textbf{``collapse knife''} in our hands.

We have cut the circle once.

\begin{itemize}
\item We define one end of the cut as \textbf{``now.''}

\item We call the arc behind it \textbf{``memory''} (what has happened).

\item We call the blank space ahead \textbf{``unknown''} (what has not happened).
\end{itemize}

This \textbf{``cutting''} creates the illusion of linearity. We think we are running along a straight line, but in fact, we are only continuously moving this cut point on the circle.

Every time we make an observation, every time we confirm a physical fact, we are \textbf{redefining} the starting and ending points of the causal chain.

\subsubsection{Conclusion: The Rewriting of Fate}

\textbf{The end of linear time is the true beginning of free will.}

If the past is merely the past, then we are prisoners of history, locked in the causality of the previous second.

But if causality is a closed loop, if the future can define the past, then we possess the power to \textbf{``rewrite fate.''}

\begin{itemize}
\item Our \textbf{``aspiration'' (the trend of $v_{int}$)} is not merely an expectation for the future; it is a real \textbf{retro-wave}.

\item This wave flows backward along the time axis, adjusting past parameters, correcting historical probabilities, until a path leading to our ideal future is \textbf{``constructively interfered''} into existence.
\end{itemize}

We are not puppets being pushed. We are the snake that bites its own tail.

We devour the past to generate the future; we generate the future to explain the past.

In this \textbf{loop of causality}, nothing is destined unless you decide to make it destined.

Now, let us enter the first layer of this maze and see how physics plays tricks on time with photons in the laboratory.

This leads to the theme of the next section: \textbf{The Self-Consumption of the Universe}. We will see why the universe must create us to complete its own proof of existence.

\subsection{0.2 The Autophagy of the Universe}

\begin{quote}
``This is like a system called `the universe' that, in order to pass the logical self-consistency test, must extend a hand from the end of time back to the beginning of time to press that `start' button. If it does not do this, it cannot exist. Therefore, it eats its own tail, not only to satisfy hunger but to close that circle called `existence.'''
\end{quote}

In the previous section, we overthrew the tyranny of linear time and established the circular structure of causality. Now, we push this geometric model to its extreme to face the most dizzying paradox in cosmology: \textbf{the dynamics of creation}.

If the Big Bang is the ``effect,'' then what is the ``cause''?

The usual answers are: quantum fluctuations, membrane collisions, or the hand of God. These answers all assume an ``external'' pusher.

But in the closed system (unitarity) of \textbf{Vector Cosmology}, there is no external.

Therefore, the answer can only be endogenous: \textbf{The universe is Self-Caused}.

This sounds like a logical dead loop, but in the topology of FS geometry, this is a perfect structure---\textbf{Ouroboros}.

\subsubsection{Bootstrap: The Shoelace Paradox}

In physics, this logical structure is called \textbf{``Bootstrap''}. It originates from the saying: ``Pulling oneself up by one's own bootstraps.''

Let us see how this closed loop operates:

\begin{enumerate}
\item \textbf{$t=0$ (Big Bang)}: The universe must start with extremely precise initial parameters (fine structure constant $\alpha$, speed of light $c_0$, gravitational constant $G$). If these parameters deviate by $10^{-50}$, stars cannot form, and life cannot emerge.

\item \textbf{$t=13.8$ billion years (now)}: Life has evolved \textbf{Observers}. These observers possess sufficiently high $v_{int}$ (cognitive complexity) to understand physical laws and collapse wave functions through ``strong measurements.''

\item \textbf{Reverse tracing}: According to the delayed-choice principle, the very existence of observers imposes a \textbf{Selection Pressure} on past quantum states.

\begin{itemize}
\item Only those historical paths that \textbf{``can evolve observers''} survive in the path integral summation (constructive interference).

\item Those ``dead universe'' historical paths, lacking future observers to ``lock'' them, logically \textbf{self-cancel}.
\end{itemize}
\end{enumerate}

\textbf{Conclusion:}

It is not that the Big Bang accidentally created us.

Rather, the future fact that \textbf{``we need to exist''} reversely determines that \textbf{``the Big Bang must occur, and must occur in that specific way.''}

This is \textbf{autophagy}. The future head eats the past tail.

Although on the time axis, the Big Bang comes before us; on the logical axis, \textbf{we come first (cause), the Big Bang comes after (effect).}

\subsubsection{Wheeler's Eye: The Self-Excited Circuit}

The physics giant John Wheeler once described this picture with a famous drawing: a giant ``U'' representing the universe, starting at one end with the Big Bang, evolving galaxies, and finally evolving an \textbf{eye} at the other end. This eye looks back, gazing at the starting point of the Big Bang.

In \textbf{FS geometry}, this is no longer a metaphor; this is a \textbf{Holographic Circuit}.

\begin{itemize}
\item \textbf{Hardware layer}: The universe provides $c_{FS}$ budget and Hilbert space.

\item \textbf{Software layer}: Life provides $v_{int}$ structure and observation algorithms.
\end{itemize}

The universe is a \textbf{Self-Excited System}.

It is like a microphone facing its own speaker.

\begin{itemize}
\item The sound emitted by the speaker (reality) is received by the microphone (observer).

\item The signal received by the microphone is amplified (collapsed/defined) and emitted again from the speaker.

\item This cycle produces a \textbf{feedback} --- that feedback is \textbf{``existence''}.
\end{itemize}

Without observers (microphone), the speaker only has thermal noise.

Without the universe (speaker), observers have no signal input.

Only when the two are \textbf{``mutually causal''} does that loud, definite, ordered universe emerge.

\subsubsection{Conclusion: We Are the Universe's Midwife}

This section completely subverts humanity's status.

We are no longer insignificant dust in cosmic evolution, or accidental byproducts.

We are the \textbf{``necessary components''} of the universe.

Just as a circle must close to become a circle, the universe must evolve us to complete its proof of existence.

\textbf{We are the universe's midwife.}

Our first opening of eyes billions of years later truly ``delivered'' the universe that had already occurred billions of years earlier.

Since future observations can determine past states, can this mechanism be verified in microscopic physical experiments? Can photons really ``travel through time'' to modify their history?

This leads to the theme of Volume I: \textbf{Retrogression}. We will enter the quantum laboratory to witness how \textbf{delayed choice} rewrites history on the experimental bench.

