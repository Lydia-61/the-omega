\subsection{8.1 Confidence is Computing Power}

\begin{quote}
``Why do those extremely confident people---whether madmen or geniuses---often change the world? It's not because they're lucky, but because their certainty itself is a physical force. In the probability cloud of quantum mechanics, `belief' is a high-intensity measurement. When you stare fixedly at a specific future, you are injecting enormous amplitude into that possible world until it transforms from phantom to entity.''
\end{quote}

\subsubsection{Weight Distribution of Wave Functions}

First, we need to physically define what \textbf{``confidence''} is.

In Hilbert space, the future is in a superposition state:

\[|\text{Future}\rangle = c_1 |\text{Success}\rangle + c_2 |\text{Failure}\rangle + c_3 |\text{Mediocrity}\rangle + \dots\]

Here, $|c_n|^2$ represents the probability of that outcome occurring.

For a confused ordinary person, their $v_{int}$ (internal intention) is scattered. Their confidence in success ($|c_1|^2$) might be only 10\%, while fear of failure ($|c_2|^2$) accounts for 50\%.

Such a wave function is \textbf{``mediocre''}, lacking clear directionality. It will be arbitrarily manipulated by environmental thermal noise ($v_{env}$).

But for an observer with \textbf{absolute confidence} (like Jobs or Elon Musk), the situation is completely different.

\textbf{Confidence is betting the entire modulus of $v_{int}$ on one specific basis.}

\begin{itemize}
\item They construct a future model $|\text{Vision}\rangle$ in their mind.

\item They invest all $c_{FS}$ budget to maintain high coherence of this model.

\item Result: $c_{\text{vision}} \to 1$, while coefficients of all other possibilities $c_{\text{others}} \to 0$.
\end{itemize}

Geometrically, this is a \textbf{``Spike''}.

They create a \textbf{``Dirac $\delta$ function''} on the probability landscape of the future. They tell the universe: \textbf{``Only this future is allowed.''}

\subsubsection{Anchor Effect: The Gravitational Source of the Future}

What physical consequences does this extreme probability distribution produce?

In Chapter 4, we discussed ``attractors.'' Usually, attractors form naturally (like heat death).

But confidence creates an \textbf{``Artificial Attractor''}.

When you are extremely certain something will happen, you place a \textbf{``phase anchor''} at that future time point $t_{future}$.

This anchor is not just a marker; it is a \textbf{``negentropy source''}.

\begin{itemize}
\item \textbf{For present reality ($t_{now}$)}: This anchor generates a reverse \textbf{``Tension''}.

\item \textbf{Response of the principle of least action}: The universe always tends to take the path with ``fastest phase accumulation.'' Because you inject huge weight (confidence) on the target path, the \textbf{``optical length''} of this path becomes shorter.
\end{itemize}

Reality begins to \textbf{distort}.

Originally random events begin mysteriously converging in directions favorable to you.

\begin{itemize}
\item Coincidences increase (synchronicity).

\item Resistance decreases (Hamiltonian bias).

\item Resources begin flowing toward you.
\end{itemize}

This is not magic; this is \textbf{``redirection of probability flow''}.

Like digging a deep pit (confidence anchor) downstream, upstream water (reality) naturally accelerates toward it.

\textbf{Confidence is computing power.}

The higher the clarity you maintain of that vision in your brain (the more $c_{FS}$ consumed), the greater the anchor's mass, the stronger its gravitational pull on reality.

\subsubsection{The Art of Refusal: Pruning Wave Functions}

Another important function of confidence is \textbf{``refusal''}.

When reality gives unexpected feedback (such as setbacks), ordinary people accept the observation results, causing the wave function to collapse to ``failure.''

But those with absolute confidence execute \textbf{``Refusal to Collapse''}.

They say: ``This is just temporary fluctuation, not the final result.''

They refuse to acknowledge the legitimacy of the ``failure'' basis.

Through this \textbf{``cognitive filtering''}, they forcibly maintain the wave function's superposition state until reality finally gives the result they want.

This is a \textbf{``Reality Distortion Field''}.

It is essentially the observer using their own powerful $v_{int}$ stability to \textbf{``override''} the randomness of the external world $v_{ext}$.

\subsubsection{Conclusion: To Believe is to See}

So, don't just see confidence as an emotion.

Confidence is a \textbf{``geometric operation''}.

It is a \textbf{``long-range navigation setting''} you perform in Hilbert space.

\begin{itemize}
\item \textbf{Doubters}: are random walking.

\item \textbf{Believers}: are flying along geodesics.
\end{itemize}

\textbf{Prophecies can self-fulfill because prophecies themselves are signposts to that future.}

When you believe in a future with all your life force, you have actually \textbf{``pulled''} that future from nothingness to the edge of reality.

Since we have placed anchors and locked direction, how exactly is reality ``dragged'' by this force? How does this ``pulling'' process physically overcome inertial resistance?

This leads to the theme of the next section: \textbf{Pulling Reality}. We will see that confidence is not passive waiting but an active, continuous \textbf{topological traction}.

