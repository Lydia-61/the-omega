\subsection{5.1 Novikov Self-Consistency Principle}

\begin{quote}
``The universe allows you to go back to the past, even allows you to point a gun at your grandfather. But the universe absolutely will not allow you to pull the trigger. If you try to pull it, the gun will jam; if you fix the gun, you will slip; if you aim, a meteor will fall to block the bullet. Physical laws will distort all probabilities, just to prevent the logical chain from breaking. History is not fragile paper; it is memory metal with self-healing ability.''
\end{quote}

\subsubsection{The Geometry of Paradox: Non-Closed Paths}

In classical logic, the grandfather paradox is a \textbf{``logical singularity''}.

\[A \implies \text{not } A \implies A\]

Geometrically, this corresponds to a \textbf{Möbius Strip}-like twisted path: you walk a circle and find yourself flipped (existence becomes non-existence).

But from the perspective of \textbf{path integrals}, cosmic evolution is a weighted sum of all possible histories.

\[K = \sum e^{iS}\]

For paths containing logical paradoxes (such as ``killing grandfather''), their physical action $S$ causes the wave function phase to undergo violent \textbf{Destructive Interference}.

This means: \textbf{The probability of paradoxical paths occurring is zero.}

It's not that ``someone forbids you,'' but that ``that path doesn't work.''

In Hilbert space, historical vectors containing paradoxes are \textbf{linearly dependent}; they cancel each other out in superposition. Only those closed-loop paths that are end-to-end and logically self-consistent can survive in the summation.

\textbf{Conclusion:} The universe only computes self-consistent solutions.

\subsubsection{Distortion of Probability: Anything for Self-Consistency}

So, if you really go back to the past and try to kill someone, what happens in the macroscopic physical world?

Russian physicist Igor Novikov proposed: \textbf{Physical laws will intervene in an extremely bizarre way.}

If you try to create a paradox, you are fighting against the \textbf{coherence} of the entire universal wave function.

To maintain unitarity (information conservation), the universe must mobilize all $c_{FS}$ budget to stop you.

\begin{itemize}
\item \textbf{Microscopic level}: The electron tunneling probability in your neurons will shift, causing you to suddenly not want to kill at that moment, or your hand to shake.

\item \textbf{Macroscopic level}: If the uncertainty principle is not enough, macroscopic probabilities will distort. Gun malfunction rates will soar from 0.01\% to 100\%. Even if you insist on killing, an earthquake might shake you away.
\end{itemize}

\textbf{This looks like ``fate's mischief,'' but it's actually an extreme manifestation of the ``principle of least action.''}

For the universe, the ``energy cost'' of creating one gun jam (low-probability event) is far lower than the cost of letting spacetime logic collapse (infinite cost).

So, the universe unhesitatingly chooses the former.

\textbf{You cannot defeat probability.} When you try to violate causality, you are fighting against the entire universe's wave function.

\subsubsection{The Boundary of Free Will}

Does this mean we have no free will?

No. In \textbf{Vector Cosmology}, free will is redefined as \textbf{``the right to choose within the self-consistent solution space''}.

You can go back to the past.

You can shake hands with your young grandfather, you can give him money, you can even fall in love with someone from that era.

Because these behaviors \textbf{do not change} the result of ``your birth,'' and may even be \textbf{necessary conditions} for your birth (this will be detailed in the next section ``Bootstrap'').

\begin{itemize}
\item \textbf{You can do anything, as long as it doesn't cause logical contradiction.}

\item \textbf{You cannot do anything if it tries to cut off the foundation of your existence.}
\end{itemize}

It's like playing an open-world game. You can go anywhere, but you cannot pass through invisible walls, nor can you delete critical system files.

\textbf{The Novikov principle is the universe's ``crash protection mechanism.''}

\subsubsection{Conclusion: The Resilience of History}

At this point, we have a new reverence for history.

History is not fragile porcelain that shatters at a touch.

History is \textbf{non-Newtonian fluid}. When you gently stroke it (conform to causality), it is soft; when you strike it hard (create paradoxes), it is harder than steel.

We can safely aspire to the future because the past is locked in a solid self-consistent closed loop.

We don't need to worry about accidentally destroying ourselves. Because if that destruction really happened, we wouldn't be here now worrying about it.

Since we cannot destroy the past, what exactly is our relationship with the past?

What if not only can we not kill grandfather, but our very existence is the reason grandfather could survive?

This leads to the theme of the next section: \textbf{The Closed Loop of Existence}.

We will see the most insane logical structure---\textbf{Bootstrap}.

It's not that grandfather gave birth to you, but that \textbf{future you, traversing spacetime, ensured that grandfather gave birth to you}.

You are mutually causal.

