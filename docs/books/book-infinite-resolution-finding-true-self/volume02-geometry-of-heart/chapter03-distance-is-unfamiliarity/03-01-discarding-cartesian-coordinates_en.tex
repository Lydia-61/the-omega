\subsection{3.1 Discarding Cartesian Coordinates}

When we try to describe ``where I am,'' our intuitive response is to take out a map or open GPS. We get a set of precise numbers: longitude, latitude, altitude. In physics, this is formalized as Cartesian coordinates $(x, y, z)$, plus time $t$, forming a point on the Minkowski four-dimensional manifold.

This coordinate system is based on an ancient assumption: \textbf{Locality}. It implies that the universe is a vast container, with objects placed in different cells, separated by empty distances. If you want to influence point B from point A, you must traverse every inch of space in between, and the speed cannot exceed light speed.

However, with the deep integration of quantum information theory and gravitational theory (AdS/CFT correspondence), we discover that this ``container''-based spacetime view is merely an \textbf{Emergent Phenomenon} at macroscopic scales. At the underlying ontological level, Cartesian coordinates are invalid.

Now, let us dismantle this container and establish an \textbf{information coordinate system} for the awakened.

\subsubsection{The Illusion of Locality: The Revelation of Bell's Inequality}

As early as 1964, John Bell proved through his inequality that any hidden variable theory based on ``local realism'' cannot explain the experimental results of quantum entanglement.

When two particles A and B are in an entangled state, regardless of how far apart they are in geometric space (even across galaxies), measuring A instantly determines B's state.

From a classical geometric perspective, this is ``spooky action at a distance.''

But from an information ontology perspective, this reveals a simple truth: \textbf{A and B, though separated geometrically, have never been separated topologically.}

They share the same wave function in Hilbert Space. The so-called ``distance'' is merely an illusion we see in low-dimensional projections. Like two ink dots on a piece of paper, they appear far apart on a two-dimensional plane, but if you fold the paper, they touch.

\textbf{The universe is a piece of paper folded extremely complexly by quantum entanglement.}

\subsubsection{Geometry from Entanglement: Weaving Tensor Networks}

If space is not a container, what is it?

Modern physics answers: \textbf{Space is a product of entanglement.}

In cutting-edge research on the holographic principle, physicists discovered that the continuous geometric structure of spacetime is woven by underlying qubits through \textbf{Tensor Networks} (such as MERA).

\begin{itemize}
\item If entanglement between qubits is cut, spatial geometry collapses and connectivity disappears.

\item As long as entanglement is established, space is connected like stitching a wound.
\end{itemize}

This directly leads to the famous \textbf{Ryu-Takayanagi formula}, equating geometric quantity (area) with information quantity (entanglement entropy):

\[S_A = \frac{\text{Area}(\gamma_A)}{4G}\]

This means that a region's \textbf{Entanglement Entropy (Information)} directly determines its \textbf{Boundary Size (Geometry)}.

\subsubsection{Redefining Distance: The Inverse of Mutual Information}

Based on this, we can discard $(x, y, z)$ and give a physical definition of \textbf{``essential distance.''}

The distance $d(A, B)$ between two entities A and B does not depend on how long light takes to travel, but on how much information they share, i.e., \textbf{Mutual Information ($I$)}.

\[d(A, B) \propto -\ln \left( \frac{I(A:B)}{I_{max}} \right)\]

\begin{itemize}
\item \textbf{When $I(A:B) \to 0$} (mutual information is zero, completely unfamiliar):

    $\ln(0) \to -\infty$, distance tends to infinity. This is why for someone without any astronomical knowledge, the Andromeda Galaxy is an unreachable ``alien world.'' Because there is no quantum correlation between their brain (internal state) and Andromeda (external state).

\item \textbf{When $I(A:B) \to I_{max}$} (complete entanglement, complete understanding):

    $\ln(1) = 0$, distance tends to zero. This is the physical meaning of \textbf{EPR=ER} (entanglement=wormhole). When entanglement reaches its extreme, spacetime undergoes topological short-circuiting, forming an Einstein-Rosen bridge, and A and B coincide physically.
\end{itemize}

\subsubsection{Conclusion: Distance is Unfamiliarity}

This new coordinate system fundamentally changes our understanding of ``travel'' and ``exploration.''

In this universe, \textbf{``far'' does not mean physically distant, but informationally ``unfamiliar.''}

You feel someone is far from you, not because they live on the other side of the Earth, but because you don't understand their thoughts or feel their pain. Your wave functions are orthogonal.

Conversely, \textbf{``near'' means ``familiar'' and ``resonant.''}

When you deeply understand the event horizon equation of black holes through learning (increasing knowledge graph connections), when you establish empathy with distant stars through meditation (adjusting consciousness frequency), you have already \textbf{moved} in the information coordinate system.

Therefore, breaking free from Cartesian coordinates is the first lesson for the awakened.

Don't ask ``where am I.''

Ask \textbf{``what am I connected to.''}

Your coordinates are the sum of all your connections. The size of your world strictly equals the scope of your love and understanding. In this sense, a closed-off billionaire may be imprisoned on a Planck-scale island, while a poet gazing at the stars may already occupy half a galaxy.

