\subsection{2.1 The Universal Projector}

When we look up at the night sky, the most lonely physical fact is \textbf{Redshift}.

In 1929, Edwin Hubble discovered that almost all distant galaxies are moving away from us. The farther away, the faster they recede. Light waves are stretched, shifting toward the red end of the spectrum. This discovery laid the foundation for the Big Bang theory, but also established a despairing cosmic picture: \textbf{dispersion and decay}. In this picture, the universe is like a debris field after an explosion, all matter irreversibly drifting apart, eventually dying alone in cold and darkness.

However, from the information ontology perspective of \textit{The Infinite Resolution}, we must perform a revolutionary \textbf{Gestalt Switch} on ``redshift.''

What if redshift is not viewed as spatial ``separation,'' but as the \textbf{``unfolding''} of information?

\subsubsection{High-Frequency Noise of the Singularity}

Let us return to the moment of singularity at $T=0$.

In standard cosmology, that was a point of infinite heat and infinite density. In information physics, that was an information source with \textbf{infinite frequency}.

Imagine if all movements and notes of Beethoven's Ninth Symphony were superimposed and played within $10^{-43}$ seconds (Planck time).

\begin{itemize}
\item \textbf{Physically}: This would be an extremely high-energy shock wave.

\item \textbf{Informationally}: This would be completely unparseable \textbf{White Noise}.
\end{itemize}

Although this contains everything (omnipotence), for any finite observer, it is equivalent to ``nothing.'' Because the information density exceeds the bandwidth limit of any receiver. \textbf{Meaning that is too dense is equivalent to meaninglessness.}

For this symphony to be ``heard,'' for the data in this compressed file to be ``read,'' the universe must perform an operation: \textbf{Down-conversion}.

\subsubsection{Expansion as Magnification}

The expansion of cosmic space is essentially a \textbf{Zoom Lens Projector}.

According to general relativity, wavelength $\lambda$ grows linearly with the cosmic scale factor $a(t)$:

\[\lambda(t) \propto a(t)\]

This means that the physical function of cosmic expansion is to stretch high-frequency quantum fluctuations originally curled up at \textbf{subatomic scales (microscopic)} into material structures spanning \textbf{billions of light-years (macroscopic)}.

\begin{itemize}
\item \textbf{Cosmic Microwave Background (CMB)}: This is the universe's oldest light. At 380,000 years after the Big Bang, its temperature was 3000K (visible light band). After 13.8 billion years of expansion, its wavelength has been stretched 1100 times, becoming microwaves. This is not merely energy cooling; this is \textbf{information magnification}. The cosmic microwave background map we see today is actually a \textbf{giant magnified photograph} of the early universe's microscopic quantum foam.

\item \textbf{Galaxy Distribution}: The large-scale galaxy walls and void networks we see today were seeded by microscopic quantum fluctuations during inflation.
\end{itemize}

\textbf{Theorem 2.1.1 (Holographic Decompression Theorem)}:

The macroscopic structure of the universe is the \textbf{Conformal Magnification} of microscopic initial conditions.

Redshift $z$ does not represent ``distance,'' but \textbf{magnification factor}.

\[z + 1 = \frac{a(t_{now})}{a(t_{then})}\]

The farther you look ($z$ larger), the more you are actually looking at a more magnified \textbf{microscopic slice}.

\subsubsection{Why Do We Need a Vast Universe?}

This answers a question that has troubled humanity for millennia: \textbf{If God only cares about life on Earth, why create a universe 93 billion light-years in diameter, filled with hundreds of billions of galaxies? Isn't this a huge waste?}

Under ``holographic decompression'' theory, this is by no means waste, but \textbf{necessity}.

If you want to unfold a highly compressed file containing $10^{90}$ bits (total cosmic information) at a resolution readable by human eyes (low frequency), you \textbf{must} need a sufficiently vast screen.

\begin{itemize}
\item If the screen is too small (universe does not expand), all galaxies would be squeezed together, all history would overlap in an instant, and we would be burned by high-energy radiation and crushed by information overload.

\item We need this 93 billion light-years of void as an \textbf{information buffer}.
\end{itemize}

Every distant galaxy is a pixel on the holographic film, projected onto the celestial screen by the spacetime projector. Their existence is to allow us to read those source codes about truth at a \textbf{``safe resolution.''}

\subsubsection{Conclusion: Separation is for Reunion}

So, do not feel sad because of redshift.

The galaxies' departure is not them abandoning us. That is \textbf{the universe spreading open its pages}.

When you see a star redshifted to the edge of vision, it means the high-frequency information it originally carried has finally been stretched to a band you can understand.

\textbf{Expansion is the prerequisite for understanding.}

Without this physical ``separation,'' there can be no cognitive ``intimacy.'' We must first separate physically before we can reunite in meaning.

